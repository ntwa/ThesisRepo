% Chapter 1

\chapter{Conclusions and Future Research} % Main chapter title

\label{discussionchapter} % For referencing the chapter elsewhere, use \ref{Chapter1} 

\lhead{Chapter \emph{Conclusions and Future Research}} % This is for the header on each page - perhaps a shortened title

%----------------------------------------------------------------------------------------
The main research questions were centred around factors that may affect utilization of a personal health informatics through intermediaries, and also the effectiveness of gamification in increasing engagement of intermediaries. This chapter expands on answers to the research questions. On providing answers to research questions this chapter also revels design considerations for the overall intervention. These design considerations include some of the aforementioned social factors that may contribute to the success of the intervention, and approaches that can be utilized in order to keep both intermediaries and beneficiaries engaged with a personal health informatics system.

\section{Discussion on Research Questions}
The main focus of this research was to uncover how social factors and motivational affordances could impact intermediated use of personal health self-monitoring applications. Intermediated technology use is an interaction model that is prevalent in contexts of low-income communities of developing world. Many health self-monitoring application are designed for personal use with their motivational affordances targeting only direct users; hence such motivational affordances have not been explored in the context of intermediated use. Therefore, in order to understand design requirements for motivation affordances targeting intermediated use this research aimed at providing answers for the following research questions.

\textbf{Research Question 1}: What is the role of social-technical settings in intermediated use of a gamified self-monitoring application targeting promotion of healthily eating and physical activity? 

The aforementioned research question had two sub-questions. The first question aimed at identifying prerequisite factors for intermediated use and the second question aimed at exploring the extent to which the identified factors are important in facilitating intermediated use. In order to provide answers to those sub-research questions, a series of studies were conducted. These studies included: one contextual enquiry and two consecutive evaluations of two versions of the prototype.  
 
The most prominent factors were social relationships, collaborative reflection from one shared device, and app's motivational affordances or socially construed motivational  affordances as the result of using the app. Perception on the value of motivational affordance (from the app and those that were socially construed) differed between members of intermediaries' set and beneficiaries' set. Social relationship and perceived motivation affordances were main determinants for how the users from two sets would proceed with interaction. Before an interaction took place there was always a negotiation for interaction. Negotiations would be initiated within a pair by either member of a pair. 

There two options of how negotiation for interaction were initiated. In the first option is where the a negotiation was started by a beneficiary user then an intermediary user would proceed with an interaction or would reject a request for interaction from a beneficiary user. In most cases of where pairs consisted of a parent working with a child, an intermediary had a tendency of realizing a rationale in fulfilling such requests. The prior social relationship between an intermediary and a beneficiary is a good starting point to guarantee success out of negotiation for interaction. Then the second point an intermediary has to be excited to motivational affordances provided by the app.  In the second option of how members of a pair negotiated to each other about initiation of interaction is where intermediary users were reach out to beneficiary users. In interactions that were initiated in this form, intermediaries were triggered by motivational affordances from the app. Motivation of intermediary users contributed to a large extent to the success of the intervention and this motivation is of two-folds; social relationship provides a relaxed environment for two sets of users to negotiate for interaction, and motivational affordances provide a an environment for an intermediary user to be excited about the interaction.

Informative evaluations revealed that involving children who are family members is the key to success of this kind of an intervention. Having a shared device increased tendency of of member of a pair to reflect collaboratively. In such context it was no longer just the matter of help seeking and help giving but more of a collaborative effort towards a joint goal. The idea of collaborative interfaces for health information within family settings has been explored in computer supported collaborative work (CSCW) literature. \cite{colineau2011motivating} designed a system to support a family to select a collective health goal and receive feedbacks that entailed comparisons between families. Their system was found to encourage members from within a family or members of different families, to work together and in particular to help each other in finding ways to live a healthily lifestyle. 

Family settings provide an idyllic opportunity for members to discuss healthy issues collaboratively. Collaboration between a parent and a child or close family members had a positive impact on child's perception as some intermediaries shared testimonies about their habituation of skills on eating healthy. In addition, intermediaries in some cases logged their data about meals because what they ate was not different from what had been eaten by their respective beneficiaries. A study by~\cite{grimes2009toward} identified four key areas of consideration in which sharing of, and reflection on, health information can be leveraged within family context as follows: (1) overlaps of routines between family members through shared meals, space, etc which can provide opportunities for collaborative data logging and reflection among family members; (2) sharing is done at the expense of balancing competing values of openness, caring, and modelling with the value of protection; (3) understanding of sensitivity on comparisons and competition based upon health information in the context of the family as it may have negative consequences; and (4) collaborative sharing of, reflecting on, health information can also foster family's bond. In the context of this research, it was evident that the app had increased the bond between participating family members as majority of them claimed that were interacting more often. This is also demonstrated by playfulness behaviours that were exhibited in the process of sharing information as it was shown in one of the excerpt in chapter \ref{prototytpe2chapter} (Evaluation of Prototype II):

\userquote{\textbf{Zandiwe}, a beneficiary} {` When she got time, when she is done with her homework she comes and sees the app. And then laughs at me like `Yo yo yo [An interjection for Xhosa speakers to express the feeling of amazement by something] you can walk yo yo yo', like `you walked a lot today' and what what [She was implying to other words said by Lindiwe]''}

In existing work from computer supported collaborative work it appears the emphasis is on parents trying to model health behaviors of their children.  For a instance in a study by ~\cite{saksono2015spaceship}, a collaborative exergame was developed in order to support both parents and kids to exercise together. Although their goal was to help kids learn from their parents, the collaborative environment was beneficial to both parents and children.

In our case it was peculiar that children were attempting to nudge their parents to live healthily. Therefore, it not only about the parent guiding the child also the child can become a facilitator for guiding the parent about health choices. This was mediated by an existing familial relationship. 

Therefore, this work continues to emphasize on the value of familiar relationships in making the collaboration more interesting. Some playful visualization techniques can make the collaboration between two users more enjoyable. In the next section, ways on which one could improve engagement of both sets of users are discussed.   

\textbf{Research Question 2}: What is the role of gamification in motivation to intermediated use of self-monitoring application targeting promotion of healthily eating and physical activity? 

To answer this research question, a summative evaluation was conducted that compared between a system without gamification and a system with gamification. Gamification demonstrated a potential too increase frequency of usage through intermediaries even though it has some challenges and limitation that need to be addressed as gamification appeared to affect both intrinsic motivation and internalization. 

The next subsection expands on the discussion on answers for the two aforementioned research questions.
 
\subsection*{Leverage Family Settings}


\subsection{Sustaining Engagement}
In order to enhance user experience, support for factors such as task mastery, support for reflection, enhancement of collaboration within a family (intra-families), or inter-families collaboration are emphasized.
\subsubsection*{Support for Autonomy}
\subsubsection*{Support for Task Mastery Features}
In this work, it was clear that gamification promoted collaboration between participating pairs that had a prior social rapport. In addition, gamification in particular social comparison and competitions, both the one socially construed by users, and the one implemented as an intentional design goal,   increase engagement of both intermediary and beneficiary users.

Despite the positive outcomes from gamification, the negative consequences as the results of increased competition have become an area of concern~\citep{jia2016personality}. It is even more concerning when such competition results into negative consequences in health settings~\citep{grimes2009toward}. To reduce these negative implications of excessive social comparison and competition, there is an emphasis on supporting challenges on the level of ``\emph{task mastery climate}'' rather than on competition that has ``\emph{ego involved climate}'' as the  former foster intrinsic motivation while the latter can harm it~\citep{saksono2015spaceship}. When ``ego is involved'', participants may do things just to maintain their self-worth, and this is equivalent to introjected regulation as postulated by organismic integration theory of SDT\citep{ryan2000:self}. In introjected regulation individuals don't see a value in regulating a behaviour rather they perform it merely for the purpose of outdoing others or maintaining their social status. For instance, the idea of having a leader board in this intervention encourages competition and it affected motivation of intermediary users who didn't do so well. Features such as fish tank and botanical garden appear to promote task mastery. For instance in one scenario reported in chapter \ref{prototytpe2chapter} (Evaluation of Prototype II),  a beneficiary user was dissatisfied by the look of their garden.


\userquote{\textbf{Lulama}, an intermediary} {``Nokhanyo saw the garden. The first day she saw just the house and brownish. She
is like `What is this'. I told her. She said `Aha! [Expressing
dissatisfaction]. It must look green and healthy'. And then
she saw the garden again and said `It is looking good.'''}

This important finding suggest that such features can encourage task mastery climate. If designers have to use a leaderboard, they need to be cautionary of negative impacts on users' competence despite its ability to foster relatedness~\citep{sailer2013:psychological}. Leader board can also result into an extreme competition between intermediaries which can result into a negative impact on a relationship by an intermediaries in cases where an intermediaries feel of being let down by their beneficiaries. Such a scenario is exhibited in the following excerpt on chapter \ref{summativeevalchapter} (Summative Evaluation)
 
\userquote{\textbf{Jenner}, a female beneficiary from Athlone, 45 yrs old} {``Sometimes may be I forget to take the phone when I go walking and he would ask me `did you take the phone with you' Ooh Gosh I forgot.  Because when I walk to Park Town to exercise and sometimes  I am in such a hurry I forget the phone, he will be crossed with me.''} 

In the aforementioned case, an intermediary got angry because of her mother tendency of forgetting to take the pedometer (phone) when she goes out for walking. Therefore this highlights the importance of paying more attention should on features that support task mastery climate.

\section{Long-term Benefits of Personalized Apps for Health} 
Long term usage and prolonged benefits of personalized apps are still debatable. One study that a conducted a two years trials that compared among three strategies: interactive smartphone application on a cell phone (CP); personal coaching enhanced by smartphone self-monitoring (PC); or handout (pamphlets) as the control group found that smart-phone to be better than control in short time but in long term the results were not different~\citep{svetkey2015cell}. It is argued that usage of these apps should go hand in hand with other conventional strategies. In addition,  long  term engagement is a key challenge in personalized apps for health. Once the novelty effect is worn out users may tend to revert to their old habits. Gamification is not exceptional when it comes to the issue of the novelty effect. In the context of intermediated technology it may even be more challenging in sustaining engagement. This is still a grey area that needs further exploration.
 
\begin{flushright}
\end{flushright}
