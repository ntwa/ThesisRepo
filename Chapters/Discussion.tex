% Chapter 1

\chapter{Overall Discussion} % Main chapter title

\label{discussionchapter} % For referencing the chapter elsewhere, use \ref{Chapter1} 

\lhead{Chapter \emph{Overall Discussion}} % This is for the header on each page - perhaps a shortened title

%----------------------------------------------------------------------------------------
In this chapter, design considerations for the overall intervention are discussed. These design considerations include social factors that contribute to the success of the intervention, and approaches that can be utilized in order to keep both intermediaries and beneficiaries engaged with a personal health informatics system.  
\section{Leverage Family Settings}
Informative evaluations revealed that involving children who are family members is the key to success of this intervention. This idea of collaborative interfaces for health information within family settings has been explored in computer supported collaborative work (CSCW) literature. \cite{colineau2011motivating} designed a system to support a family to select a collective health goal and receive feedbacks that entailed comparisons between families. Their system was found to encourage members from within a family or members of different families, to work together and in particular to help each other in finding ways to live a healthily lifestyle. 

Family settings provide an idyllic opportunity for members to discuss healthy issues collaboratively. Collaboration between a parent and a child or close family members had a positive impact on child's perception as some intermediaries shared testimonies about their habituation of skills on eating healthy. In addition, intermediaries in some cases logged their data about meals because what they ate was not different from what had been eaten by their respective beneficiaries. A study by~\cite{grimes2009toward} identified four key areas of consideration in which sharing of, and reflection on, health information can be leveraged within family context as follows: (1) overlaps of routines between family members through shared meals, space, etc which can provide opportunities for collaborative data logging and reflection among family members; (2) sharing is done at the expense of balancing competing values of openness, caring, and modelling with the value of protection; (3) understanding of sensitivity on comparisons and competition based upon health information in the context of the family as it may have negative consequences; and (4) collaborative sharing of, reflecting on, health information can also foster family's bond. In the context of this research, it was evident that the app had increased the bond between participating family members as majority of them claimed that were interacting more often. This is also demonstrated by playfulness behaviours that were exhibited in the process of sharing information as it was shown in one of the excerpt in chapter \ref{prototytpe2chapter} (Evaluation of Prototype II):

\userquote{\textbf{Zandiwe}, a beneficiary} {` When she got time, when she is done with her homework she comes and sees the app. And then laughs at me like `Yo yo yo [An interjection for Xhosa speakers to express the feeling of amazement by something] you can walk yo yo yo', like `you walked a lot today' and what what [She was implying to other words said by Lindiwe]''}

In existing work from computer supported collaborative work it appears the emphasis is on parents trying to model health behaviors of their children.  For a instance in a study by ~\cite{saksono2015spaceship}, a collaborative exergame was developed in order to support both parents and kids to exercise together. Although their goal was to help kids learn from their parents, the collaborative environment was beneficial to both parents and children.

In our case it was peculiar that children were attempting to nudge their parents to live healthily. Therefore, it not only about the parent guiding the child also the child can become a facilitator for guiding the parent about health choices. This was mediated by an existing familial relationship. 

Therefore, this work continues to emphasize on the value of familiar relationships in making the collaboration more interesting. Some playful visualization techniques can make the collaboration between two users more enjoyable. In the next section, ways on which one could improve engagement of both sets of users are discussed.

\section{Sustaining Engagement}
This collaboration was fostered by gamification.

\begin{flushright}
\end{flushright}
