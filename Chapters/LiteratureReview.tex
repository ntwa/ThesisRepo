% Chapter 1

\chapter{Literature Review} % Main chapter title

\label{literaturereview} % For referencing the chapter elsewhere, use \ref{Chapter1} 

\lhead{Chapter \ref{literaturereview}.\emph{Literature Review}} % This is for the header on each page - perhaps a shortened title

%----------------------------------------------------------------------------------------
\section{Behaviour Change Support Technologies}
B.J. Fogg, one of the early pioneers to formalize behaviour change technology as an area of research, coined a term “captology” which is an acronym for \emph{Computers As Persuasive Technologies} (CAPT-ology), with focus on the planned persuasive effects of computer technologies~\citep{fogg1999persuasive}. In persuasive systems, persuasion is intentional and usually implemented through persuasive stimuli; hence providing a system with the ability to persuade~\citep{hamari2014persuasive}. Persuasive technologies have applications in domains such as health-care, education and training, environmental sustainability, etc..

Research on behaviour change technologies in perspective of computing has evolved from studying digital interventions in early 1990s which were basically for intervening behaviours in the preventive health area primarily through reminders; to persuasive technologies - systems that implement various software functionality that utilize approaches such as social learning or comparison etc; to the current behaviour change support systems (BCSS) which provide models and frameworks for designing and evaluating persuasive technologies~\citep{langrial2012digital}.~\cite{Oinas-Kukkonen:foundation} defined a BCSS  as ``a socio-technical information system with psychological and behavioural outcomes designed to form, alter or reinforce attitudes, behaviours or an act of complying without using coercion or deception''.

Separate Models to guide design and evaluation of persuasive technologies have been proposed due to limitations in models from information systems such \emph{Technology Acceptance Model} (TAM)~\citep{Oinas-kukkonen:psd}.~\cite{fogg2009behavior} recommended a behaviour model for persuasive design which asserts that for a person to perform a targeted behaviour, he or she must (1) be sufficiently motivated, (2) have the ability to perform the behaviour, and (3) be triggered to perform the behaviour.~\cite{fogg2009behavior2} also recommended a behaviour grid that one can use to design persuasive technologies. In this behaviour grid, persuasive strategies are matched to targeted behaviours. 

\cite{Oinas-kukkonen:psd} extended Fogg's work~\citep{fogg2009behavior} with a more comprehensive model known as a \emph{Persuasive System Design} (PSD) model which suggested three initial steps: (1) analysis of the persuasion context, i.e. with focus on the intent of persuasion and context of use, user, and technology; (2) selection of persuasive features to use; and (3) selection of persuasion strategies to use i.e. whether to use a direct or indirect route of persuasion. The PSD model also outlined 28 design principles discerned into the following five categories: (1) \emph{primary task support}, which includes activities such as, reduction of complex behaviours into simple tasks, guiding the user through experiences while persuade along the way, tailoring of persuasive information to factors relevant to a user group, personalization of content, and self-monitoring for users to keep track of their performance towards their specified goals; (2) \emph{dialogue support}, which includes praises, rewards, reminders, similarity, liking, and social roles; (3) \emph{system credibility support}, which includes trustworthiness, expertise, surface credibility etc; and (4)\emph{social support}, which includes social learning, social comparison, and competition.

An extension to the PSD model was an Outcome Change (O/C) matrix of which one can use when analysing an intent of persuasion~\citep{Oinas-Kukkonen:foundation}. The O/C matrix matches the type of change that needs to be applied with a specific outcome. A change could either be compliance (C) or behaviour (B) or attitude (A), change. While an expected outcome could be forming, altering, or reinforcing any of the aforementioned types of change. The extended PSD model with O/C matrix is called BCSS as mentioned in the classes of behaviour change systems above. BCSS is considered to be the foundation for studying persuasive systems and it is meant to provide a base for analysis, design, and evaluation of persuasive technologies. 

\section{Behaviour Change Technologies for Health}
Health promotion and management drive most initiatives to design persuasive technologies due to an unprecedented increase in prevalence of lifestyle-chronic diseases. A systematic review by~\cite{hamari2014persuasive} on ability of persuasive technologies to persuade, included 95 studies of which approximately 47\% targeted health and exercise domains alone. This shows that health is an important area of concern when the notion of persuasive technologies comes in mind.

\cite{chatterjee2009healthy} classified three generations  of technological evolution of hardware and software utilized in implementations of behaviour change interventions in health. The first generation started to emerge from 1960's and it was characterized by the prescriptive nature of information flow from physician, health care provider, or technology-based system to a health care recipient. Decades worth of research has shown that phone-based or simple messaging technologies can improve the quality of health care management and clinical outcomes. The second generation is characterized by the descriptive nature of information interaction between a user and the persuasive technologies and examples of such systems include interactive Web sites, personal data assistants (PDAs) that allow activity recording, and simple sensors that record and report basic health parameters. The third generation extends second generation by providing body-wearable sensors that support advanced health monitoring, use of context-aware computing that can use information of a person's location within their environment and the determination of their activity at the moment of measurement, and real-time exchange of information to support “just in time” messaging.  While the second generation utilized PCs and later cellphones, the third generation is dominated mostly by cellphones.

Digital interventions are the ones that have received most appraisal because of existing randomized clinical trials. The evidence of their dominance in public health is demonstrated by the preponderance in publications that report on the use web based interventions integrated with SMS text-messaging on clinical settings. Existing systematic reviews~\citep{cole2010text,fjeldsoe2009behavior,krishna2009healthcare} report more on the use of SMS reminders and feedbacks on diabetes self-management, smoking cessation, and weight reduction therapy. However, published literature on digital interventions is being criticized of lacking adequate information on how individual systems were designed, hence such systems are usually poorly described since most work is being published by public health practitioners without involvement of computer scientists~\citep{Oinas-Kukkonen:foundation}.

The main advantage of behaviour change interventions in health is their ability to allow personalization of health information which is advocated by literature since it allows consideration of individual needs of a person and it also gives a targeted person, a  sense of control over their healthcare~\citep{mccallum2012gamification}. This research was focusing on personalized technologies that support data collection and feedback for an objective of health persuasion. These systems are referred to as wellness applications or personal health informatics. 

\section{Personal Informatics for Health Behaviour Change}
Personal informatics is a class of interactive applications that support users to improve self-understanding of various aspects of their life by providing technological means that allow individuals to both collect and analyse personal data related to habits, behaviours, and thoughts~\citep{li2011personal,li2012personal}. A personal informatics system augments the activity of \emph{self-reflection} by complementing individuals in storing events that can hardly be recalled due to limitations in humans' memory~\citep{li2010stage}. The goal of personal informatics systems is to support individuals in having a better understanding of their lifestyle or behaviours. These systems are important in promotion of positive behaviors such as healthy eating~\citep{lee2006pmeb}, recycling~\citep{comber2013designing}, energy conservation~\citep{seligman1977feedback}, etc..

Research on personal informatics systems tends to focus on, effective ways of collecting personal data that require less efforts from users perspective, and optimal support for self-reflection through feedback mechanisms~\citep{li2011understanding}. Support for self-reflection is usually achieved through data visualizations techniques such as bar charts or more affective formats that use abstract visualizations (i.e. virtual pets~\citep{lin2006:fish}, informal art displays~\citep{fan2012spark}, etc.). There are also systems that have implemented social comparison features or competitions with others through social networks i.e. BinCam~\citep{comber2013bincam,comber2013designing} which uses social norms influence as a motivation strategy to encourage individuals within a household to be more conscious of their recycling behaviours by comparing themselves with other households.

Models and frameworks for designing personal informatics have been proposed.~\cite{kamal2010understanding} presented a framework for designing a system that integrates online social networks and personal informatics to promote positive health behaviour. The framework was informed by theories from both health behaviour change and social networks.~\cite{li2010stage} proposed a model for understanding how people use personal informatics by transitioning between the following five stages: preparation, collection, integration, reflection, and action. The importance of identifying barriers at each stage is emphasized since these barriers may also cascade to later stages to hinder the process of data collection and self-reflection. In order to address cascading barriers, it was recommended that the design process should be carried out in an holistic approach that involves iterations between stages. The aforementioned model aimed at helping with the process of designing a personal informatics system. There are also studies that have explored design implications for data logging systems that support self-reflection. For instance ~\cite{li2011understanding} highlighted that such tools should be designed to address six questions that users ask themselves when engaging with their personal data; these questions are based on, status towards achieving their goal, history for the purpose of discovering patterns that are crucial to the preferred behaviour, formation of goals to facilitate in attaining a preferred behaviour, discrepancies between their behaviour and goal, context of past behaviour in order to discover patterns, and discovering of factors that may affect their behaviours. Another study by~\cite{macleod2013personal} suggested factors that drive motivation of chronically ill people in engaging with their personal data as; curiosity, and self-discovery of what is happening in their health.

The most recent model to help in understanding how people use personal informatics systems, suggested that these systems are meant to be fully integrated into people's daily lives~\citep{epstein2015lived}. This model extends~\cite{li2010stage}'s model, by splitting the preparation stage into \emph{deciding to track} and \emph{selecting tool} processes; and combining collection, integration, and reflection into tracking and acting. This model also includes further stages beyond tracking and acting and these were lapsing, and resuming tracking.  From lapsing, issues that contribute to discontinuation or intermittent usage are explored, while in resuming tracking, issues such as switching of tools, incorporation of previous history/data while resuming to use are explored on this stage.  

Guiding an end user through an action/acting stage for the objective of minimizing barriers in execution of this stage as suggested by a personal informatics model~\citep{li2010stage}, can be perceived as an attempt to nudge individuals towards certain behaviours. There has been a debate from HCI research community of whether behavioural nudges are ethically accepted or not as some researchers are proposing a more neutral approach while others recommend application of intervals of behavioural nudges upon tracking (collection and reflection) activity.  For instance,~\cite{munson2012mindfulness} advocates that the focus on personal informatics  should be towards enabling end users to better know owns behaviour instead of applying behavioural nudges, and suggests that adoption should be voluntary. Also in \cite{epstein2015lived}'s model it is highlighted that sometimes people use personal tracking systems for other reasons beyond behavioural change goals such as instrumental benefits (i.e. to get rewards from location trackers like Foursquare), or out of curiosity. However, \cite{epstein2015lived}, still  shows that in most usage that is related to health domain i.e. in physical activity, behaviour change goal is a dominant motivational factor~\citep{epstein2015lived}; hence suggestions on what actions an end user should take are inevitable. In addition to this, technology is considered not to be neutral~\citep{Oinas-kukkonen:psd}. Technology has a capability of presenting social cues that trigger emphatic responses from humans~\citep{foggpersuasivebook}. If no action is recommended, still an action can come from within a person using the system as the result of self-reflection. According to~\cite{fogg1998persuasive} cited in~\cite{Oinas-kukkonen:psd}, intent of persuasion can originate from either of the three sources which are: (1) from the people who create or produce interactive technology; (2) from people who give access to or distribute the interactive technology to others; and (3) from the people who use or adopt an interactive technology. The intent of persuasion may also come from within personal informatics systems that don't recommend or suggest any actions and this through cognitive dissonance after self-reflection. People like their views of the world to be consistent and It also assumed that people always make rational and informed decisions~\citep{Oinas-kukkonen:psd}. By using personal informatics, individuals' decisions can be improved by being be able to see the discrepancies between their desired behaviours versus their performance~\citep{comber2013designing}. If there are inconsistencies, then a cognitive dissonance is introduced which may mediate a change of attitude or behaviour in order to restore consistency between beliefs and actions~\citep{Oinas-kukkonen:psd}. Therefore, an act of tracking(collection and reflection) itself can mediate a behaviour change through cognitive dissonance. The motivation of usage of personal informatics in domains such as health and finance has been found to be related to a behaviour change goal~\citep{epstein2015lived}. From this perspective, a basic personal informatics system with simply self-monitoring support can be viewed as a persuasive technology in contexts such as personal health and finance, because of its ability to trigger cognitive dissonance which can be considered as a persuasive stimulus. 

Knowing one self can be important in adoption of a better lifestyle. For instance one study found the use of pedometer alone (without other motivational affordances) increased physical activity by about 1 mile of walking per day~\citep{bravata2007using} cited in ~\cite{albaina2009flowie}. 

Personal informatics are useful in promotion of a healthy lifestyle~\citep{korhonen2010personal}. Such systems can be applied in prevention of onset of chronic conditions by motivating healthy individuals to change their lifestyle. Specifically, personal informatics can provide support for cognitive behaviour therapy (CBT) in public health settings~\citep{mattila2008mobile}, for individuals who are clinically obese~\citep{nih2000practical}. Literature from public health suggests that recording of personal behaviour (self-monitoring) as an essential part of behaviour modification therapy~\citep{nih2000practical}. Health self-management programs usually ask participants to keep records of their activities, physiological variables and other health-related data; personal informatics applications can make this process simpler and easier~\citep{medynskiy2010salud}. For instance, participants may record their daily calorie intake,and then have graphs that show trends of how far they have gone with reducing their intake. Instead of a person noting down this information on a paper diary, and drawing a graph afterwards, personal informatics can support this process to make it more efficient. The essence of self-monitoring is to promote self-awareness of one’s behaviour. That consciousness is fostered through behaviour observation. And behaviour observation can be achieved through behaviour recording.  

From literature, there is a wide range of mobile phone based personal informatics systems for promotion of physical activity, and also some of the systems include dietary advice functionality. Some of these are specifically for chronically ill patients e.g. ``A few Touch Application'' which targeted individual with  \emph{Type 2 Diabetes}~\citep{arsand:mobile}, and a system described by \cite{arteaga2010:persuasive} that targeted teenagers with weight management issues. There also exists systems that target general populations, and used for promoting healthy eating habits and engagement in physical activity e.g. Fish'n'Steps\citep{lin2006:fish}, UbiFit Garden\citep{klasnja2009:using}, Wellness diary\citep{mattila2008mobile}, ActivMon\citep{burns2012using}, PmEB\citep{lee2006pmeb}, iCrave\citep{hsu2014persuasive}, etc.  These systems promote behaviours that are beneficial in preventing weight gain or encouraging weight loss. These systems operate by facilitating logging of personal behaviour’s data, and one of the key strategies in attaining a behavioural change involves setting of personal health goals, which has been recently used in many developed systems. This idea is derived from a goal setting theory~\citep{strecher1995goal}. An example of a goal could be to walk for at least 30 minutes every day or to increase the number of times a person eats fruits and vegetables or to reduce the amount of starch in a meal. Li et al (2011a) shows that individuals who use personal informatics usually transition between two phases of behaviour change and these are self-discovery and maintenance. In the self-discovery stage individuals collect a lot of data they can use to discover patterns in their behaviours. After discovering of a pattern they can move to the maintenance stage. The maintenance stage entails setting of a personal goal and monitoring of a status towards achieving that goal. Users don't stay permanently in one phase. It is possible for an individual in the maintenance phase to go back to discovery phase if there is a new unknown pattern that has emanated and appears to affect their behaviour.

Despite such tremendous development in the field of personal informatics for health promotion, most of these systems are designed for the developed world context. Even existence of randomized clinical trials on utilization of a simple technology such as SMS is largely dominated by countries from developed world~\citep{cole2010text}. From HCI point of view, engagement with personalized systems is currently considered to be more personal from data collection to reflection processes. These applications are personal in the essence of ownership of hardware, applications, data stored in application, and the process of interacting with a system for both data collection, and reflection. The HCI context of the existing applications may not be versatile in developing world perspective especially in low income communities of where  both sharing in usage of technology and indirect usage through intermediary or proxy users are common~\citep{kaplan2006can,sambasivan2010}. HCI in the developing world is a complex relationship between technology, multiple users, indirect stakeholders, observers, and bystanders~\citep{parikh2006}. An interaction model that assumes one phone/device one person might not always be feasible in such contexts. Also in many contexts, interaction with technology may not be direct; intermediation by another person occurs when the primary user is not capable of using a device entirely on their own \citep{sambasivan2010}. For users with limited technology literacy or education, direct access to a user interface might not even be feasible~\citep{parikh2006}, hence intermediation might be the only means for these people to be able to perceive the benefits derived by the proliferation of mobile phones or any other ICTs. While many people in developing world context might lack textual and digital literacy, low-income communities are diverse and often there are some members who have competent skills to operate technology \citep{sambasivan2010}, and these people may be able to help others to benefit from technology usage.

The complexity of usage through intermediaries is beyond help on the spot \citep{sambasivan2010}, hence, it cannot be merely solved by endeavours to simplify the user interface. In exploring of why intermediated technology use is beyond help on spot, one has to look at the notion of collectivist societies. In collectivist societies, people engage in tasks in group formation. For instance, India is considered to be a collectivist society of where individuals are prone to group orientation towards tasks~\citep{parikh2006}. This encourages usage of technology through human intermediaries. In such usage at least two users are involved in one interaction process. There is much more complexity on factors that influence intermediated technology use and hence it cannot be simply explained by existing interaction models from computer supported collaborative work~\citep{parikh2006}. Sukumaran et al (2009) emphasizes the importance of having a better understanding of locally specific interaction models to address culturally influenced issues in using information technology throughout the developing world. Intermediated interaction in an example of such interactions that needs to be clearly understood.

In the context of personal informatics, frequency of usage may vary among different domains, with the ones targeting physical activity being used more frequent (on daily basis), while other domains usage is from once a week and beyond~\citep{epstein2015lived}. Therefore, for a context where an end user needs help, motivation to use, is no longer just for this user but also it has to consider the person helping. This research was particularly focusing on how a personal health informatics system  designed for a personal use can be adapted in the context where two sets of users are being involved in an interaction process (the first one being a beneficiary of that technology, meaning a person receiving help on an interaction task to both collect and self-reflect on their personal data, while the second one is an intermediary user, a person providing assistance to a beneficiary user).

The next section highlights the broader view of intermediated technology use in the context of both developing and developed world communities.  

\section{Intermediated Technology Use}
According to~\cite{heeks1999tyranny}, cited in~\cite{bailur2012complex}, human intermediaries bridge a gap between what the poor have and what they would need in order to use ICTs. Therefore, human infrastructure within ICTD context plays an instrumental role in facilitating information and communication access in low income communities~\citep{sambasivan2010human}. Intermediaries play roles beyond being translators of policies to the ground level, therefore, it has been suggested that, lack of understanding of their position and motivation played a part in contributing to failure of public access venues (PAVs) as ICTD interventions for bridging the digital divide~\citep{bailur2010liminal}. Without the presence of these facilitators in PAVs, the groups that are excluded from access due to their age, socio-economic status, level of education/literacy, gender, disability or caste are more likely to face barriers in accessing information~\citep{ramirez2013infomediaries}. Factors that affect and shape the outcome of facilitating information and communication access have been explored, for instance,~\cite{bailur2010liminal} used structuration theory\citep{jones2008giddens}, to understand the contradiction and conflict of intermediaries on interacting with their different networks i.e. how they play a liminal role with stakeholders of PAVs and multimedia centres (i.e. NGO or government, donor agency on one side and community on the other side).~\cite{bailur2012complex} argues that PAVs' intermediaries should not be taken for granted in the space of ICTD because they play a complex position of brokers and translators as they assume multiple identities to different stakeholders of which their roles are constantly negotiated and performed within these multiple constructed networks. Another study is by \cite{ramirez2013infomediaries} which investigated how human factors such as empathy and technical skills of infomediaries influence the outcomes of the process of infomediation to users at PAVs. 

The ecosystem of utilization of intermediaries in PAVs or other community centres has also been examined through lens of HCI. Focus on HCI has been on engagement of all layers of users involved in intermediated interactions.~\cite{parikh2006}'s study in India provided a taxonomy of intermediated information tasks from HCI perspective;  of which  different modes of access were distinguished, and each one of them was suggested to have its own design requirements. These modes of access were: (1) cooperative, of whereby several users fairly collaborate without domination by a single or fewer users; (2) dominated interactions, of whereby users collaborate but they is one or fewer users who dominate others in manipulating user interfaces; (3) intermediated interactions, this whereby the first user manipulates interfaces while the  rest of users are just observing what is happening; and (4) indirect interactions, of where one or multiple users are being assisted to interact with a system without being being present or observing while manipulation of user interfaces is taking place. ~\cite{sukumaran2009intermediated} conducted an experiment that investigated how social prominence of an intermediary versus technology in a computer kiosk affects perceived information characteristics and attitudes towards an interaction by a beneficiary user/secondary user and found out that when the technology was more visible and an intermediary did not monopolize access (situation of social equality), beneficiaries tended to feel more engaged and positive. 

Although intermediaries in public access venues are considered as policy implementers on the ground level through working with communities, their position is very complex as they are the bottom of the hierarchy but they are also perceived not to be part of the community; hence they cannot specifically identify with a certain group since their roles are adapted according to circumstances~\citep{bailur2010liminal}. Motivation of intermediaries in this context of PAVs is negotiated relative to their particular network. A different ethnography study by~\cite{sambasivan2010}, explored the dynamics of intermediation beyond public access venues (\emph{i.e in inherent home, or community settings that involve neighbours and family members as intermediaries} --these intermediaries are more embedded to the community as they are considered part of it). \cite{sambasivan2010} presented three distinct forms of intermediated interactions: inputting intent into the device in proximate enabling, interpretation of device output in proximate translation, and both input of intent, and interpretation of output in surrogate usage. This study also discussed: (1) social mediators of motivation for intermediation (i.e interpersonal trust or prior social rapport, a give and take economy, social structures (i.e. access constraints due gender, economic status, tendency of reliance on others etc)); and (2) design implications to enhance engagement of users (primary and secondary users) such as : reorientation of technology  to allow sharing between primary and secondary users for asymmetric engagement; and supporting persistent storage of information for retrieval at later stages by beneficiary users. The study also proposed that measurement of use should go beyond ownership to also consider those who benefit without direct usage. 

The concept of informal help in technology use within family and social network settings is not an exclusive  phenomenon of developing world only; it is present in developed world as well. For instance ~\cite{poole:chh} cited in~\cite{katule2016leveraging} examined the dynamics of computer help-seeking and giving behaviors in the context of family and social networks settings, their findings indicated that an important factor that  encourages help-seeking behaviors is availability of unlimited help provided as a part of a longer-term relationship, while in the case of help-givers, they are motivated by a sense of being accountable to their family members and friends.

In the next subsection it is discussed of how intermediaries have been used in other health behaviour change interventions in the context of developing world and what is the gap from literature.

\section{Intermediaries in Supporting Health Behaviour Change}
In the context, of health behaviour change, typical examples of human intermediaries is on utilization of community health workers who provide access to health information to less privileged individuals in resource-constrained environments.  In many ICTD projects, CHWs, access health information on behalf of communities in which direct access to health resources is not possible, and are an effective bridge between communities and government-based resources~\citep{katule2016leveraging}. One project in India used community health workers (CHWs) -referred to as  ASHAs (Accredited Social Health Activists), of where these ASHAs were empowered with mobile phones that contained persuasive messages. These messages gave ASHAs credibility in persuading pregnant and postnatal women together with their relatives on maternal health issues~\citep{ramachandran2010mobile,ramachandran2010research}. Most of these ASHAs are women.

A project in Lesotho~\citep{molapo2013software}, empowered rural health trainers with a software application for creation of digital  training  content, voice-over images  that can be used by low-literate CHWs to train clients in the villages. While the main objective of these podcasts was for training purposes, upon CHWs showing them to their clients, there were unintentional persuasive effects that motivated these clients to get tested for diseases such as tuberculosis.

A study by~\cite{kumar2015mobile} in India, used a feminist reflexivity lens to study how patriarchal structures and social conventions constrain women in accessing maternal health information, and how these women leverage help from intermediaries within communities to navigate their way out. They further highlighted different groups of intermediaries who facilitate dissemination of information and examples of these intermediaries include but not limited to mobile shop owners, children and youth, and ASHAs.  It was also found that even ASHAs may also face constraints on the process of transferring mobile media; hence may seek help from their family members.~\cite{vashistha2016mobile} conducted a fourteen (14) weeks experiment to compare three distributions channels in dissemination of mobile videos on maternal health; mobile shop owners, laptop owners, and ASHAs. Both of the three distribution channels were found not to be very different, however, ASHAs were found to be more effective in distributing videos to the people who where interested of such videos.

In the aforementioned projects that utilize CHWs, these CHWs were acting as human access to information that had a persuasive effect. A challenge with utilizing CHWs is that their availability is limited to fewer visits in intervals of weeks or months; hence may not be suitable for a technology such as a personal health informatics of which its beneficiaries may need to engage with a it more frequent. Also other forms of distribution and viewing have limitations as it was found in~\cite{vashistha2016mobile}'s study that dissemination decreased over time, therefore, it was  suggested an exploration of alternative mechanisms  to extrinsically motivate intermediaries and viewers for broader video distribution.

In the context of children and youth within family settings, one may argue that their innate tendency to care for members of their families or communities may be a sufficient motivational factor for sharing health information, however, a study by~\cite{epstein2015lived} found that users of personal informatics that target health domains such as physical activity have tendency of using them more frequent (at least once per day) compared to personal informatics targeting other domains. Introducing such a system in an ecosystem of intermediated technology use it that will have the following implication on its utilization. There is a possibility that people who are less familiar with such systems to seek help more often. Dependence on  an natural intrinsic motivation of intermediaries alone may hinder availability of such a system to beneficiaries. The outcome of this is that there will be an intermittent usage which may have an impact on self-reflection, therefore, introducing a bottleneck in persuasion. The caveat of relying on natural intrinsic motivation of children to help their parents is also demonstrated in a study by~\cite{kiesler:twi} about informal help, of where parents reported that they were sometimes hesitant to seek help from their children to avoid negative experiences (i.e annoying their children because of asking for help more often). This proves that for systems such as personal health informatics of which help may be solicited more often, there is a need to explore on motivation techniques to enhance user experience of intermediaries. In the next section, the discussion is centred on a theoretical foundation on, motivation and user experience strategies that were applied at later stages in order to support usage through family intermediaries. Motivation is explored through the lens of self-determination theory.
\section{A Self-Determination Theory Approach to Motivation}
Motivation is categorized into intrinsic motivation (i.e. inherently embedded with ones' values and goals), and extrinsic motivation (i.e. doing something because of expecting some external outcome)~\citep{ryan2000intrinsic}. Therefore, intrinsic motivation the locus of control is internal to the person while in extrinsic motivation the locus of control is external to the person~\citep{lee2015:relating}.

Self-determination theory (SDT)\citep{deci1985:intrinsic}, a well grounded theory of human motivation, is concerned with how individuals develop interest to engage with certain activities that were once considered uninteresting~\citep{ryan2000intrinsic}. The focus of SDT is on the process of internalization of external motivated behaviour~\citep{ryan2000intrinsic}. The theory stipulates different levels of internalization for self-regulation of uninteresting but important activities to become interesting, of where by these levels are classified into four stages namely; (1) external regulation, (2) introjected regulation, (3) identified regulation, and (4) integration~\citep{ryan2000intrinsic}. In external regulation, individuals self-regulate because of an external outcome; in introjected, individuals self-regulate as an attempt to raise their self-worth with respect to others; in identified, individuals have put value into an activity, therefore, they try to self-regulate an activity because they consider it as important probably for achieving a much broader goal; and in integration, individuals have fully assimilated the self-regulation to their core values and beliefs.  Integration shares values with intrinsic motivation although it is not intrinsic motivation since its self-regulation is due fulfilment of an external outcome while in intrinsic motivation self-regulation of an activity is as the result of an activity itself being interesting~\citep{ryan2000intrinsic}. The internalization process is governed by social and environmental factors of which individuals function~\citep{ryan2000:self,lee2015:relating}.

SDT suggests that an intrinsically motivated activity is performed out of satisfying some psychological needs, therefore, for an uninteresting activity to become interesting through external rewards, social factors must provide support for the following three basic psychological needs; competence, relatedness, and autonomy~\citep{ryan2000intrinsic}. Autonomy is freedom to self-initiate and regulate a behaviour. Autonomy emphasizes on the importance of individual's ability to choose their own way of presenting oneself without peer pressure. With autonomy people can uniquely choose an identify to represent oneself. Autonomy gives individual freedom to choose when and how they want to initiate a behaviour. Competence emphasizes the need for individuals to be presented with challenges that give individual a chance to sharpen their skills in order to match those challenges. This is process of providing challenges is appropriate for ones' health psychological development and  overall well-being~\citep{zhang2008motivational}. Relatedness is the desire by individuals to feel a sense of belongingness. People like to be connected to others.

SDT has two sub-theories namely; (1) cognitive evaluation theory, which focuses on supporting of the aforementioned basic psychological needs and (2) organismic integration theory, which focuses on the process of internalization by discerning extrinsic motivators that can foster intrinsic motivation from extrinsic motivators that can harm intrinsic motivations~\citep{ryan2000:self,lee2015:relating}.

SDT has been used to understand motivation on various activities or behaviours such as; gaming~\citep{ryan2006:motivationalpull}, physical activity~\citep{power2011:obesity}, tobacco cessation~\citep{williams2006:testing}, energy saving~\citep{webb2013:self}, etc. This study brings the motivation pull of gamification in encouraging intermediaries to assist in utilization of personal health informatics. In the next section, gamification is discussed from perspective of self-determination theory.
\section{Self-Determination Theory Support in Gamification}
\cite{zhang2008motivational} situated the need for autonomy, competence, and relatedness into motivational affordances to use ICTs and emphasized on the importance of supporting the aforementioned needs in order to foster motivation in usage of ICT systems.~\cite{ryan2006:motivationalpull} investigated motivational pull behind video games using self-determination theory and found that needs for autonomy, competence, and relatedness independently predict enjoyment and future game play. This motivational aspect of gaming has attracted researchers to explore its usage beyond gaming context. Gamification is the use of game design elements in non-game context~\citep{deterding2011game}. Gamification tends to invoke users’ intrinsic motivation through gameful experiences and affordances~\citep{hamari2014persuasive}. Gamification is used outside game context to increase interest on uninteresting but instrumental activities (i.e. physical activity, crowd-sourcing tasks such as image annotation etc.).

\cite{sailer2013:psychological} used self-determination theory to understand the motivational pull of these game elements that can be used in non-game contexts of which their research attempted to provide examples of matching game elements to motivation mechanisms; for instance badges can be used to foster a sense of competence, while a leader board can be used to foster a sense of relatedness as it puts emphasis on collaboration among members of different teams. There are different schools of thought with whether gamification itself is a game or not. Most debates are centred around what a typical game entails (i.e. presence of rules, meta-games, immersion, voluntarism in their adoption, etc.)~\citep{seaborn2015:gamification}. According to~\cite{deterding2011game}, users of gamification can socially construct a perception of whether gamification qualifies as a game or not. The paper further argues what can be experienced in gamification can be termed as a gameful experience with an experiential ``flicker'' between gameful, playful, and other modes of experience and engagement.~\cite{seaborn2015:gamification} argue that the goal of gamification is different from games, therefore, it should rather be considered as an act of integrating user experience to an activity outside game context. The use of gamification is influenced by social factors such as social influence, recognition, reciprocal benefit, and network exposure; hence it is important to have a community of people who are committed to the goals that the gamification promotes~\citep{hamari2013social}.

\section{Games in Personalized Health Interventions}
Following the diffusion of video games in many digital devices of which these games are solely used for entertainment purposes, there is an increasing interest on the potential of such entertaining platforms in influencing positive changes in health behaviours~\citep{king2013gamification}. Traditionally, games were sedentary, but nowadays there are games that require user to exert body movements in order to play a game. Researchers are exploring of how games could be adapted to engage people with personalized interventions for health~\citep{mccallum2012gamification}. Use of games for health includes exer-games, games with purpose (serious games), and gamification.
\subsection{Exergames}
In the past, traditional video games were mostly sedentary. Exergames are defined as a combination of exertion which is more than sedentary activities and video games, which may include strength training, balance, and flexibility activities~\citep{oh2010defining}. Exergames increase the amount of energy expended by the body~\citep{graves2010physiological}. Examples of exergames include  dance video games i.e. ``Dance Dance Revolution''~\citep{lieberman2006dance} or games such as Nintendo Wii Fit~\citep{gobel2010serious}. There also outdoor exergames such Zombie run, which allows a user to experience immersion while jogging through narratives that make the user believe that they are being chased by Zombies ~\citep{southerton2013zombies}. However, using of an exergame should never be confused to exercising~\citep{oh2010defining}. According to~\cite[p. 126]{caspersen1985physical} cited by~\cite{oh2010defining}, ``\emph{Exercise is doing a physical activity intentionally to improve or maintain physical fitness with a planned, repetitive, and structured format}''; hence playing an exergame entails exerting but it remains a physical activity which which may be remain for entertainment purposes unless an activity of exerting is performed according to the definition of exercise~\citep{oh2010defining}. An exergame is far much better than a sedentary video game as it promotes physical activity which are important for increasing energy expenditure. This form of energy expenditure which doesn't fit in a category of exercise can be refereed to as \emph{NEAT} -- non-exercise activity thermogenesis. NEAT activities such as walking, taking stairs or exergaming (playing exergames) have been found to expend a significant amount of energy~\citep{fujiki2008neat}.

Exergames for elderly population have been explored. For instance a study by \cite{brauner2013increase} developed an exergame that required players to pick fruits using an avatar that represents them on screen of where an interaction with an avatar was through a Microsoft kinetic sensor for detecting body movements and gestures. The aforementioned game was found to be enjoyable by the elderly participants. However, there was no personalization of information to support the player into resuming from previous state of the game and in addition the only interaction was through body movements; hence the interaction may be less complex (demanding less assistance from intermediaries) compared to the one required by a gamified personal health informatics. A review on popular personal apps has revealed that many don't accommodate the needs of older adults~\citep{silva2014:smartphones}; hence this study emphasizes that one can leverage on existing usage through intermediaries for such populations as this mode of interaction is already prevalent in many low income communities of developing countries. Exergames and systems that support competition on health self-reflection, that involve  parents and children have also been studied~\citep{grimes2009toward,saksono2015spaceship} but not in the context of intermediaries supporting beneficiaries as it appears as if both parents and children are direct users of such systems. 

\subsection{Serious Games in Health}
\cite{mccallum2012gamification} defines serious games as games that have been specifically designed to achieve some change in the player i.e. change in knowledge, attitude, physical ability, cognitive ability, health, or mental well-being. Serious games are sometimes referred to as games with purpose of  where their intention is to provide experience and emotion with the goal of conveying a meaning at the end~\citep{marsh2011serious}. Areas on which serious games can be utilized in personal health include; preventive (exergames), therapeutic (rehabitainment), assessment (self-ranking), educational (medical information), and informatics (personal health records)~\citep{mccallum2012gamification}. As serious games add user experience to an outside activity (probably uninteresting one), then there is an overlap of goals between serious games and gamification. 

There is a rapid increase in the number of gamification-related studies within persuasive technology field~\citep{hamari2014persuasive}. The following are examples of systems that use games/gamification in motivating particular healthily  behaviours.~\cite{lin2006:fish} developed and evaluated \emph{{Fish'n'Steps}} system, a computer game which links player's daily footsteps count to the growth and emotional state of a pet fish in a tank. Another application of games in physical activity is demonstrated through \emph{{Neat-o-Games}}~\citep{fujiki2008neat}, a ubiquitous collection of PDA based games which allowed players to accumulate activity points which they could use in a race game. In addition, activity points could be used to provide hints in puzzle solving game called Sudoku. The usage of points in Sudoku, resulted in players to be left behind in a race game; hence motivated to do more physical activity. \emph{Flowie} system designed for home settings, was a virtual coach to motivate elderly individuals to walk through use of, a frame casing with a touch screen display that shows a flower of which its vitality corresponds to the amount of physical activity captured by using context aware sensors~\citep{albaina2009flowie}. A concept similar to Flowie system is that of UbiFit garden, which generates a garden with flowers of different types that discerned different types of activity such as cardio, walking, housework, etc.~\citep{klasnja2009:using}. There are also other studies that use gamification to motivate adolescents or teenagers in behaviours such as frequent monitoring of blood glucose or physical activity~\citep{arteaga2010:persuasive,cafazzo2012:bant}. Apart from interventions that target promotion of physical activity, there are apps game based apps designed to encourage healthily eating. An example of such an app is \emph{It's Time to Eat} - a mobile game designed to motivate children to practice health eating habits by taking care of their virtual pet~\citep{pollak2010s}. In their game, players were required to send a photo of their breakfast, and then a nutritionist would give it a certain score. Based on the total score, a virtual pet would respond with an emotional response. A healthily breakfast resulted into a pet becoming happy while if the meal is not healthily a pet would become sad. The ubiquity of a mobile phone presents an opportunity for delivery of interventions that utilize aforementioned systems. 

There is also a different approach that doesn't constitute adding user experience to a health behaviour instead a sedentary game is interlaced with the targeted health behaviour. An example of such games are the ones that take advantage of motivation factors of video games by introducing a burst of physical activity during a session a sedentary game play. For instance a system called \emph{PLAY MATE}, which was an alteration of an open source computer based game called \emph{Neverball},  which requires players to collect coins in a limited amount of time~\citep{berkovsky2012physical}. In PLAY MATE, players were allowed to gain in game virtual rewards (extra time) in return for performing a jump in the middle of a session of sedentary game play. Jumps were detected through a sensory devices (accelerometer and gyroscope) worn on the waist. The preliminary evaluation of the aforementioned system indicated that skilled players had a tendency of performing less jumps compared to less skilled players; hence the game was modified to include an adaptive algorithm of where the level of difficult was personalized according to player's completion time of previous levels.

Most of the interventions reported on literature are carried out in contexts that are not constrained in resources with exceptions of few such as study that developed an exergame for families in low socio economic areas~\citep{saksono2015spaceship} or another study that used user centred approach in development of mobile game based applications for promotion of physical activity in low socio-economic status youths~\citep{blackman2016developing}. In the context of this research, the focus is in low resource settings of a developing country which may not be the same as a context of low socio economic status in developed countries. The second drawback of those gamified personalized interventions is that they tend to be designed for direct/primary users of technology as consideration is only on direct beneficiaries. In most of these interventions, the person (be an adult or young person) who is a targeted beneficiary of information on the app is expected to be an actual manipulator of user interfaces of such a system. Therefore, none of the aforementioned studies has explored utilization of gamification of where one user facilitate an interaction process while an actual beneficiary remains as an observant or indirect/secondary user.   



\begin{flushright}
\end{flushright}
