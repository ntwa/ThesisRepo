% Chapter 1

\chapter{Conclusions and Future Research} % Main chapter title

\label{discussionchapter} % For referencing the chapter elsewhere, use \ref{Chapter1} 

\lhead{Chapter \emph{Conclusions and Future Research}} % This is for the header on each page - perhaps a shortened title

%----------------------------------------------------------------------------------------
The main research questions were centred around factors that could affect utilization of a personal health informatics application through intermediaries, and also the effectiveness of gamification in increasing both engagement of intermediaries and collaboration between members of a pair in intermediated use. This chapter revisits the research questions and presents a discussion of how they were addressed. It also summarizes on takeaways which  are regarded as design considerations for motivational affordances in intermediated use context of a self-monitoring application for promotion of healthy behaviours. These design considerations revealed social factors that could contribute to success of an intervention such as the one in this study, and motivational strategies that could be utilized in order to keep both intermediaries and beneficiaries engaged with a personal health informatics (self-monitoring) system/application.

\section{Discussion on Research Questions}
The main focus of this research was to uncover how social factors and persuasive systems' inspired motivational affordances could impact intermediated use of personal health self-monitoring applications. Intermediated technology use is an interaction model that is prevalent in contexts of low-income communities of developing world. Many health self-monitoring applications are designed for personal use with their motivational affordances targeting only direct users; hence such motivational affordances have not been explored in the context of intermediated use. Therefore, in order to understand design requirements for motivation affordances targeting intermediated use this research aimed at providing answers for the following research questions.

\textbf{RQ1}: What is the role of social-technical settings in intermediated use of a gamified self-monitoring application targeting promotion of healthily eating and physical activity? 

The aforementioned research question had two sub-questions. The first sub-question aimed at identifying prerequisite factors that could affect intermediated use and the second sub-question aimed at exploring the extent to which an understanding of the identified factors is important in the context of intermediated use. In order to provide answers to those research sub-questions, a series of studies were conducted. These studies included: one contextual enquiry (Chapters \ref{contextualenqchapter}) and two consecutive evaluations of two versions of the prototype (Chapters \ref{prototype1chapter} and \ref{prototype2chapter}). 

The most vivid factors that were manifested by aforementioned studies include but not limited to: social relationship; collaborative reflection from a shared device; and motivational affordances  either from the app or socially construed as the result of using the app. Prior social relationship was instrumental in facilitation of negotiation for interaction between members of a pair (an intermediary user and a beneficiary user). Before an interaction took place there was always a negotiation for interaction; hence presence or absence of a prior social relationship was a determinant of whether a negotiation would be successful or not.

Prior social relationship facilitated also a collaborative reflection. Negotiations would be initiated within a pair by either member of a pair. Once a negotiation to initiate interaction was successful then intermediary users navigated through the app. In the process of navigating through the app, anything they found interesting would be shared with their respective beneficiary users. This would spark a conversation between the two users (members of a pair). As the result of sharing information between members of a pair there was a collaborative reflection. 

Informative evaluations revealed that involving children who are family members is the key to success of this kind of an intervention, and by having an app running on a shared device had increased tendency of members of a pair to reflect collaboratively. In such contexts it was no longer just the matter of help seeking and help giving but more of a collaborative effort towards a joint goal. The idea of collaborative interfaces for health information within family settings has been explored in computer supported collaborative work (CSCW) literature. \cite{colineau2011motivating} designed a system to support a family to select a collective health goal and receive feedbacks that entailed comparisons between families. Their system was found to encourage members from within a family or members of different families, to work together and in particular to help each other in finding ways to live a healthily lifestyle. 

Therefore, family settings may provide an idyllic opportunity for members to discuss healthy issues collaboratively. Collaboration between a parent and a child or close family members had a positive impact on child's perception as some intermediaries shared testimonies about their habituation of skills on eating healthy. In addition, intermediaries in some cases logged their data about meals because what they ate was not different from what had been eaten by their respective beneficiaries. A study by~\cite{grimes2009toward} identified four key areas of consideration in which sharing of, and reflection on, health information can be leveraged within family context as follows: (1) overlaps of routines between family members through shared meals, space, etc which can provide opportunities for collaborative data logging and reflection among family members; (2) sharing is done at the expense of balancing competing values of openness, caring, and modelling with the value of protection; (3) understanding of sensitivity on comparisons and competition based upon health information in the context of the family as it may have negative consequences; and (4) collaborative sharing of, reflecting on, health information can also foster family's bond. In the context of this research, it was evident that the app had increased the bond between participating family members as majority of them claimed that were interacting more often. This is also demonstrated by playfulness behaviours that were exhibited in the process of sharing information as it was shown in one of the excerpt in chapter \ref{prototype2chapter} (Prototype II):

\userquote{\textbf{Zandiwe}, a beneficiary} {` When she got time, when she is done with her homework she comes and sees the app. And then laughs at me like `Yo yo yo [An interjection for Xhosa speakers to express the feeling of amazement by something] you can walk yo yo yo', like `you walked a lot today' and what what [She was implying to other words said by Lindiwe]''}

In existing work from computer supported collaborative work it appears the emphasis is on parents trying to model health behaviors of their children.  For a instance in a study by ~\cite{saksono2015spaceship}, a collaborative exergame was developed in order to support both parents and kids to exercise together. Although their goal was to help kids learn from their parents, the collaborative environment was beneficial to both parents and children.

In the context of this research it was peculiar that children were attempting to nudge their parents to live healthily. Therefore, it is not about only a parent attempting to guide his/her child also a child could become a facilitator to guide a parent about healthy choices. This was mediated by an existing familial relationship. In addition to that in most cases of where pairs consisted of a parent working with a child, an intermediary had a tendency of realizing a rationale in fulfilling  requests for interaction from their respective beneficiary users even in cases where intermediaries felt their autonomy was being violated. Empathy led to such intermediaries becoming accountable to the well-being of the people they cared about. The bond was further strengthened by the presence of motivational affordances which had a role to play in making intermediaries believe that information on the app was theirs as well and not for only the beneficiary users who were being assisted; as the result those intermediaries were being responsible team players. Therefore, in such situations reflection was done collaboratively and not at the personal level as it is common in existing personal health informatics applications.  However, perceived interest on motivational affordances differed between intermediary users and beneficiary users. Comparison based on abstract things like points were less meaningful to older beneficiaries as they tended to value more on perceived benefits and social support from others. In the absence of interaction among beneficiaries there was a tendency to have less engagement from the side of beneficiaries. Hence strategies that need to be applied for this user group need to take into consideration of availability of social support from people who already know each other. This kind of social support indicated a tendency to increase relatedness among beneficiary users that were reported in evaluation of a prototype in Chapter \ref{prototype2chapter}. One important conclusion out of this finding is the need to have separate persuasive strategies that discern between beneficiaries and intermediaries. Existing game mechanics may work well with intermediaries while on beneficiaries social support should be encouraged in order to leverage motivational affordance provided by social comparison.

In response to the first main research question it can be concluded that motivational affordances could foster collaboration and subsequently a relationship bond of members of a participating pair in an intervention provided that there is a prior social relationship between members of a pair. This implies a combination of motivational affordances and familiar relationships is crucial in making the collaboration more interesting and enjoyable. Therefore, a prior social relationship and perceived motivation affordances were main determinants for two users from two sets (intermediary set, and beneficiary set) to view any efforts to interaction as carried out on the behalf of the respective team and not for a beneficiary user alone from the team.  

The second research is provided below. This aimed at exploring the impact of using gamification as a means to motivate collaboration that leads to intermediated use of a self-monitoring application.

\textbf{RQ2}: How gamification plays a role in motivating intermediated use of self-monitoring application targeting promotion of healthily eating and physical activity?

This research question was broken down into seven research sub-questions as provided below:

\begin{enumerate}[label=\alph*.]
\item What is the impact of gamification in supporting self-determination of intermediary users to engage with a self monitoring application in intermediated use context?
\item What is the impact of gamification in supporting self-determination of beneficiary users to engage with a self-monitoring application in intermediated use context?
\item What is the impact of gamification on motivation of beneficiaries to self monitor diet?
\item What is the impact of gamification on motivation of beneficiaries to self monitor physical activity?
\item What is the impact of gamification in frequency of utilizing the self-monitoring application in intermediated use context?
\item How the presence of gamification affects the relationship between an intermediary user and beneficiary user?
\item To what extent gamification may encourage or discourage internalization of intermediated use behaviour?

\end{enumerate} 

In order to address the aforementioned research sub-questions that contributed to the second main research question, a summative evaluation (Chapter~\ref{summativeevalchapter}) was conducted that compared between a system without gamification and a system with gamification. 

In support for self-determination in research sub-question 2(a), gamification demonstrated a potential to increase perceived competence of intermediary users in using the self-monitoring app with their respective beneficiary users while aspects of perceived autonomy and perceived relatedness didn't show improvement. Many insights on design considerations were highlighted on aspects of autonomy. One most important was freedom to choose which gamification features to participate in and at what level in order to cater for intermediary users with different personalities and skills. On aspects of relatedness, features that promote socialization and relatedness were only effective for users who are not co-located and this resonated with findings from other literature. 

In research sub-questions; 2(b), 2(c), 2(d), all three aspects of self-determination theory in three sub-questions (to use the app, to self-monitor diet, and to self-monitor physical activity) were the same for each respective comparison between logbook and gamification conditions. However, the presence of a self-monitoring app regardless of whether the app has gamified motivational affordances or not, appeared to increase self-determination of beneficiaries in monitoring their health at endline in comparison to baseline. A significant change was observed in self-monitoring of diet since there was less interest or knowledge in tracking of diet while for physical activity participants seemed to already be doing implicit tracking through approximation of amount of physical activity done while doing daily errands. To majority of the beneficiary participants, gamification was of less importance for reasons that have already been highlighted on discussion for the first research question. Reflecting upon all series of user evaluation studies that were conducted, insights reveal that motivational affordances from gamification that were used in this study may work effectively on intermediaries (who in this case were mostly children) and young beneficiaries. 

In the last three sub-questions; 2(e), 2(f), and 2(g), gamification increased frequency of usage through intermediaries as the result, gamification fostered collaboration in cases were both intermediaries and beneficiaries were interested with gamification features. However, this usage and collaboration from Chapter \ref{summativeevalchapter} (Summative Evaluation) appeared to be mostly accounted to introjected internalization and had negative consequences.
 
Despite the positive outcomes from gamification, the negative consequences as the results of increased competition have become an area of concern~\citep{jia2016personality}. It is even more concerning when such competition results into negative consequences in health settings~\citep{grimes2009toward}. To reduce these negative implications of excessive social comparison and competition, there is an emphasis on supporting challenges on the level of ``\emph{task mastery climate}'' rather than on competition that has ``\emph{ego involved climate}'' as the  former foster intrinsic motivation while the latter can harm it~\citep{saksono2015spaceship}. When ``ego is involved'', participants may do things just to maintain their self-worth, and this is equivalent to introjected regulation as postulated by organismic integration theory of SDT\citep{ryan2000:self}. In introjected regulation individuals don't see a value in regulating a behaviour rather they perform it merely for the purpose of outdoing others or maintaining their social status. For instance, the idea of having a leader board in this intervention encourages competition and it affected motivation of intermediary users who didn't do so well. Features such as fish tank and botanical garden appear to promote task mastery. For instance in one scenario reported in chapter \ref{prototype2chapter} (Prototype II),  a beneficiary user was dissatisfied by the look of their garden. 

\userquote{\textbf{Lulama}, an intermediary} {``She (Nokhanyo) saw the garden. The first day she saw just the house and brownish. She
is like `What is this'. I told her. She said `Aha! [Expressing
dissatisfaction]. It must look green and healthy'. And then
she saw the garden again and said `It is looking good.'''}

This important finding suggests that such features could encourage task mastery climate. If designers have to use a leaderboard, they need to be cautionary of negative impacts on users' competence despite its ability to foster relatedness~\citep{sailer2013:psychological}. Leader board can also result into an extreme competition between intermediaries which can result into a negative impact on a relationship by an intermediaries in cases where an intermediaries feel of being let down by their beneficiaries. Such a scenario is exhibited in the following excerpt on Chapter \ref{summativeevalchapter} (Summative Evaluation)
 
\userquote{\textbf{Jenner}, a female beneficiary from Athlone, 45 yrs old} {``Sometimes may be I forget to take the phone when I go walking and he would ask me `did you take the phone with you' Ooh Gosh I forgot.  Because when I walk to Park Town to exercise and sometimes  I am in such a hurry I forget the phone, he will be crossed with me.''} 

In the aforementioned case, an intermediary got angry because of her mother tendency of forgetting to take the pedometer (phone) when she goes out for walking. Therefore this highlights the importance of paying more attention should on features that support task mastery climate. 

In the context of results on Chapter \ref{summativeevalchapter} (Summative Evaluation), the negative impact of social comparison were more manifested in intermediaries. Beneficiaries who participated in evaluation were doing a lot of social comparison seemed not to affect the motivations was not affecting them in a negative way of where in most cases it proved to increase their enjoyment and motivation to engage with the app. Social comparison challenged beneficiaries to continuously set and revise goals of living healthy. Therefore in the context of beneficiaries it showed indications of promoting task mastery climate.  In addition,  \ref{prototype2chapter} (Prototype II) that as longer as beneficiaries were engaged even in a different way outside gamification context, collaboration between the two users even though their goals may be entirely different. Intermediary users may have goals of achieving rewards from game mechanics while beneficiary users may have goal of accumulating more steps or eating healthy to receive social support from among themselves.

In conclusion, it is clear that both gamification and other motivational affordances that were indirectly situated in the app promoted collaboration between participating pairs that had a prior social rapport. In addition, gamification in particular social comparison and competitions, both the one socially construed by users, and the one implemented as an intentional design goal, increase engagement of both intermediary and beneficiary users.
Gamification was effective in-terms of increasing engagement of intermediaries even though it had some challenges and limitations that need to be addressed as it appeared to affect both intrinsic motivation and internalization in the summative evaluation chapter (Chapter \ref{summativeevalchapter}). The number of participants who were active during summative evaluation had increased hence the negative effects of gamification and especially the leaderboard were conspicuous. 

\section{Limitations}
There are several limitations to generalizability of the findings from this research. The first aspect of limitation is on the sample size. Due not to having sufficient resources and difficulties in recruitment of participants, I ended up using convenient sampling and through approach the number of participants was limited. The sample was not probabilistic and was limited in size; hence affected power of the study and in addition technical glitches contributed to further reduction of power of the study. There is a need to repeat the same study with slightly larger sample size using probabilistic approaches.   The second aspect of limitation is that the researcher relied on interviews in capturing all the experiences. An ethnography study and a diary could be useful in capturing important insights on user experience that could not be recalled or revealed by interviews. The third aspect of limitation of this study is that evaluation of motivational affordances was done in a holistic manner; therefore it difficult to discern the impact of individuals users. In addition to that, evaluation of motivational affordances didn't take into consideration of how personality of users could affect the intervention.  

However despite having limitations that affect generalizability, the series of these small studies was important in laying a good foundation towards an understanding of important social dynamics in order to utilize intermediated use in the context of personal health informatics applications. 

\section*{Future Directions}
In order to enhance user experience, support for factors such as task mastery, support for reflection, enhancement of collaboration within a family (intra-families), or inter-families collaboration should be further explored. In addition, factors such as personality of users and different styles of parenting such as authoritative, neglectful, permissive, and authoritarian should be explored in the context of intermediated use of a personal health informatics application. Spatial arrangement of two users and technology also need further exploration in order to support optimal flow. This issue had an effect on flow of both sets of users. 

Other concepts that need to be further explored by future studies include sustained usage over a long period of time. Long term usage and prolonged benefits of personalized apps are still debatable. One study that a conducted a two years trials that compared among three strategies: interactive smartphone application on a cell phone (CP); personal coaching enhanced by smartphone self-monitoring (PC); or handout (pamphlets) as the control group found that smart-phone to be better than control in short time but in long term the results were not different~\citep{svetkey2015cell}. It is argued that usage of these apps should go hand in hand with other conventional strategies. In addition,  long  term engagement is a key challenge in personalized apps for health. Once the novelty effect is worn out users may tend to revert to their old habits. Gamification is not exceptional when it comes to the issue of the novelty effect. In the context of intermediated technology use it may even be more challenging in sustaining engagement since we are dealing with more that one layer of users. This is still a grey area that needs further exploration.
\begin{flushright}
\end{flushright}
