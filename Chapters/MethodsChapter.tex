% Chapter 1

\chapter{Methodology} % Main chapter title

\label{methodologychapter} % For referencing the chapter elsewhere, use \ref{Chapter1} 

\lhead{Chapter 1. \emph{Research Findings}} % This is for the header on each page - perhaps a shortened title

%----------------------------------------------------------------------------------------
\section{Preliminary Contextual Inquiry}
\subsection{Description of the Study}
The main objective of collecting preliminary contextual information was to elicit requirements to inform the design of a health self-monitoring application operated through intermediaries.  I had obtained ethics approval from Human Research Ethics Committee of Faculty of Health Sciences at University of Cape Town. I and one research assistant recruited participants at the diabetic and endocrine clinic located at Groote Schuur hospital in Cape Town. This is an outpatient clinic which runs on Thursdays and Fridays.\newline
Majority of the participants we recruited were above mid-aged. The rationale for choosing this group of participants was based on the fact that the technology I had envisaged to develop was targeting users who are less knowledgeable in cellphones and technology in general, therefore this group of participants was a representative of prospective beneficiary users of such a technology. My objective was to interview only overweight and obese patients but we also included few participants who appeared to be thin but were diabetic. Since diabetes is a lifestyle related disease, I found that it would be interesting to also understand utilization of cellphones, and access to information even to individuals who appear not to be overweight but these individuals may give out insights on various issues related technology utilization, and barriers to adoption of behaviors that are considered to be healthy.\newline
\subsection{Methods}
I together with my research assistant approached individual participants and explained the purpose of the study. People who agreed to participate were given consent forms to sign. We used a semi structured questionnaire attached at Appendix A to interview participants. Each participant was interviewed for a period of 20 to 30 minutes. The questionnaire had four groups of questions and these included demographics; cellphone ownership and utilization; access to health information and pedometers; and barriers to diet and physical activity.We interviewed a total of thirty participants.We started interviews in March 2013 and concluded at the beginning of May 2013. I analyzed quantitative data using descriptive statistics. I also examined qualitative data by extracting thematic areas which were key to informing the design of the prototype I had envisaged.\newline
\section{Implementation of the First Prototype of a Health Self-Monitoring Application}
\subsection{Implementation Platform}
I spent time developing an application based on findings from the aforementioned contextual inquiry. I was faced with a decision of whether to develop a native mobile application or web application. I decided to choose a web application. The rationale for choosing a web application was based on the following factor, I anticipated that users at evaluation stage were going to use their own mobile devices.\newline
I started developing the prototype using Django Python, Mobile Jquery, HTML 5.0, and MySQL. On the server side everything was implemented using Django Python and MySQL as the database. There were some limitations with this approach as not everything could be done on a web application. One draw back was a pedometer which required a native implementation because one has to implement code to interact with an accelerometer sensor which is platform dependent. Initially the plan was to find an off the shelf pedometer with an open API (Application Programming Interface). I came across a device called Fitbit\footnote{ https://www.fitbit.com/}. It has a RESTful\footnote{http://www.w3.org/TR/ws-arch/} web API that one could use to transfer steps' data to the Internet. I was unable to adopt Fitbit because it was not going to work in the context of participants I planned to recruit. Fitbit device required one to have a personal computer and a high end smart phone such as Samsung Galaxy III.\newline
I made a decision to implement a pedometer application that runs on android phones. I used open source code found on the Internet. This pedometer was capable of capturing steps walked by an individual and transfer them to the web application.\newline
\subsection{Implemented Features}
I developed a prototype that had the following features (screenshots are attached at appendix B):
\begin{itemize}
\item{Steps walked display at intervals of days, weeks, or months.}
\item{Recording of meals eaten specified by portions of food groups i.e. Lunch(fruits and vegetables,starch,dairy,fat,and high sugar food).}
\item{Meals charts showing the overall summary of each food consumed in intervals of days, weeks, or months.}
\item{Goal Setting that enables users to specify their desired targets in meals and steps walked.}
\item{A Reward sub-component(system) implemented using aquariums, botanical garden, leader boards integrated points and badges }
\end{itemize}
The premise of the aforementioned reward system was to engage users by supporting their motivation affordances\citep{zhang2008motivational} to engage with the prototype. I drew the design inspiration from a Self-Determination Theory (SDT)\citep{deci1985intrinsic}. SDT postulates that in order to support individuals intrinsic motivation needs you have to support three basic psychological needs which are perceived autonomy, perceived competence, and perceived relatedness. 

The prototype linked the reward system with Facebook social plugins that allow users to comment on or like each other. The idea was to allow each pair of users (an intermediary and beneficiary) to improve their relatedness with respect to other pairs of users. I also integrated a feedback mechanism that would appear on a Facebook group that comprises of all intermediary users as members. 


\section{Pilot Evaluation I}
\subsection{Description of the Study}
The plan was to evaluate the first version of the prototype with patients in  hospital settings. This plan didn't materialize since it was going to be arduous to negotiate ethics as far as the scientific contribution is concerned. One of the conditions for the study to go forward was to ensure that I carry it out as an RCT(Randomized Control Trial) with clinical or health outcomes afterwards. I had to redesign the study so that I work with participants outside of the hospital settings. I obtained a different ethics approval from FSREC(Faculty of Science Research Ethics Committee) of University of Cape Town.\newline
I recruited participants to help in evaluating an application through an NGO called Mamelani Projects\footnote{http://www.mamelani.org.za/} in Cape Town. This NGO works on  community health programs that target youth and women inhabiting in marginalized areas in the outskirts of Cape Town. Recruitment took place in September 2014. The NGO linked me with participants from a Township called Phillipi in Cape Town. A township in South African context is a suburb which is mostly inhabited by previously disadvantaged populations from an apartheid government. Phillipi township has both formal and informal settlements.  
\subsection{Methods}
 I recruited a total of six mid-aged women (35-45 years of age) to be beneficiary users of the prototype. These beneficiary participants came together with their family members to act as their intermediaries. The oldest intermediary was 23 years of age. Each beneficiary participant formed of a pair with their respective intermediary participant. I explained to participants of what they are expected to do. All twelve participants(6 intermediaries and six beneficiaries) signed informed consent and assent forms. After the signing of forms, I spent time teaching intermediaries on how to use the application. After the training intermediary participants I left participants with the application to use from the beginning of October 2014 to the end of November 2014  
\section{Second Iteration of Prototype Development}
After evaluation of the aforementioned first prototype, I did the second iteration of prototype development. This entailed fixing of bugs discovered during testing of the first prototype, and improvement of software functionality to closely emulate support for competition, autonomy, and relatedness. I revamped the prototype not to rely too much on Facebook as they were challenges in utilization of Facebook by participants. Some intermediaries had never used Facebook before, and the rest who were using Facebook, didn't utilize it so frequent hence reminders and other forms of feedback were not timely received by both intermediaries and beneficiaries. I opted to utilize SMS reminders in the second prototype. In addition, the township where I carried the first pilot evaluation had poor Internet signal and this affected accessibility of Facebook social features that were embedded in the app. Therefore, I designed my own social features to allow users to comment on or send messages to each other. This improved the ability of the app to load much faster as it didn't require any third party's plugins to load.
\section{Pilot Evaluation II} 
\subsection{Description of the Study}
Endeavours to continue working with participants from Phillipi were futile as I was advised to stop the recruitment process by the NGO I was working with. This was due to concerns that experimental phones were going to pose risk to both participants and the researcher. Therefore, in the second pilot evaluation, I had to find participants from Langa township which was more safer compared to Phillipi. I worked with a research assistant who was a resident of Langa. The research assistant helped with recruitment and screening of participants to identify the ones that met inclusion criteria to be recruited as participants. This time I had adjusted recruitment criteria after reflecting on the success and failure of the application deployed during the aforementioned first pilot evaluation. In the the evaluation of prototype I observed that it was nearly impossible for the intervention to have an impact if members of a pair don't cohabit or live within a proximity from each other. Therefore, this time I made a decision to only recruit beneficiary users who would elect intermediaries that are related to them and live nearby or in the same house with them. 
\subsection{Data Collection Methods}
\subsubsection{Interviews}

\begin{flushright}
\end{flushright}
