% Chapter 1

\chapter{Introduction} % Main chapter title

\label{introductionchapter} % For referencing the chapter elsewhere, use \ref{Chapter1} 

\lhead{Chapter 1. \emph{Introduction}} % This is for the header on each page - perhaps a shortened title

%----------------------------------------------------------------------------------------
\section{Background}
Obesity and overweight are currently global health concerns. A systematic review by~\cite{guh2009incidence} concluded that both overweight and obesity are associated with increased incidence of multiple co-morbidities including type 2 diabetes, cancer and cardiovascular diseases (CVD). The number of people who are considered to be either overweight or obese stands to an approximation of  1.3 billion people~\citep{steyn2006chronic}. A survey by~\cite{abegunde:theburden} which included a total of 23 low-income and middle-income countries had projected a loss US\$84 billion of economic production in between 2006 and 2015 from heart disease, stroke, and diabetes alone in there will be no efforts to intervene.

Co-morbidities that are associated with obesity are likely to inundate health care systems~\citep{pollak2010s}. In addition to that, at the moment health-care systems have failed to optimally treat chronic conditions such as diabetes due to lack of time to continuously provide  patient  care which is essential in management of chronic conditions~\citep{quinn2008welldoc}.
This need calls for innovative and citizen-centric  interventions to foster lifestyle changes in order to, both prevent or delay onset of chronic conditions and support patients in self-management of a chronic conditions~\citep{korhonen2010personal,aarsand2012mobile,higgins2016smartphone}. There has been a growing number of initiatives by both commercial and research communities in development of wearable sensors and mobile applications that can nudge individuals to eat healthy and increase their level of physical activity~\citep{chen2014healthytogether}. Citizen centric interventions are now possible due to the current advancements in hardware and software technologies which have facilitated creation of opportunities for automation of health self-management processes~\citep{arsand:mobile}. 

One interesting development trend in both academia and industry is the use of mobiles in health. The mobiles have become an effective way for ``just-in time'' delivery of interventions that target psychological processes~\citep{hsu2014persuasive}. These devices are currently omnipresent and people carry them most of the time~\citep{mattila2008mobile}; hence their presence brings a ``kairo factor'' in delivery of interventions that target both health promotion ~\citep{pollak2010s} and persuasion~\citep{hsu2014persuasive}. Smartphone based applications have been rapidly gaining popularity as effective tools to support delivery of personalized health information~\citep{handel2011mhealth}. One of the prevalent adoption of mobile health apps is their use in self-monitoring to augment \emph{cognitive behaviour therapy} - treatment of behaviour in clinical settings~\citep{mattila2008mobile,medynskiy2010salud}. These apps facilitate data collection of one's health parameters through inbuilt tools such as GPS, accelerometer (body activity sensor), etc; hence present an innovative way of monitoring and improving both health and fitness~\citep{higgins2016smartphone}. In order for such tools to support, changes in health behaviour and promotion of healthy lifestyle, theory based strategies such gamification (for enhancement of motivation), enabling self-reflection through goal setting and feedback (for improvement of self-efficacy), and SMS reminders are often applied~\citep{consolvo2009goal,cole2010text,hamari2014persuasive,hamari2014does,higgins2016smartphone}.
\section{Statement of the Problem}
A review by~\cite{higgins2016smartphone} presented evidence that self-monitoring apps can help patients reach their health and fitness goals. Also these apps can support individuals who are not patients to become aware of their behaviours which is an important step towards taking necessary actions for healthily lifestyle. However, such apps have limitations as they don't support specific interaction models that accommodate sharing of devices and indirect usage. Such mode of interaction are prevalent and relevant in the context of developing world ; hence they may not replicate well to some populations of users~\citep{kaplan2006can,sambasivan2010}, especially the ones that face barriers to direct access to user interfaces or technology~\citep{kumar2015mobile}. This research was exploring of how one could support a personal health informatics technology of which its usage is facilitated by intermediaries users on behalf of beneficiary users (indirect users). Despite a vast amount of literature on \emph{intermediated technology use}, such persuasive technologies have not been extensively explored in this context. Persuasive technologies tend to have their unique design considerations, and intermediated technology use has its socio-technical aspects; hence one has to understand factors to consider and how to go about implementing a useful intervention that can work in such a complex context. This study had two main research questions as presented below.
\begin{enumerate}
%\setcounter{enumi}{1}
\item How social dynamics may affect motivation of intermediated use in the context of a personal health informatics system targeting promotion of healthily eating and physical activity?

\textbf{Sub-questions}
\begin{enumerate}[label=\alph*.]
\item What social factors have in impact on intermediated use of a personal health informatics system?
\item To what extent those social factors affect the motivation of engaging with a personal health informatics system in intermediated use context?
\end{enumerate}
\item How effective are social incentives such as gamification in improving motivation  for intermediated use of personal health informatics targeting promotion of healthily eating and physical activity?

\textbf{Sub-questions}
\begin{enumerate}[label=\alph*.]
\item What is the impact of gamification in supporting self-determination of intermediary users to engage with a personal health informatics in intermediated use context?
\item What is the impact of gamification in supporting self-determination of beneficiary users to engage with a personal health informatics in intermediated use context?
\item What is the impact of gamification in increasing self-monitoring in intermediated use context?
\item What is the impact of gamification on motivation of beneficiaries to self monitor diet?
\item What is the impact of gamification on motivation of beneficiaries to self monitor physical activity?
\item To what extent gamification may encourage or discourage internalization of intermediated use?
\end{enumerate}
\end{enumerate}  
\section{Research Contribution}
This research was grounded by user evaluations, ideas from past studies, and  existing theories of human motivation. In total there were three users evaluation studies. These series of user studies were carried out in three townships in South Africa at different intervals of time. Each user study helped to uncover unique insights that were important in getting answer to the aforementioned research questions. Each user study consisted of several pairs of users of where each pair consisted a beneficiary user (the person who seek for help in using a personal health informatics system), and an intermediary user (a person who acted as a help give to a beneficiary user in order to interact with the aforementioned application). Beneficiary users elected their respective intermediary users and the pair had access to the app for a certain number of days before questionnaires and interviews were administered.  

Data collection techniques consisted mostly of triangulation of apps usage logs, interviews and questionnaires. In order to solve the problem, prior to carrying out any prototype development and evaluation, the study kick-started with a contextual investigation to uncover preliminary understanding of users context through administering semi-structured questionnaire to adult participants who were opportunistically approached in a hospital settings in Cape Town. Contextual investigation was followed by iterations in development and informative evaluations of mobile application prototypes. Motivational affordances implemented on prototypes included gamification features such leader-boards, badges, avatars, virtual pets caring (garden and fish tank), and social interaction features.  Through the course of eliciting feedback of user studies, I as the researcher was able to generate insights in an iterative manner of where each iterative user study informed the formulation and execution of the successive user study. 

From informative evaluations, the study concluded with a summative evaluation which had an objective of measuring the effectiveness of using gamification in promotion of intermediated use in the context of self-monitoring applications. 
 
The contribution of this research is mainly on understanding of social dynamics and motivational affordances to consider when designing a personal health informatics (PHI) for intermediated use. In this dissertation it is suggested that rather than designing a PHI only for the beneficiary, one can design for intermediated use, explicitly acknowledging the presence of more than one user of the application. This research demonstrated that it is feasible to frame the design of a personal application in way that promotes collaboration between an intermediary user and a beneficiary user; hence reaching the goal of motivating intermediated use. The dissertation highlights some social configurations that are crucial for a self-monitoring application in intermediated use context. The dissertation further emphasizes of the importance of pairing users within family settings to foster an environment that encourage intermediated use.  The study indicated that when a pair consist of immediate family members, then the prior social relationship may promote internalization of help-giving behaviours on the side of intermediaries. Prior social relationship appears to be a prerequisite for setting up an intervention and it can provide rationale for intermediaries to perceive gamification as something that is fun to use but at the same time as something done for good cause which in this case it was to help someone you care about. With presence of that care, and by adding a gamification layer,  collaboration and family bonds show indications of improvement, however, in some situation competition appears to harm existing family bond between members of a pair instead of promoting it especially when one member of a pair feels of being let down by the other member of a pair. Strengths and weaknesses of different motivational affordances in-terms of promoting aspects of autonomy, competence, and relatedness are also discussed in details in order to offer insights to both designers and researchers in designing of future interventions.
\section{Thesis Organization}
The following is an organization of this thesis. Chapter 1 is \emph{\textbf{Introduction}} which provides the background information of the problem, research questions, and lastly the contribution of this research to knowledge. Chapter 2 is \emph{\textbf{Literature Review}} which mainly covers  the theoretical underpinning of this research in terms of related work and the conceptual framework that lays a foundation for this research. Chapter 3 is about \emph{\textbf{Study Context}} which situates this work into South African context by providing a rationale why carrying  a study in South African townships was important. Chapter 4 presents \emph{\textbf{Contextual Enquiry}} that was conducted at the beginning of the study to understand how technology is being utilized in general in the context of older adults who are prospective beneficiary users of the technology. In addition this contextual enquiry aimed to understand if there were particular usage of technology that were health related. Chapter 5 is \emph{\textbf{Prototype I}} of where it describes development and evaluation of the first prototype.The contextual enquiry chapter, together ideas grounded by literature served as basis in which preliminary functional requirements were formulated and then used to develop the first prototype. Chapter 6 is \emph{\textbf{Prototype II}}, this was an improvement of the first prototype. This improvement was as a result of both qualitative feedback from evaluation of the first prototype and researchers' observation of context in the field. Chapter 6 is \emph{\textbf{Summative Evaluation}} of where the second prototype was evaluated with a placebo group to discern the isolated the effect of gamification from existing family bonds. Chapter 7 is \emph{\textbf{Overall Discussion}} as it highlights reflection from the three studies and it offers insights in terms of design lessons based on strengths and weaknesses s of the gamified personalized application for intermediated use. in designing and evaluating three different studies.  Chapter 8 is \emph{\textbf{Conclusion and Future Work}} which concludes on takeaways from this research and introduce the basis for future research.         
\begin{flushright}
\end{flushright}
