% Chapter 1

\chapter{Introduction} % Main chapter title

\label{introductionchapter} % For referencing the chapter elsewhere, use \ref{Chapter1} 

\lhead{Chapter 1. \emph{Introduction}} % This is for the header on each page - perhaps a shortened title

%----------------------------------------------------------------------------------------
\section{Background}
Obesity and overweight are currently global health concerns. A systematic review by~\cite{guh2009incidence} concluded that both overweight and obesity are associated with increased incidence of multiple co-morbidities including type 2 diabetes, cancer and cardiovascular diseases (CVD). Co-morbidities that are associated with obesity are likely to inundate health care systems as the number of people who are considered to be either overweight or obese stands to an approximation of  1.3 billion people~\citep{steyn2006chronic}. Health-care systems have also failed to optimally treat chronic conditions such as diabetes due to lack time  to continuously provide  patient  care which is essential in management of such chronic conditions~\citep{quinn2008welldoc}. This calls for innovative citizen-centric  interventions to foster lifestyle changes in order to, both prevent or delay onset of chronic conditions and support patients in self-management of a chronic conditions~\citep{korhonen2010personal,aarsand2012mobile,higgins2016smartphone}.

Advancements in hardware and software technologies have enhanced automation of health self-management processes \citep{arsand:mobile}. Mobile health apps (smart phone based application) that support self-monitoring are becoming useful in augmenting cognitive behaviour therapy~\citep{medynskiy2010salud}. In order for such tools to support health behaviour change, theory based strategies such gamification (for enhancement of motivation), goal setting and feedback (for improvement of self-efficacy)  and SMS reminders are often applied.
\section{Thesis Organization}

\begin{flushright}
\end{flushright}
