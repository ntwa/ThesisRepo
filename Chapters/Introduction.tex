% Chapter 1

\chapter{Introduction} % Main chapter title

\label{introductionchapter} % For referencing the chapter elsewhere, use \ref{Chapter1} 

\lhead{Chapter 1. \emph{Introduction}} % This is for the header on each page - perhaps a shortened title

%----------------------------------------------------------------------------------------
\section{Background}
Obesity and overweight are currently global health concerns. A systematic review by~\cite{guh2009incidence} concluded that both overweight and obesity are associated with increased incidence of multiple co-morbidities including type 2 diabetes, cancer and cardiovascular diseases (CVD). The number of people who are considered to be either overweight or obese stands to an approximation of  1.3 billion people~\citep{steyn2006chronic}. A survey by~\cite{abegunde:theburden} which included a total of 23 low-income and middle-income countries had projected a loss US\$84 billion of economic production in between 2006 and 2015 from heart disease, stroke, and diabetes alone in the absence of any measures in place. Co-morbidities that are associated with obesity are likely to inundate health care systems~\citep{pollak2010s}. In addition to that, at the moment health-care systems have failed to optimally treat chronic conditions such as diabetes due to lack of time to continuously provide  patient  care which is essential in management of chronic conditions~\citep{quinn2008welldoc}. This calls for innovative and citizen-centric  interventions to foster lifestyle changes in order to, both prevent or delay onset of chronic conditions and support patients in self-management of a chronic conditions~\citep{korhonen2010personal,aarsand2012mobile,higgins2016smartphone}.

Advancements in hardware and software technologies have presented opportunities for automation of health self-management processes~\citep{arsand:mobile}. Mobile phones are becoming omnipresent and people carry them most of the time~\citep{mattila2008mobile}; hence their presence brings a ``kairo factor'' in delivery of interventions that target both health promotion ~\citep{pollak2010s} and persuasion~\citep{hsu2014persuasive}. Smartphone based applications are rapidly gaining popularity as effective tools to support delivery of personalized health information~\citep{handel2011mhealth}. Mobile health apps (smart phone based applications) that support self-monitoring are becoming useful in augmenting cognitive behaviour therapy - treatment of behaviour in clinical settings~\citep{mattila2008mobile,medynskiy2010salud}. These apps facilitate data collection of one's health parameters through inbuilt tools such as GPS, accelerometer (body activity sensor), etc; hence present an innovative way of monitoring and improving both health and fitness~\citep{higgins2016smartphone}. In order for such tools to support health behaviour change, theory based strategies such gamification (for enhancement of motivation), goal setting and feedback (for improvement of self-efficacy)  and SMS reminders are often applied~\citep{consolvo2009goal,cole2010text,hamari2014persuasive,hamari2014does,higgins2016smartphone}.
\section{Statement of the Problem}
A review by ~\cite{higgins2016smartphone} presented evidence that these apps can better help patients reach their health and fitness goals.
However, such apps have limitations as they don't support specific interaction models that accommodate sharing of devices and indirect usage. Such mode of interaction are prevalent and relevant in the context of developing world ; hence they may not replicate well to some populations of users~\citep{kaplan2006can,sambasivan2010}, especially the ones that face barriers to direct access to user interfaces or technology~\citep{kumar2015mobile}. This research was exploring of how one could support a personal health informatics technology of which its usage is facilitated by intermediaries users on behalf of beneficiary users (indirect users). Despite a vast amount of literature on \emph{intermediated technology use}, such persuasive technologies have not been extensively explored in this context. Persuasive technologies tend to have their unique design considerations, and intermediated technology use has its socio-technical aspects; hence one has to understand factors to consider and how to go about implementing a useful intervention that can work in such a complex context. This study had two main research questions as presented below. 
\section{Research Contribution}
\section{Thesis Organization}

\begin{flushright}
\end{flushright}
