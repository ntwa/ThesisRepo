% Chapter 1

\chapter{Introduction} % Main chapter title

\label{introductionchapter} % For referencing the chapter elsewhere, use \ref{Chapter1} 

\lhead{Chapter 1. \emph{Introduction}} % This is for the header on each page - perhaps a shortened title

%----------------------------------------------------------------------------------------
\section{Background}
Obesity and overweight are currently global health concerns. A systematic review by~\cite{guh2009incidence} concluded that both overweight and obesity are associated with increased incidence of multiple co-morbidities including type 2 diabetes, cancer and cardiovascular diseases (CVD). Co-morbidities that are associated with obesity are likely to inundate health care systems as the number of people who are considered to be either overweight or obese stands to an approximation of  1.3 billion people~\citep{steyn2006chronic}. Health-care systems have also failed to optimally treat chronic conditions such as diabetes due to lack time  to continuously provide  patient  care which is essential in management of such chronic conditions~\citep{quinn2008welldoc}. This calls for innovative citizen-centric  interventions to foster lifestyle changes in order to, both prevent or delay onset of chronic conditions and support patients in self-management of a chronic conditions~\citep{korhonen2010personal,aarsand2012mobile,higgins2016smartphone}.

Advancements in hardware and software technologies have presented opportunities for automation of health self-management processes~\citep{arsand:mobile}. Mobile phones are becoming omnipresent and people carry them most of the time~\citep{mattila2008mobile}; hence their presence brings a ``kairo factor'' in delivery of interventions that target both health promotion ~\citep{pollak2010s} and persuasion~\citep{hsu2014persuasive}. Smartphone based applications are rapidly gaining popularity as effective tools to support delivery of personalized health information~\citep{handel2011mhealth}. Mobile health apps (smart phone based application) that support self-monitoring are becoming useful in augmenting cognitive behaviour therapy~\citep{mattila2008mobile,medynskiy2010salud}. These apps facilitate data collection of one's health parameters through inbuilt tools such as GPS, accelerometer etc; hence present an innovative way of monitoring and improving both health and fitness~\citep{higgins2016smartphone}. In order for such tools to support health behaviour change, theory based strategies such gamification (for enhancement of motivation), goal setting and feedback (for improvement of self-efficacy)  and SMS reminders are often applied~\citep{cole2010text,hamari2014persuasive,hamari2014does,higgins2016smartphone}.
\section{Statement of the Problem}
A review by ~\cite{higgins2016smartphone} presented evidence that these apps can better help patients reach their health and fitness goals.
However, such apps have limitations as they don't support specific interaction models that accommodate sharing of devices and indirect usage. Such modes of interaction are prevalent and relevant in the context of developing world ; hence they may not replicate well to some populations of users~\citep{kaplan2006can,sambasivan2010}, especially the ones that face barriers to direct access to user interfaces or technology~\citep{kumar2015mobile}. This research was exploring of how one could support a personal health informatics technology of which its usage is facilitated by intermediaries users on behalf of beneficiary users (indirect users). Despite a vast amount of literature on \emph{intermediated technology use}, such persuasive technologies have not been extensively explored in this context. Persuasive technologies tend to have their unique design considerations; hence one has to understand how they can be utilized in the context of intermediaries. This study had two main research questions as presented below. 
\section{Research Contribution}
\section{Thesis Organization}

\begin{flushright}
\end{flushright}
