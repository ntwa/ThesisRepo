% Chapter 1

\chapter{Study Context} % Main chapter title

\label{contextchapter} % For referencing the chapter elsewhere, use \ref{Chapter1} 

\lhead{Chapter 1. \emph{Study Context}} % This is for the header on each page - perhaps a shortened title

%----------------------------------------------------------------------------------------
\section{Obesity}
This section describes obesity from the clinical point of view to show the link between obesity and lifestyle. Obesity is as the  result of a positive imbalance between what is consumed and what is expended by the body of where excess energy is stored in fat cells~\citep{steyn2006chronic}. This positive imbalance is due to two factors and these are (1) overeating especially of energy dense diet (food that is either high in fat or sugar), and (2) a sedentary lifestyle. Results of well-conducted randomized control trials have concluded that the two aforementioned factors increase risk of obesity~\citep{swinburn2004diet}.

This is what will happen if an individual eats energy dense diet. A diet that is high in simple carbohydrates may result into a sharp elevation  of postprandial insulin levels which could lead to increased triglyceride storage in the adipose tissue depots. After a spike of insulin level, the body senses that it has consumed all this energy but it doesn't need the whole of it; hence it stores it into fat cells. If many cycles of storage happen it implies there will be an increase in fat depots and if a person doesn't expend enough energy to exceed what is taken in then the fat will remain in depots. As insulin spikes happens it is likely for a person to feel hungry just after not long enough from eating. This is due to an immediate conversion of all the sugar into energy, then the body converts it into fat upon realizing it exceeds the current required energy. After that the fat is stored into depots; hence there is no sugar left in the blood stream, as the result a person may be tempted to keep on eating to compensate for depletion of glucose~\citep{bouchard1993exercise}. This poses a risk of going into many cycles of eating and probably being predisposed to binge eating disorder~\citep{collins2009behavioral}. Individuals with binge eating disorders lose self-control of their eating patterns. Therefore, as a person becomes obese they are predisposed to losing control of their eating pattern and this may worsen their current situation of obesity. 

Most of the time clinical diagnosis of obesity relies on measure of body mass index reffered to as BMI  which is obtained by getting person’s weight in kilograms (kg) and divide it by their height in meters squared ($m^2$). In some populations, a person is considered obese if their BMI is above 30 kg/m\SP{2}~\citep{steyn2006chronic} while in other populations the cut off point may be different. But there is controversy on using BMI alone as some people can weigh more and may not necessarily be obese (having extra fat), because the extra weight may be due to having extra muscle, bone or water; hence in addition to BMI, a measure of waist circumference is also recommended to clinically diagnose obesity~\citep{janssen2004waist}. 

People with a BMI over 30 kg/m\SP{2} are predisposed to the risk of co-morbidities related to obesity~\citep{de2000clinical}. Therefore, lifestyle modification is crucial in dealing with obesity pandemic. The next section provides more background on the relevance of the problem within South African context.
\section{Context Description}
This study was conducted with participants from low socio economic neighbourhoods of Cape Town. There were four study sites of which one of them was a diabetic and endocrinology clinic which is frequented by patients from low socio economic areas, while the remaining sites were three low socio economic townships in South Africa. 

The rationale for a decision to work with participants from low socio economic neighbourhoods is supported by literature. A review by~\cite{dinsa2012obesity} suggested that in countries with medium human development index of which South Africa is included, groups of low socio economic status  also are affected by obesity, and the trend shows that women of low socio economic status are mostly affected compared to women of high social economic status. Some barriers to adoption of healthy life style that are present in low socio economic communities in the west also appear to recur in low socio economic urban communities in Cape Town, South Africa. In studies that have been conducted in developed countries, it has been revealed that in low social economic areas, there is a presence of some environmental factors that may influence behaviour patterns that predispose individuals to obesity. The environment may play a role of both promoting intake of unhealthy food and discouraging of physical activity. Some of those factors could be lack of access to recreational facilities, or poorly designed built environment which lacks roads for pedestrians, lack of public transport that promotes use of private transport. The environments in which people live in are complex and their individual and their combined elements have a marked effect on behaviour and dietary intake~\citep{swinburn2004diet}. Food choices can be largely influenced by cultural issues and other factors such as price, portion size, taste, variety, and accessibility of foods~\citep{ali2009factors}. The environment may also promote obesity by increasing the likelihood of consuming big portions of meals that are considered high in fat~\citep{hill1998environmental}. These contextual factors that may put individuals at risk of becoming overweight or clinically obese were also somehow present in the context of participants of this research. Many low income neighbourhood in Cape Town are not safe; hence it prevents people from doing simple physical activity such as walking. In addition, the meal outlets in townships sell food that is high in calories. In the contextual enquiry that is reported on the next chapter, majority of the diabetic and obese participants claimed that healthy food in supermarkets is expensive, and in addition they have to eat what the rest of the family eats because they cannot afford to prepare two separate meals. The preliminary study that is reported on the next chapter observed that the notion of healthy food is not quite understood, therefore, the application that was tested in this context helped participants to understand that you can still live a healthy lifestyle by utilizing whatever resources you have. The aim of this research was to explore how to design to support motivation in intermediated use of a personal health informatics in the context of South African low income townships. 

The problem of obesity in South Africa is quite alarming. Statistics have shown that almost 60\% of South Africans are overweight~\citep{ng:global}. Urbanization or emigration of people from upcountry to cities  have been suggested as possible reasons for adoption of unhealthy behaviours as the city lifestyle encourages people to be more sedentary and increase in consumption of caloric dense food~\citep{ali2009factors}.The populations that live in low socio economic areas are facing a lot of challenges. Most of apartheid policies towards health didn't focus towards these populations and some of the current health and economical concerns are as result of amplifications of apartheid social clusters \citep{benatar2013challenges}. Above stated reasons justify why it is crucial to focus on low social economic areas. In addition to health concerns, we have already discussed of how sharing technology and indirect user could hamper utilization of technology in health interventions in the context of low socio ecoonomic areas of developing countries.    
\begin{flushright}
\end{flushright}
