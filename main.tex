%%%%%%%%%%%%%%%%%%%%%%%%%%%%%%%%%%%%%%%%%
% Masters/Doctoral Thesis 
% LaTeX Template
% Version 1.43 (17/5/14)
%
% This template has been downloaded from:
% http://www.LaTeXTemplates.com
%
% Original authors:
% Steven Gunn 
% http://users.ecs.soton.ac.uk/srg/softwaretools/document/templates/
% and
% Sunil Patel
% http://www.sunilpatel.co.uk/thesis-template/
%
% License:
% CC BY-NC-SA 3.0 (http://creativecommons.org/licenses/by-nc-sa/3.0/)
%
% Note:
% Make sure to edit document variables in the Thesis.cls file
%
%%%%%%%%%%%%%%%%%%%%%%%%%%%%%%%%%%%%%%%%% 

%----------------------------------------------------------------------------------------
%	PACKAGES AND OTHER DOCUMENT CONFIGURATIONS
%----------------------------------------------------------------------------------------

\documentclass[11pt, oneside]{Thesis} % The default font size and one-sided printing (no margin offsets)

\graphicspath{{Pictures/}} % Specifies the directory where pictures are stored

\usepackage[comma, sort&compress]{natbib} % Use the natbib reference package - read up on this to edit the reference style; if you want text (e.g. Smith et al., 2012) for the in-text references (instead of numbers), remove 'numbers' 

%\usepackage{booktabs}
%\usepackage{array}
\usepackage{amsmath,array}
\newcolumntype{L}[1]{>{\raggedright\arraybackslash}m{#1}}
%\usepackage{apacite}
\usepackage{multirow}
\usepackage{fixltx2e}
\usepackage{enumerate}
\usepackage{enumitem}
\usepackage{bibentry}
\def\SPSB#1#2{\rlap{\textsuperscript{\textcolor{red}{#1}}}\SB{#2}}
\def\SP#1{\textsuperscript{\textcolor{black}{#1}}}
\def\SB#1{\textsubscript{\textcolor{black}{#1}}}
\newenvironment{myquote}
               {\list{}{\rightmargin   \leftmargin
                        \parsep        0in }%
                \item\relax}
               {\endlist}
\newcommand{\userquote}[2]{\begin{samepage}\begin{myquote} 
     \em{\small{#2\begin{flushright}---#1\end{flushright}}}
   \end{myquote}
   \end{samepage}}
\newcolumntype{x}[1]
            {>{\raggedright}p{#1}}
\newcommand{\tn}{\tabularnewline}

\hypersetup{urlcolor=blue, colorlinks=true} % Colors hyperlinks in blue - change to black if annoying
\title{\ttitle} % Defines the thesis title - don't touch this

\begin{document}
\bibpunct[, ]{(}{)}{;}{a}{,}{,} 
\frontmatter % Use roman page numbering style (i, ii, iii, iv...) for the pre-content pages

\setstretch{1.3} % Line spacing of 1.3

% Define the page headers using the FancyHdr package and set up for one-sided printing
\fancyhead{} % Clears all page headers and footers
\rhead{\thepage} % Sets the right side header to show the page number
\lhead{} % Clears the left side page header

\pagestyle{fancy} % Finally, use the "fancy" page style to implement the FancyHdr headers

\newcommand{\HRule}{\rule{\linewidth}{0.5mm}} % New command to make the lines in the title page

% PDF meta-data
\hypersetup{pdftitle={\ttitle}}
\hypersetup{pdfsubject=\subjectname}
\hypersetup{pdfauthor=\authornames}
\hypersetup{pdfkeywords=\keywordnames}

%----------------------------------------------------------------------------------------
%	TITLE PAGE
%----------------------------------------------------------------------------------------

\begin{titlepage}
\begin{center}

\textsc{\LARGE \univname}\\[1.5cm] % University name
\textsc{\Large Doctoral Thesis}\\[0.5cm] % Thesis type

\HRule \\[0.5cm] % Horizontal line
{\huge \bfseries \ttitle}\\[0.4cm] % Thesis title
\HRule \\[1.5cm] % Horizontal line

\begin{minipage}{0.4\textwidth}
\begin{flushleft} \large
\emph{Author:}\\
{\authornames} % Author name - remove the \href bracket to remove the link
\end{flushleft}
\end{minipage}
\begin{minipage}{0.4\textwidth}
\begin{flushright} \large
\emph{Supervisors:} \\
{\supname} % Supervisor name - remove the \href bracket to remove the link  
\end{flushright}
\end{minipage}\\[3cm]
 
\large \textit{A thesis submitted in fulfilment of the requirements\\ for the degree of \degreename}\\[0.3cm] % University requirement text
\textit{in the}\\[0.4cm]
\groupname\\\deptname\\[2cm] % Research group name and department name
 
{\large \today}\\[4cm] % Date
%\includegraphics{Logo} % University/department logo - uncomment to place it
 
\vfill
\end{center}

\end{titlepage}

%----------------------------------------------------------------------------------------
%	DECLARATION PAGE
%	Your institution may give you a different text to place here
%----------------------------------------------------------------------------------------

\Declaration{

\addtocontents{toc}{\vspace{1em}} % Add a gap in the Contents, for aesthetics

I, \authornames, declare that this thesis titled, '\ttitle' and the work presented in it are my own. I confirm that:

\begin{itemize} 
\item[\tiny{$\blacksquare$}] This work was done wholly or mainly while in candidature for a research degree at this University.
\item[\tiny{$\blacksquare$}] Where any part of this thesis has previously been submitted for a degree or any other qualification at this University or any other institution, this has been clearly stated.
\item[\tiny{$\blacksquare$}] Where I have consulted the published work of others, this is always clearly attributed.
\item[\tiny{$\blacksquare$}] Where I have quoted from the work of others, the source is always given. With the exception of such quotations, this thesis is entirely my own work.
\item[\tiny{$\blacksquare$}] I have acknowledged all main sources of help.
\item[\tiny{$\blacksquare$}] Where the thesis is based on work done by myself jointly with others, I have made clear exactly what was done by others and what I have contributed myself.\\
\end{itemize}
 
Signed:\\
\rule[1em]{25em}{0.5pt} % This prints a line for the signature
 
Date:\\
\rule[1em]{25em}{0.5pt} % This prints a line to write the date
}

\clearpage % Start a new page

%----------------------------------------------------------------------------------------
%	QUOTATION PAGE
%----------------------------------------------------------------------------------------

\pagestyle{empty} % No headers or footers for the following pages

\null\vfill % Add some space to move the quote down the page a bit

\textit{``Education is what remains after one has forgotten what one has learned in school.''}

\begin{flushright}
Albert Einstein
\end{flushright}

\vfill\vfill\vfill\vfill\vfill\vfill\null % Add some space at the bottom to position the quote just right

\clearpage % Start a new page

%----------------------------------------------------------------------------------------
%	ABSTRACT PAGE
%----------------------------------------------------------------------------------------

\addtotoc{Abstract} % Add the "Abstract" page entry to the Contents

\abstract{\addtocontents{toc}{\vspace{1em}} % Add a gap in the Contents, for aesthetics
There is an increasing prevalence of chronic diseases that are associated with living unhealthy lifestyle in both developed and developing world contexts. In order to help combat this unfavourable trend, public health researchers are advocating towards shifting care to the hands of citizens through utilization of personalized interventions . The objective of these initiatives is to support individuals beyond the point of care. ICTs, specifically mobile phones coupled with sophisticated persuasive mechanisms such as gamification or simple strategies such as SMS reminders provide an opportune platform for delivery of personalized interventions that target health behaviour change. In order to support delivery of such personalized interventions, researchers in human-computer interaction have developed an area of research referred to as personal informatics, which focuses on data collection and feedback mechanisms. These approaches aim at supporting individuals to be able to quantify different aspects of their lives through self-reflection. These systems have been developed with motivational affordance to sustain their utilization by end users. However, such systems are developed with only one user in mind of which is a direct user of technology. Such systems may not scale well in contexts of indirect users of technology, meaning people who use technology through a facilitation of a human interface (intermediary user) situated between indirect users (beneficiaries of technology) and technology. Therefore, there was a need to explore how motivational affordance of a personal health informatics could be extended to work in the context of an interaction that requires a collaboration that leads to indirect usage. This was a collaboration between the person helping and the person being helped.

In order to understand design implications for indirect technology use in the context of a personal health informatics, several approaches were used to understand social dynamics that could affect utilization of technology through a human interface in between indirect users and technology. Prototypes of mobile self monitoring applications for physical activity and diet were developed and used as the starting point for uncovering unknown issues. As the result of evaluation of the aforementioned prototypes , the researcher suggested both social technical arrangements and prerequisites that increase the likelihood of success in utilization of such interventions. One of the important social technical arrangement was a prior social relationship between a human interface and a beneficiary of technology through a human interface. Self-determination theory was used to understand how motivation for collaboration between the two sets of users (the human interface and beneficiary user of technology ) could be enhanced. Gamification, a design pattern inspired by games  was found to be the source of a significant increase in perceived competence, an aspect of self-determination theory. Therefore, gamification was found to be a catalyst for increasing collaboration between a human interface and beneficiary user of technology provided that the two users that form a pair had a prior social relationship. The collaborative gamified system showed promising results towards utilization of personal health informatics in the context of indirect usage. The most promising combination of a human interface and beneficiary user is the one that entails family members, possibly a child and a parent.

Despite the success of gamification in increasing perceived competence of the human interface, and hence collaboration between a human interface and beneficiary user, there are some design implications that need to be taken into consideration in order to understand how internalization of collaboration between two members of a pair working together could be improved. This entails exploration of features that support task mastery climate versus those that support ego-involved. Future research could also explore how different styles of parenting could affect the way the intervention is perceived by the two sets of users.
}

\clearpage % Start a new page

%----------------------------------------------------------------------------------------
%	ACKNOWLEDGEMENTS
%----------------------------------------------------------------------------------------

\setstretch{1.3} % Reset the line-spacing to 1.3 for body text (if it has changed)

\acknowledgements{\addtocontents{toc}{\vspace{1em}} % Add a gap in the Contents, for aesthetics

The first people to acknowledge are my research supervisors for their guidance and wisdom in my difficult PhD journey;Dr Melissa Densmore and Prof. Ulrike Rivett. I would like to extent my acknowledgement to the Late Prof. Gary Marsden who was my main supervisor before his demise. May his soul rest in peace. Through him that is where I started to build the passion for human-computer interaction. He was more that just a supervisor. He welcome us as a group to his his home and we became like part of his family. I would like to also to acknowledge his widow, Gil Marsden together two children made us feel at home while in Cape Town.

I would also like to acknowledge all colleagues at ICT4D Lab, University of Cape Town (UCT) whom within their presence the lab was not only an academic environment but we were like one big family supporting each other. Special acknowledgements goes to Christopher Chepken, Raymond Mughwanya, Grace Ssekakubo, Richard Ssembatya, Mvurya Mgala, Thomas Reitmaier, and many more whose names I have not mentioned here but I appreciate their contribution. Also special acknowledgements goes to the ICOMMS research group of which of I was part of.  

I also acknowledge Craig Balfour, a system administrator at the department for his technical support on the infrastructure for hosting the application online. Special thanks also goes Eve Gill, the financial administrator at the Department of Computer Science for being effective in processing funds whenever they were required. Also I would like to thank HPI Research School and Centre of Excellence both at UCT for their generous scholarship, and travel and research grants that were very instrumental in facilitation of my PhD journey. My acknowledgement are extended Memalani NGO in Cape Town who played a role in recruitment of participants at some stage of role. In additional to Mamelani, special thanks go to Mrs Minah Radebe of Langa in  Cape Town and her field worker who were both very instrumental in assisting towards recruitment of participants and facilitation of evaluation in general.

The last group of people I would like to pass my acknowledgement are my country mates at University Cape Town who made my stay in Cape Town appear like a home far from home.   
}
\clearpage % Start a new page

%----------------------------------------------------------------------------------------
%	LIST OF CONTENTS/FIGURES/TABLES PAGES
%----------------------------------------------------------------------------------------

\pagestyle{fancy} % The page style headers have been "empty" all this time, now use the "fancy" headers as defined before to bring them back

\lhead{\emph{Contents}} % Set the left side page header to "Contents"
\tableofcontents % Write out the Table of Contents

\lhead{\emph{List of Figures}} % Set the left side page header to "List of Figures"
\listoffigures % Write out the List of Figures

\lhead{\emph{List of Tables}} % Set the left side page header to "List of Tables"
\listoftables % Write out the List of Tables

%----------------------------------------------------------------------------------------
%	ABBREVIATIONS
%----------------------------------------------------------------------------------------

\clearpage % Start a new page

\setstretch{1.5} % Set the line spacing to 1.5, this makes the following tables easier to read

\lhead{\emph{Abbreviations}} % Set the left side page header to "Abbreviations"
\listofsymbols{ll} % Include a list of Abbreviations (a table of two columns)
{
\textbf{BBM} & \textbf{B}lack \textbf{B}erry \textbf{M}essenger \\
\textbf{BMI} & \textbf{B}ody \textbf{M}ass \textbf{I}ndex \\
\textbf{CHW} & \textbf{C}ommunity \textbf{H}ealth \textbf{W}orkers\\
\textbf{GPS} & \textbf{G}lobal \textbf{P}ositioning \textbf{S}ystem\\
\textbf{HCI} & \textbf{H}uman \textbf{C}omputer \textbf{I}nteraction\\
\textbf{ICT} & \textbf{I}nformation and \textbf{C}ommunications \textbf{T}echnology \\
\textbf{ICTD} & \textbf{I}nformation and \textbf{C}ommunications \textbf{T}echnology and \textbf{D}evelopment\\
\textbf{MMS} & \textbf{M}ultimedia \textbf{M}essaging \textbf{S}ervice\\
\textbf{NEAT} & \textbf{N}on-\textbf{E}xercise \textbf{A}ctivity \textbf{T}hermogenesis\\
\textbf{PDA} & \textbf{P}ersonal \textbf{D}igital \textbf{A}ssistant\\
\textbf{PSD} & \textbf{P}ersuasive \textbf{S}ystem \textbf{D}esign\\
\textbf{RCT} & \textbf{R}andomized \textbf{C}ontrolled \textbf{T}rial\\
\textbf{SIM} & \textbf{S}ubscriber \textbf{I}dentity \textbf{M}odule \\
\textbf{SMS} & \textbf{S}hort \textbf{M}essaging \textbf{S}ervice \\
\textbf{USSD} & \textbf{U}nstructured \textbf{S}upplementary \textbf{S}ervice \textbf{D}ata \\
\textbf{URL} & \textbf{U}niform \textbf{R}esource \textbf{L}ocator\\
%\textbf{Acronym} & \textbf{W}hat (it) \textbf{S}tands \textbf{F}or \\
}

%----------------------------------------------------------------------------------------
%	DEDICATION
%----------------------------------------------------------------------------------------
\clearpage % Start a new page
\setstretch{1.3} % Return the line spacing back to 1.3

\pagestyle{empty} % Page style needs to be empty for this page

\dedicatory{This work is dedicated to my great parents, my father, Andalwisye, and my late mother, Mary for supporting me throughout my academic journey.} % Dedication text




%\addtocontents{toc}{\vspace{2em}} % Add a gap in the Contents, for aesthetics
 %----------------------------------------------------------------------------------------
%	PUBLICATIONS
%----------------------------------------------------------------------------------------
\clearpage % Start a new page

\setstretch{1.5} % Set the line spacing to 1.5, this makes the following tables easier to read

\lhead{\emph{List of Publications}} % Set the left side page header to "Abbreviations"
\listofpubs % Include a list of Abbreviations (a table of two columns)
{
\nobibliography*
%\bibliographystyle{unsrt}
  \section*{List of Publications}
  \begin{enumerate}
    \item \bibentry{katule2016leveraging}
    \item \bibentry{katule2016family}
  \end{enumerate}
}
%\addtocontents{toc}{\vspace{2em}} % Add a gap in the Contents, for aesthetics

%----------------------------------------------------------------------------------------
%	THESIS CONTENT - CHAPTERS
%----------------------------------------------------------------------------------------

\mainmatter % Begin numeric (1,2,3...) page numbering

\pagestyle{fancy} % Return the page headers back to the "fancy" style

% Include the chapters of the thesis as separate files from the Chapters folder
% Uncomment the lines as you write the chapters
% Chapter 1

\chapter{Introduction} % Main chapter title

\label{introductionchapter} % For referencing the chapter elsewhere, use \ref{Chapter1} 

\lhead{Chapter 1. \emph{Introduction}} % This is for the header on each page - perhaps a shortened title

%----------------------------------------------------------------------------------------
\section{Background}
Obesity and overweight are currently global health concerns. A systematic review by~\cite{guh2009incidence} concluded that both overweight and obesity are associated with increased incidence of multiple co-morbidities including type 2 diabetes, cancer and cardiovascular diseases (CVD). The number of people who are considered to be either overweight or obese stands to an approximation of  1.3 billion people~\citep{steyn2006chronic}. A survey by~\cite{abegunde:theburden} which included a total of 23 low-income and middle-income countries had projected a loss US\$84 billion of economic production in between 2006 and 2015 from heart disease, stroke, and diabetes alone in the absence of any measures in place. Co-morbidities that are associated with obesity are likely to inundate health care systems~\citep{pollak2010s}. In addition to that, at the moment health-care systems have failed to optimally treat chronic conditions such as diabetes due to lack of time to continuously provide  patient  care which is essential in management of chronic conditions~\citep{quinn2008welldoc}. This calls for innovative and citizen-centric  interventions to foster lifestyle changes in order to, both prevent or delay onset of chronic conditions and support patients in self-management of a chronic conditions~\citep{korhonen2010personal,aarsand2012mobile,higgins2016smartphone}.

Advancements in hardware and software technologies have presented opportunities for automation of health self-management processes~\citep{arsand:mobile}. Mobile phones are becoming omnipresent and people carry them most of the time~\citep{mattila2008mobile}; hence their presence brings a ``kairo factor'' in delivery of interventions that target both health promotion ~\citep{pollak2010s} and persuasion~\citep{hsu2014persuasive}. Smartphone based applications are rapidly gaining popularity as effective tools to support delivery of personalized health information~\citep{handel2011mhealth}. Mobile health apps (smart phone based applications) that support self-monitoring are becoming useful in augmenting cognitive behaviour therapy - treatment of behaviour in clinical settings~\citep{mattila2008mobile,medynskiy2010salud}. These apps facilitate data collection of one's health parameters through inbuilt tools such as GPS, accelerometer (body activity sensor), etc; hence present an innovative way of monitoring and improving both health and fitness~\citep{higgins2016smartphone}. In order for such tools to support health behaviour change, theory based strategies such gamification (for enhancement of motivation), goal setting and feedback (for improvement of self-efficacy)  and SMS reminders are often applied~\citep{consolvo2009goal,cole2010text,hamari2014persuasive,hamari2014does,higgins2016smartphone}.
\section{Statement of the Problem}
A review by ~\cite{higgins2016smartphone} presented evidence that these apps can better help patients reach their health and fitness goals.
However, such apps have limitations as they don't support specific interaction models that accommodate sharing of devices and indirect usage. Such mode of interaction are prevalent and relevant in the context of developing world ; hence they may not replicate well to some populations of users~\citep{kaplan2006can,sambasivan2010}, especially the ones that face barriers to direct access to user interfaces or technology~\citep{kumar2015mobile}. This research was exploring of how one could support a personal health informatics technology of which its usage is facilitated by intermediaries users on behalf of beneficiary users (indirect users). Despite a vast amount of literature on \emph{intermediated technology use}, such persuasive technologies have not been extensively explored in this context. Persuasive technologies tend to have their unique design considerations, and intermediated technology use has its socio-technical aspects; hence one has to understand factors to consider and how to go about implementing a useful intervention that can work in such a complex context. This study had two main research questions as presented below. 
\section{Research Contribution}
The contribution of this research is mainly on human factors to consider when designing a personal health informatics (PHI) for intermediated used. In this dissertation it is suggested that rather that designing a PHI only for the beneficiary, one can design for intermediated use, explicitly acknowledging the presence of more than one user of the application.
\section{Thesis Organization}

\begin{flushright}
\end{flushright}

% Chapter 1

\chapter{Literature Review} % Main chapter title

\label{literaturereview} % For referencing the chapter elsewhere, use \ref{Chapter1} 

\lhead{Chapter \ref{literaturereview}.\emph{Literature Review}} % This is for the header on each page - perhaps a shortened title

%----------------------------------------------------------------------------------------
\section{Behaviour Change Support Technologies}
B.J. Fogg was one of the early pioneers to formalize behaviour change technology as an area of research, and coined a term “captology” which is an acronym for Computers As Persuasive Technologies (CAPT-ology) with focus on the planned persuasive effects of computer technologies~\citep{fogg1999persuasive}. In persuasive systems, persuasion is intentional and usually implemented through persuasive stimuli hence providing the ability to persuade~\citep{hamari2014persuasive}. Persuasive technologies have application in domains such as health-care, education and training, and in environmental sustainability behaviours, etc.

Behaviour change technologies research has evolved from digital interventions that appeared in early 1990s which were basically for intervening behaviours in the preventive health area primarily through reminders; to persuasive technologies that were supported with various software functionality that utilize approaches such as social learning or comparison etc; to the current behaviour change support systems (BCSS) which provide models and frameworks for designing and evaluating persuasive technologies \cite{langrial2012digital}. A BCSS is defined by \cite{Oinas-Kukkonen:foundation}  as ``a socio-technical information system with psychological and behavioural outcomes designed to form, alter or reinforce attitudes, behaviours or an act of complying without using coercion or deception''.

\cite{Oinas-kukkonen:psd} argues that models for user acceptance of information systems such Technology acceptance model (TAM) are not sufficient in understanding adoption and utilization persuasive technologies, therefore, models have been proposed based on seminal work and these models have outlined strategies of which one can use in order to design technologies that are persuasive.  

The first model was proposed by \cite{fogg2009behavior} asserted that for a person to perform a targeted behaviour, he or she must (1) be sufficiently motivated, (2) have the ability to perform the behaviour, and (3) be triggered to perform the behaviour, and recommended a behaviour grid that one can use to design persuasive technologies \cite{fogg2009behavior2}. In this behaviour grid, persuasive strategies are matched  to targeted behaviours. 

Foggs's work was extended by \cite{Oinas-kukkonen:psd} who proposed a more comprehensive model known as a ``\emph{Persuasive System Design}'' (PSD) model which provides guidance about how to; analyse the persuasion context by focusing on the intent of persuasion and contexts of use, user and technology; classify persuasive features, and identify persuasion strategies to use i.e. whether to use a direct or indirect route of persuasion. The PSD model outline 28 designed principles discerned into the following five categories: (1) \emph{primary task support}, which includes activities such as, reduction of complex behaviours into simple tasks, guiding the user through experiences while persuade along the way, tailoring of persuasive information to factors relevant to a user group, personalization of content, and self-monitoring for users to keep track of their performance towards their specified goals; (2) \emph{dialogue support}, which includes praises, rewards, reminders, similarity, liking, and social roles; (3) \emph{system credibility support}, which includes trustworthiness, expertise, surface credibility etc; and (4)\emph{social support}, which includes social learning, social comparison, and competition.

The PSD model was further extended by providing an Outcome Change (O/C) matrix to use when analysing an intent of persuasion~\citep{Oinas-Kukkonen:foundation}. The O/C matrix matches the type of change that needs to be applied with a specific outcome. A change could either be compliance (C) or behaviour (B) or attitude (A) change. While an expected outcome could be forming, altering, or reinforcing any of the aforementioned type of change. The extended PSD model with O/C matrix is called BCSS as mentioned in the classes of behaviour change systems above. BCSS is considered to be the foundation for studying persuasive systems and it is meant to provide a base for analysis, design, and evaluation of persuasive technologies.   

\section{Behaviour Change Technologies for Health}
Health promotion and management drives most initiatives to design persuasive technology because of an increasing prevalence of lifestyle-chronic diseases. A systematic review by \cite{hamari2014persuasive} on ability of persuasive technologies to persuade, included 95 studies of which approximately 47\% targeted health and exercise domains alone.

\cite{chatterjee2009healthy} classified three generations  of technological evolution of hardware and software used to implement persuasive technologies in health. The first generation started to emerge from 1960's and it was characterized by the prescriptive nature of information flow from physician, health care provider, or technology-based system to a health care recipient. Decades worth of research has shown that phone-based or simple messaging technologies can improve the quality of health care management and clinical outcomes. The second generation is characterized by the descriptive nature of information interaction between a user and the persuasive technologies and examples of such systems include interactive Web sites, personal data assistants (PDAs) that allow activity recording, and simple sensors that record and report basic health parameters. The third generation extends second generation by providing body-wearable sensors that support advanced health monitoring, use of context-aware computing that can use information of a person's location within their environment and the determination of their activity at the moment of measurement, and real-time exchange of information to support “just in time” messaging.  While the second generation utilized PCs and later cellphones, the third generation is dominated mostly by cellphones.

Digital interventions are the ones that have received most appraisal because of existing randomized clinical trials. The evidence of their dominance in public health is demonstrated by the preponderance in publications that report on the use web based interventions integrated with SMS text-messaging on clinical settings. Existing systematic reviews~\citep{cole2010text,fjeldsoe2009behavior,krishna2009healthcare} report more on the use of SMS reminders and feedbacks on diabetes self-management, smoking cessation, and weight reduction therapy. However, published literature on digital interventions is being criticized of lacking adequate information on how individual systems were designed, hence such systems are usually poorly described since most work is being published by public health practitioners without involvement of computer scientists~\citep{Oinas-Kukkonen:foundation}.
 
This research is focusing on data collection and reflection for persuasion. In the next sub-section, personal informatics are discussed together with their applications in personalization of health interventions. 
\section{Personal Informatics for Health Behaviour Change}
Personal informatics is a class of interactive applications that support users to improve self-understanding of various aspects of their life by providing technological means that allow individuals to both collect and analyse personal data related to habits, behaviours, and thoughts~\citep{li2011personal,li2012personal}. Personal informatics augments the activity of self-reflection by complementing individuals in storing events that can hardly be recalled due to limitations in humans' memory~\citep{li2010stage}. The goal of a personal informatics system is to support an individual to have a better understanding of some personal behaviour i.e. in promotion of positive behaviors such healthy eating~\citep{lee2006pmeb}, recycling~\citep{comber2013designing}, energy conservation~\citep{seligman1977feedback}, etc.

The core research agenda on personal informatics systems is on data collection and self-reflection~\citep{li2011understanding}. Research questions focus on how to design interfaces to support collection of personal data in an effortless manner and how  better users could reflect on their personal data through some feedback mechanisms. In personal informatics, reflection is usually supported through visualizations such as bar charts or using other more persuasive formats such as a virtual fish bowl that could represent level of physical activity (i.e Ubifit\citep{klasnja2009:using}). Researchers have also tried to come up with frameworks for designing personal informatics  that integrate theories from both behaviour change and social networks~\citep{kamal2010understanding}. There are systems that have implemented social comparison features or competitions with others through social networks i.e. BinCam \citep{comber2013:designing,comber2013bincam} which uses social norms influence as a motivation strategy to encourage individuals within a household to be more conscious of their recycling behaviours by comparing themselves with other households.

\cite{li2010stage} proposed a model for understanding how people use personal informatics by transitioning between the following five stages;preparation, collection,integration, reflection, and action. In this model, it is emphasized the importance of identifying barriers at each stage which may also cascade to later stages,therefore the design process should be taken in a holistic approach which includes iterations between stages. This model aims at helping with the process of designing a personal informatics.~\cite{li2011understanding} proposes that \emph{reflection} phase should address six questions that users ask themselves when engaging with their personal data and these questions are based on; status towards achieving  their goal, history for the purpose of discovering patterns that are crucial  to the preferred behaviour, formation of goals to facilitate in attaining a preferred behaviour; discrepancies between their behaviour and goal; context of past behaviour in order to discover patterns, and discovering of factors that may affect their behaviours. These questions provide design implications for motivating tools to support self-reflection. Also,~\cite{macleod2013personal} proposes factors that drive motivation of chronically ill people in engaging with their personal data as; curiosity, and self-discovery of what is happening in their health.~\cite{epstein2015lived} proposed a different model of understanding how people use personal informatics systems by integrating these systems into their daily lives. This was referred to as lived informatics model of personal informatics~\citep{epstein2015lived} which extended  ~\cite{li2010stage}'s model, by splitting the preparation stage into \emph{deciding to track} and \emph{selecting tool} processes, and combining collection,integration,and reflection into tracking and acting. This model also include further stages beyond tracking and acting and these were lapsing, and resuming tracking.  From lapsing, issues that contribute to discontinuation or intermittent usage are explored, while in resuming tracking, issues such as switching of tools, incorporation of previous history/data while resuming to use are explored on this stage.  

Guiding an end user through an action/acting stage for the objective of minimizing barriers in execution of this stage as suggested by a personal informatics model ~\citep{li2010stage}, can be perceived as an attempt to nudge individuals towards certain behaviours. There has been a debate from HCI research community of whether behavioural nudges are ethically accepted or not as some researchers are proposing a more neutral approach while others recommend application of intervals of behavioural nudges upon tracking (collection and reflection) activity.  For instance,~\cite{munson2012mindfulness} advocates that the focus on personal informatics  should be towards enabling end users to better know owns behaviour instead of applying behavioural nudges, and suggests that adoption should be voluntary. Also in \cite{epstein2015lived}'s model it is highlighted that sometimes people use personal tracking systems for other reasons beyond behavioural change goals such as instrumental benefits (i.e. to get rewards from location trackers like Foursquare), or out of curiosity.  However, \cite{epstein2015lived}, still  shows that in most usage that are related to health domain i.e in physical activity, behaviour change goal is a dominant motivational factor \citep{epstein2015lived}, therefore, suggestions on what actions an end user should take are inevitable. In addition to this, technology is considered not to be neutral~\citep{Oinas-kukkonen:psd}. Technology has a capability of presenting social cues that trigger emphatic responses from humans~\citep{foggpersuasivebook}. If no action is recommended, still an action can come from within a person using the system as the result of self-reflection. According to~\cite{fogg1998persuasive} cited in~\cite{Oinas-kukkonen:psd}, intent of persuasion can originate from either of the three sources which are: (1) from the people who create or produce interactive technology; (2) from people who give access to or distribute the interactive technology to others; and (3) from the people who use or adopt an interactive technology. The intent of persuasion may also come from within personal informatics systems that don't recommend or suggest any actions and this through cognitive dissonance after self-reflection. People like their views of the world to be consistent and It also assumed that people always make rational and informed decisions~\citep{Oinas-kukkonen:psd}. By using personal informatics, individuals' decisions can be improved by being be able to see the discrepancies between their desired behaviours versus their performance~\citep{comber2013designing}. If there are inconsistencies, then a cognitive dissonance is introduced which may mediate a change of attitude or behaviour in order to restore consistency between beliefs and actions~\citep{Oinas-kukkonen:psd}. Therefore, an act of tracking(collection and reflection) itself can mediate a behaviour change through cognitive dissonance. The motivation of usage of personal informatics in domains such as health and finance has been found to be related to a behaviour change goal~\citep{epstein2015lived}. From this perspective, a basic personal informatics system with simply self-monitoring support can be viewed as a persuasive technology in contexts such as personal health and finance, because of its ability to trigger cognitive dissonance which can be considered as a persuasive stimulus. 

Personal informatics which are also referred to as wellness applications in some literature are useful in promotion of a healthy lifestyle~\citep{korhonen2010personal}. Personal informatics can be applied in prevention of onset of chronic conditions by motivating healthy individuals to change their lifestyle. Personal informatics for health falls under personalized health. Personalized health is about tailoring and personalizing, health interventions to individual needs of a patient~\citep{mccallum2012gamification}. Specifically, personal informatics can provide support for cognitive behaviour therapy (CBT) in public health settings~\citep{mattila2008mobile}, for individuals who are clinically obese~\citep{nih2000practical}. Literature from public health suggests that recording of personal behaviour (self-monitoring) as an essential part of behaviour modification therapy~\citep{nih2000practical}. Health self-management programs usually ask participants to keep records of their activities, physiological variables and other health-related data; personal informatics applications can make this process simpler and easier~\citep{medynskiy2010salud}. For instance, participants may record their daily calorie intake,and then have graphs that show trends of how far they have gone with reducing their intake. Instead of a person noting down this information on a paper diary, and drawing a graph afterwards, personal informatics can support this process to make it more efficient. The essence of self-monitoring is to promote self-awareness of one’s behaviour. That consciousness is fostered through behaviour observation. And behaviour observation can be achieved through behaviour recording.  

From literature, there is a wide range of mobile phone based personal informatics systems for promotion of physical activity, and also some of the systems include dietary advice functionality. Some of these are specifically for chronically ill patients e.g. ``A few Touch Application'' which targeted individual with  \emph{Type 2 Diabetes}~\citep{arsand:mobile}, and a system described by \cite{arteaga2010:persuasive} that targeted teenagers with weight management issues. There also exists systems that target general populations, and used for promoting healthy eating habits and engagement in physical activity e.g. Fish'n'Steps\citep{lin2006:fish}, UbiFit Garden\citep{klasnja2009:using}, Wellness diary\citep{mattila2008mobile}, ActivMon\citep{burns2012using}, PmEB\citep{lee2006pmeb}, iCrave\citep{hsu2014persuasive}, etc.  These systems promote behaviours that are beneficial in preventing weight gain or encouraging weight loss. These systems operate by facilitating logging of personal behaviour’s data, and one of the key strategies in attaining a behavioural change involves setting of personal health goals, which has been recently used in many developed systems. This idea is derived from a goal setting theory~\citep{strecher1995goal}. An example of a goal could be to walk for at least 30 minutes every day or to increase the number of times a person eats fruits and vegetables or to reduce the amount of starch in a meal. Li et al (2011a) shows that individuals who use personal informatics usually transition between two phases of behaviour change and these are self-discovery and maintenance. In the self-discovery stage individuals collect a lot of data they can use to discover patterns in their behaviours. After discovering of a pattern they can move to the maintenance stage. The maintenance stage entails setting of a personal goal and monitoring of a status towards achieving that goal. Users don't stay permanently in one phase. It is possible for an individual in the maintenance phase to go back to discovery phase if there is a new unknown pattern that has emanated and appears to affect their behaviour.

Despite such tremendous developments in the field of personal informatics for health behaviour change, most of these systems are designed for the developed world context. Even existence of randomized clinical trials on utilization of a simple technology such as SMS is largely dominated by countries from developed world~\citep{cole2010text}. From HCI point of view, engagement with personalized systems is currently considered to be more personal from data collection to reflection processes. These applications are personal in the essence of ownership of hardware, applications, data stored in application, and the process of interacting with the system for both data collection, and reflection. The HCI context of the existing applications may not be versatile in developing world perspective especially in low income communities of where  both sharing in usage of technology and indirect usage through intermediary or proxy users are common~\citep{kaplan2006can,sambasivan2010}. HCI in the developing world is a complex relationship between technology, multiple users, indirect stakeholders, observers, and bystanders~\citep{parikh2006}. An interaction model that assumes one phone/device one person might not always be feasible in such contexts. Also in many contexts, interaction with technology may not be direct; intermediation by another person occurs when the primary user is not capable of using a device entirely on their own \citep{sambasivan2010}. For users with limited technology literacy or education, direct access to a user interface might not even be feasible~\citep{parikh2006}, hence intermediation might be the only means for these people to be able to perceive the benefits derived by the proliferation of mobile phones or any other ICTs. While many people in developing world context might lack textual and digital literacy, low-income communities are diverse and often there are some members who have competent skills to operate technology \citep{sambasivan2010}, and these people may be able to help others to benefit from technology usage.

The complexity of usage through intermediaries is beyond help on the spot \citep{sambasivan2010}, hence, it cannot be merely solved by endeavours to simplify the user interface. In exploring of why intermediated technology use is beyond help on spot, one has to look at the notion of collectivist societies. In collectivist societies, people engage in tasks in group formation. For instance, India is considered to be a collectivist society of where individuals are prone to group orientation towards tasks~\citep{parikh2006}. This encourages usage of technology through human intermediaries. In such usage at least two users are involved in one interaction process. There is much more complexity on factors that influence intermediated technology use and hence it cannot be simply explained by existing interaction models from computer supported collaborative work~\citep{parikh2006}. Sukumaran et al (2009) emphasizes the importance of having a better understanding of locally specific interaction models to address culturally influenced issues in using information technology throughout the developing world. Intermediated interaction in an example of such interactions that needs to be clearly understood.

Frequency of usage of personal informatics may vary among different domains, with the ones targeting physical activity being used more frequent (on daily basis), while other domains usage is from once a week and beyond~\citep{epstein2015lived}. Therefore, for a context where an end user needs help, motivation to use, is no longer just for this user but also it has to consider the person helping. This research was particularly focusing on how a personal health informatics system  designed for a personal use can be adapted in the context where two sets of users are being involved in an interaction process (the first one being a beneficiary of that technology, meaning a person receiving help on an interaction task to both collect and self-reflect on their personal data, while the second one is an intermediary user, a person providing assistance to a beneficiary user).  

In the next sub-section it is discussed of how intermediaries have been used in other health behaviour change interventions in the context of developing world.
\section{Intermediaries in Supporting Health Behaviour Change}
In the context, of health behaviour change, typical examples of human intermediaries is on utilization of community health workers to provide access to health information to less privileged individuals in resource-constrained environments.  In many ICTD projects, CHWs, access health information on behalf of communities in which direct access to health resources is not possible, and are an effective bridge between communities and government-based resources~\citep{katule2016leveraging}. One project in India empowered community health workers (CHWs) with mobile phones that contained persuasive messages, and these messages added credibility of community health workers to persuade pregnant and postnatal women together with their relatives on maternal health issues~\citep{ramachandran2010mobile,ramachandran2010research}. Another project in Lesotho~\citep{molapo2013software}, empowered rural health trainers with a software application for creation of digital  training  content, voice-over images  that can be used by low-literate CHWs to train clients in the villages. While the main objective of these podcasts was for training purposes, upon CHWs showing them to their clients, there were unintentional persuasive effects that motivated these clients to get tested for diseases such as tuberculosis. In the two aforementioned projects, CHWs were acting as human access to information that had a persuasive effect. A challenge with utilizing CHWs is that their availability is limited to fewer visits in intervals of weeks or months, hence they may not be suitable for a technology such as a personal health informatics of which its beneficiaries may need to engage with a system more frequent. 
\section{Factors Affecting Motivation of Intermediary Users}
According to~\cite{heeks1999tyranny}, cited in~\cite{bailur2012complex}, human intermediaries bridge a gap between what the poor have and what they would need in order to use ICTs. Therefore, human infrastructure within ICTD context plays an instrumental role in facilitating information and communication access in low income communities~\citep{sambasivan2010human}. Intermediaries play roles beyond being translators of policies to the ground level, therefore, it has been suggested that, lack of understanding of their position and motivation played a part in contributing to failure of public access venues (PAVs) as ICTD interventions for bridging the digital divide~\citep{bailur2010liminal}. Without the presence of these facilitators in PAVs, the groups that are excluded from access due to their age, socio-economic status, level of education/literacy, gender, disability or caste are more likely to face barriers in accessing information~\citep{ramirez2013infomediaries}. Factors that affect and shape the outcome of facilitating information and communication access have been explored, for instance, \cite{bailur2010liminal} used structuration theory\citep{jones2008giddens}, to understand the contradiction and conflict between the different networks of intermediaries i.e. how they play a liminal role with stakeholders of PAVs and multimedia centres (i.e. NGO or government, donor agency on one side and community on the other side). \cite{bailur2012complex} argues that PAVs' intermediaries should not be taken for granted in the space of ICTD because they play a complex position of brokers and translators as they assume multiple identities to different stakeholders of which their roles are constantly negotiated and performed within these multiple constructed networks. Another study is by \cite{ramirez2013infomediaries} which investigated how human factors such as empathy and technical skills of infomediaries influence the outcomes of the process of infomediation to users at PAVs. 

The ecosystem of utilization of intermediaries in PAVs or other community centres has also been examined through lens of HCI. Focus on HCI has been on engagement of all layers of users involved in intermediated interactions.~\cite{parikh2006}'s study in India provided a taxonomy of intermediated information tasks from HCI perspective;  of which  different modes of access were distinguished, and each one of them was suggested to have its own design requirements. These modes of access were: (1) cooperative, of whereby several users fairly collaborate without domination by a single or fewer users; (2) dominated interactions, of whereby users collaborate but they is one or fewer users who dominate others in manipulating user interfaces; (3) intermediated interactions, this whereby the first user manipulates interfaces while the  rest of users are just observing what is happening; and (4) indirect interactions, of where one or multiple users are being assisted to interact with a system without being being present or observing while manipulation of user interfaces is taking place. ~\cite{sukumaran2009intermediated} conducted an experiment that investigated how social prominence of an intermediary versus technology in a computer kiosk affects perceived information characteristics and attitudes towards an interaction by a beneficiary user/secondary user and found out that when the technology was more visible and an intermediary did not monopolize access (situation of social equality), beneficiaries tended to feel more engaged and positive. 

Although intermediaries in public access venues are considered as policy implementers on the ground level through working with communities, their position is very complex as they are the bottom of the hierarchy but they are also perceived not to be part of the community, hence they cannot specifically identify with a certain group since their roles are adapted according to circumstances~\citep{bailur2010liminal}. Motivation of intermediaries in this context of PAVs is negotiated relative to their particular network. A different ethnography study by~\cite{sambasivan2010}, explored the dynamics of intermediation beyond public access venues (\emph{i.e in inherent home, or community settings that involve neighbours and family members as intermediaries} --these intermediaries are more embedded to the community as they are considered part of it). \cite{sambasivan2010} presented three distinct forms of intermediated interactions: inputting intent into the device in proximate enabling, interpretation of device output in proximate translation, and both input of intent, and interpretation of output in surrogate usage. This study also discussed both social mediators for motivation for intermediation (i.e interpersonal trust or prior social rapport, a give and take economy, social structures (i.e. access constraints due gender, economic status, tendency of reliance on others etc)), and design implications to enhance engagement of user (primary and secondary users) such as : reorientation of technology  to allow sharing between primary and secondary users for asymmetric engagement; and supporting persistent storage of information for retrieval at later stages by beneficiary users. The study also proposed that measurement of use should go beyond ownership to also consider those who benefit without direct usage. 

The concept of informal help in technology use within family and social network settings is not an exclusive  phenomenon for developing world only; it is present in developed world as well. For instance,~\cite{poole:chh} cited in~\cite{katule2016leveraging}, examined the dynamics of computer help-seeking and giving behaviors in the context of family and social networks settings, their findings indicated that an important factor that  encourages help-seeking behaviors is availability of unlimited help provided as a part of a longer-term relationship, while in the case of help-givers, they are motivated by a sense of being accountable to their family members and friends.

Existing literature on informal help to use technology doesn't suggest of how one support the two users to negotiate interaction in case a beneficiary user requires frequent engagement which may be the case for a personal health informatics. Relying on motivation of beneficiaries users alone may not be sufficient as intermediaries may not see the urgency in providing help on regular basis. To rely on existing intrinsic motivation of intermediaries alone may not be sufficient for engagement. Intermediaries may not always be physically present or not willing to help in regular. For instance, in a study by ~\cite{kiesler:twi} about informal help, parents were sometimes hesitant to seek help from their children to avoid negative experiences, (i.e being bashed for being nagging parent and too reliant). Therefore, this hesitation may be as the result of past encounters of negative experience. Therefore, it crucial to understand of ways to enhance enhance engagement of intermediaries in order encourage them to support ongoing use of a personal health informatics.

In the next sections, a self-determination approach to motivation is discussed, and user experience strategies that can be used in the context of intermediaries. 
\section{Enhancement of Motivation Through Self-Determination Theory Approach}
Motivation is categorized into intrinsic motivation (i.e. inherently embedded with ones' values and goals), and extrinsic motivation (i.e. doing something because of expecting some external outcome)~\citep{ryan2000intrinsic}. Self-determination theory (SDT)\citep{deci1985:intrinsic},a well grounded theory of human motivation, is concerned with how individuals develop interest to engage with certain activities that were once considered uninteresting~\citep{ryan2000intrinsic}. The focus of SDT is on the process of internalization of external motivated behaviour~\citep{ryan2000intrinsic}. The theory stipulates different levels of internalization for self-regulation of uninteresting but important activities to become interesting, of where by these levels are classified into four stages namely,(1)external regulation, (2) introjected regulation, (3) identified regulation, and (4) integration~\citep{ryan2000intrinsic}. In external regulation, individuals self-regulate because of an external outcome; in introjected, individuals self-regulate as an attempt to raise their self-worth with respect to others, in identified, individuals have put value into an activity they try to self-regulate activity as important; and in integration, individuals have fully assimilated the self-regulation to their core values and beliefs.  Integration shares values with intrinsic motivation although it is not intrinsic motivation since its self-regulation is due an external outcome while in intrinsic motivation self-regulation of an activity is as the result of an activity itself being interesting~\citep{ryan2000intrinsic}. The internalization process is governed by social and environmental factors on which individuals function~\citep{ryan2000:self,lee2015:relating}.

SDT suggests that an intrinsically motivated activity is performed out of satisfying some psychological needs, therefore for an uninteresting activity to become interesting through external rewards, social factors must provide support for the following three basic psychological needs; competence, relatedness, and autonomy~\citep{ryan2000intrinsic}. SDT has two sub-theories namely; (1) cognitive evaluation theory, which focuses on supporting of the aforementioned basic psychological needs and (2) organismic integration theory, which focuses on the process of internalization by discerning extrinsic motivators that can foster intrinsic motivation from extrinsic motivators that can harm intrinsic motivations~\citep{ryan2000:self,lee2015:relating}.

SDT has been used to understand motivation on various activities or behaviours such as gaming\citep{ryan2006:motivationalpull}, physical activity\citep{power2011:obesity}, tobacco cessation\citep{williams2006:testing}, energy saving \citep{webb2013:self}, etc. This study brings the motivation pull of gamification in encouraging intermediaries to assist in utilization of personal health informatics. In the next section, gamification is discussed from perspective of self-determination theory.
\section{Self-Determination Theory Support in Gamification}
~\cite{ryan2006:motivationalpull} investigated motivational pull behind video games using self-determination theory and found that needs for autonomy, competence, and relatedness independently predict enjoyment and future game play. This motivational aspect of gaming has attracted researchers to explore its usage beyond gaming context. Gamification is the use of game design elements in non-game context~\citep{deterding2011game}. This implies that gamification is used outside game context to increase interest on uninteresting but instrumental activities (i.e. physical activity, crowd-sourcing tasks such as image annotation etc..).~\cite{sailer2013:psychological} also used self-determination theory to understand the motivational pull of these game elements that can be used in non-game contexts of which their research attempted to provide examples of matching game elements to motivation mechanisms; for instance badges can be used to foster a sense of competence, while a leader board can be used to foster a sense of relatedness as it puts emphasis on collaboration among members of different teams.

There is a debate of whether the use of gamification can bring gaming experience. According to~\cite{deterding2011game}, users of gamification can socially construct a perception of whether gamification qualifies as a game or not. The paper further argues that gamification brings gameful experiences of whereby there is an experiential ``flicker'' between gameful, playful, and other modes of experience and engagement (i.e. engagement for instrumental purposes, etc..).~\cite{seaborn2015:gamification} argues that the goal of gamification is different from games, therefore, it should rather be considered as an act of integrating user experience to an activity outside game context. The use of gamification is influenced by social factors such as social influence,recognition, reciprocal benefit, and network exposure, therefore it is important to have a community of people who are committed to the goals that the gamification promotes~\citep{hamari2013social}.

\section{Games and Gamification in Personalized Health Interventions}
Following the diffusion of video games in many digital devices of which these games are solely used for entertainment purposes, there is an increasing interest on the potential of such entertaining platforms in influencing positive changes in health behaviours~\citep{king2013gamification}. A systematic review of research on persuasive technology has shown a recent precedented increase of gamification literature~\citep{hamari2014persuasive}. Traditionally, games were sedentary, but nowadays there are games that require user to exert body movements in order to play a game. Researchers are exploring of how games could be adapted to engage people with personalized interventions for health~\citep{mccallum2012gamification}. Use of games for health includes exer-games, games with purpose (serious games), and gamification.
\subsection{Exergames}
\subsection{Serious Games for Health}
\subsection{Gamification for Health}

\begin{flushright}
\end{flushright}

% Chapter 1

\chapter{Study Context} % Main chapter title

\label{contextchapter} % For referencing the chapter elsewhere, use \ref{Chapter1} 

\lhead{Chapter 1. \emph{Study Context}} % This is for the header on each page - perhaps a shortened title

%----------------------------------------------------------------------------------------
\section{Obesity}
This section describes obesity from the clinical point of view to show the link between obesity and lifestyle. Obesity is as the  result of a positive imbalance between what is consumed and what is expended by the body of where excess energy is stored in fat cells~\citep{steyn2006chronic}. This positive imbalance is due to two factors and these are (1) overeating especially of energy dense diet (food that is either high in fat or sugar), and (2) a sedentary lifestyle. Results of well-conducted randomized control trials have concluded that the two aforementioned factors increase risk of obesity~\citep{swinburn2004diet}.

This is what will happen if an individual eats energy dense diet. A diet that is high in simple carbohydrates may result into a sharp elevation  of postprandial insulin levels which could lead to increased triglyceride storage in the adipose tissue depots. After a spike of insulin level, the body senses that it has consumed all this energy but it doesn't need the whole of it; hence it stores it into fat cells. If many cycles of storage happen it implies there will be an increase in fat depots and if a person doesn't expend enough energy to exceed what is taken in then the fat will remain in depots. As insulin spikes happens it is likely for a person to feel hungry just after not long enough from eating. This is due to an immediate conversion of all the sugar into energy, then the body converts it into fat upon realizing it exceeds the current required energy. After that the fat is stored into depots; hence there is no sugar left in the blood stream, as the result a person may be tempted to keep on eating to compensate for depletion of glucose~\citep{bouchard1993exercise}. This poses a risk of going into many cycles of eating and probably being predisposed to binge eating disorder~\citep{collins2009behavioral}. Individuals with binge eating disorders lose self-control of their eating patterns. Therefore, as a person becomes obese they are predisposed to losing control of their eating pattern and this may worsen their current situation of obesity. 

Most of the time clinical diagnosis of obesity relies on measure of body mass index reffered to as BMI  which is obtained by getting person’s weight in kilograms (kg) and divide it by their height in meters squared ($m^2$). In some populations, a person is considered obese if their BMI is above 30 kg/m\SP{2}~\citep{steyn2006chronic} while in other populations the cut off point may be different. But there is controversy on using BMI alone as some people can weigh more and may not necessarily be obese (having extra fat), because the extra weight may be due to having extra muscle, bone or water; hence in addition to BMI, a measure of waist circumference is also recommended to clinically diagnose obesity~\citep{janssen2004waist}. 

People with a BMI over 30 kg/m\SP{2} are predisposed to the risk of co-morbidities related to obesity~\citep{de2000clinical}. Therefore, lifestyle modification is crucial in dealing with obesity pandemic. The next section provides more background on the relevance of the problem within South African context.
\section{Context Description}
This study was conducted with participants from low socio economic neighbourhoods of Cape Town. There were four study sites of which one of them was a diabetic and endocrinology clinic which is frequented by patients from low socio economic areas, while the remaining sites were three low socio economic townships in South Africa. 

The rationale for a decision to work with participants from low socio economic neighbourhoods is supported by literature. A review by~\cite{dinsa2012obesity} suggested that in countries with medium human development index of which South Africa is included, groups of low socio economic status  also are affected by obesity, and the trend shows that women of low socio economic status are mostly affected compared to women of high social economic status. Some barriers to adoption of healthy life style that are present in low socio economic communities in the west also appear to recur in low socio economic urban communities in Cape Town, South Africa. In studies that have been conducted in developed countries, it has been revealed that in low social economic areas, there is a presence of some environmental factors that may influence behaviour patterns that predispose individuals to obesity. The environment may play a role of both promoting intake of unhealthy food and discouraging of physical activity. Some of those factors could be lack of access to recreational facilities, or poorly designed built environment which lacks roads for pedestrians, lack of public transport that promotes use of private transport. The environments in which people live in are complex and their individual and their combined elements have a marked effect on behaviour and dietary intake~\citep{swinburn2004diet}. Food choices can be largely influenced by cultural issues and other factors such as price, portion size, taste, variety, and accessibility of foods~\citep{ali2009factors}. The environment may also promote obesity by increasing the likelihood of consuming big portions of meals that are considered high in fat~\citep{hill1998environmental}. These contextual factors that may put individuals at risk of becoming overweight or clinically obese were also somehow present in the context of participants of this research. Many low income neighbourhood in Cape Town are not safe; hence it prevents people from doing simple physical activity such as walking. In addition, the meal outlets in townships sell food that is high in calories. In the contextual enquiry that is reported on the next chapter, majority of the diabetic and obese participants claimed that healthy food in supermarkets is expensive, and in addition they have to eat what the rest of the family eats because they cannot afford to prepare two separate meals. The preliminary study that is reported on the next chapter observed that the notion of healthy food is not quite understood, therefore, the application that was tested in this context helped participants to understand that you can still live a healthy lifestyle by utilizing whatever resources you have. The aim of this research was to explore how to design to support motivation in intermediated use of a personal health informatics in the context of South African low income townships. 

The problem of obesity in South Africa is quite alarming. Statistics have shown that almost 60\% of South Africans are overweight~\citep{ng:global}. Urbanization or emigration of people from upcountry to cities  have been suggested as possible reasons for adoption of unhealthy behaviours as the city lifestyle encourages people to be more sedentary and increase in consumption of caloric dense food~\citep{ali2009factors}.The populations that live in low socio economic areas are facing a lot of challenges. Most of apartheid policies towards health didn't focus towards these populations and some of the current health and economical concerns are as result of amplifications of apartheid social clusters \citep{benatar2013challenges}. Above stated reasons justify why it is crucial to focus on low social economic areas. In addition to health concerns, we have already discussed of how sharing technology and indirect user could hamper utilization of technology in health interventions in the context of low socio ecoonomic areas of developing countries.    
\begin{flushright}
\end{flushright}

\input{Chapters/ContextEnqChapter}
% Chapter 1

\chapter{Prototype I} % Main chapter title

\label{prototype1chapter} % For referencing the chapter elsewhere, use \ref{Chapter1} 

\lhead{Chapter \emph{Prototype I}} % This is for the header on each page - perhaps a shortened title
%----------------------------------------------------------------------------------------
\section{Development of the Prototype}
The initial task was to develop the first version of the application prototype. The prototype had features that allow monitoring of physical activity and diet of an individual. The manipulation of user interfaces of the app was specifically targeted to help givers/intermediary users. The prototype was designed to encourage one help-giver to work together with one help-seeker by forming one pair of users. In order to make the act of helping to be perceived as both important and meaningful by intermediary users, the first message displayed when opening the app was explicit that an intermediary user is helping someone they know to manage their wellness. In the case of motivating ongoing use, the app had included gamification features of where each pair could be awarded points, badges, nice looking gardens, and fish tanks. The essence of having these features was to enable pairs of users to have a set of challenges that will promote competence which is one of the core aspects of self-determination theory. In addition to the aforementioned features, within each pair of users' garden and fish tank, there was a Facebook social plug-in that could allow members from different teams/pairs to comment on or like each other. The presence of these social features was to promote relatedness which is also one of the aspects of self-determination theory. Facebook groups were also utilized to give feedback or remind users to engage with the application. The first prototype didn't explicitly have any functionality to support autonomy. Ideally, the information flow on the high-level representation of the system to encourage intermediated use was designed as depicted on Figure \ref{figure:prototype_1}. A web app was developed using a combination of several web technologies such as HTML, JQuery, JavaScript, and CSS on the client side while the server side was implemented using Django Python framework. Sample screen-shots of the prototype are shown on Figure \ref{figure:prototype_1_screens}. Authentication was done through Facebook accounts of intermediary users. 

\begin{figure}[htbp]
  \centering
    \includegraphics[width=0.8\textwidth]{Figures/prototype_1.png}
    \rule{35em}{0.5pt}
  \caption{Information flow in the first prototype.}
  \label{figure:prototype_1}
\end{figure}

\begin{figure}[htbp]
  \centering
    \includegraphics[width=0.8\textwidth]{Figures/Version1/Prototype1Screenshots.png}
    \rule{35em}{0.5pt}
  \caption{Sample screen-shots of the first prototype}
  \label{figure:prototype_1_screens}
\end{figure}

This prototype aimed at encouraging increase in physical activity and  decrease of sedentary behaviours through informing individuals (help-seekers) of their current behaviour trends. In addition, also individuals could monitor whether they were eating healthy or not. Some visualization techniques that are similar to this have been previously used  in systems that were designed for direct users such as ``Fish'n'Steps''\citep{lin2006:fish}, ``Ubifit''\citep{klasnja2009:using},  and ``Few Touch Application''\citep{arsand:mobile}. The idea of using a plate to showing distribution of meals' nutrition mimicked a practice that was used by dietitians at the hospital were we conducted the contextual inquiry reported in the previous chapter. In this design metaphor, the amount of each food group that needs to be consumed is presented as a bisector of a pie chart. 
\section{Prototype Evaluation}
There was a slightly change of plan  due to contention about the study design and implications to ethics when dealing with patients as per requirements of Faculty of Health Sciences Human Research Ethics Committee (FHS-HREC) of University of Cape Town. FHS-HREC was interested to have clinical outcomes as one of the expected outputs. These requirements were going to increase the scope of the work of which its main area of contribution was supposed to be in human-computer interaction. Also, the envisaged technology was only under research; hence it was not expected to be comprehensive enough and ready for any clinical trials even after completions of all evaluations that were meant to be carried out throughout this research. Therefore, I had to look for a different group of participants outside hospital settings. This involved reapplication of ethical approval to an institution body different from the first one that approved the study reported in the previous chapter. The ethical approval was obtained from Faculty of Science Research Ethics Committee (FSREC) of University of Cape Town (see Appendix \ref{AppendixB}).

In order to evaluate the aforementioned prototype, I recruited participants through help from an NGO
based in Cape Town called ``\textbf{\textit{Mamelani Projects}}''\footnote{http://www.mamelani.org.za/}. This NGO was carrying out outreach programs on health education in less privileged
communities. Mamelani was training women on issues of HIV/AIDS, nutrition, and gender equality. 

The NGO helped to recruit participants among people who were part of their trainings in Philippi township. Criteria for recruitment were as follows: (1) participants that were mid-aged and above (contextual enquiry suggested that most prospective beneficiaries could be above mid-aged) and (2) participants must had an intermediary person willing to work with them (someone they trusted or close to them). The NGO identified the targeted participants that met the inclusion criteria. A total of six adult participants were recruited of which both were women above mid-aged (\textgreater= 35 years of age). Each one of the adults brought one intermediary to form a pair. Three intermediaries were girls in between 19-23 years of age. The remaining three intermediaries were boys aged between 14 and 19 years of age. 

Participants were informed of their rights. Participants were also informed of which of their information will be collected by the study. Both adults and their respective intermediaries signed consent forms except for intermediaries who were minors, these signed assent forms that were approved by their respective parents/guardians. 

The next step entailed training of intermediaries on how to use the app. The application was deployed to the field from the end of October 2014 to beginning of December 2014. In order to limit potential complications from deploying the intervention on multiple platforms, each pair of participants was given one Android phone (Samsung GT-S5300) running the pedometer app. Participants were required to utilize the web application hosted at University of Cape Town by using a web browser built into their phones. In order to retain participants in the study each intermediary participant and each beneficiary participant who remained as part of the study received ZAR30 (\~US\$3) worth of airtime every week for the duration of the study. I collected qualitative feedback in the middle and at the end of the study. All the names used in qualitative feedback are just pseudonyms to protect identity of participants.
\section{Findings}
Observations and qualitative feedbacks from the six cases/pairs are presented below. The first three pairs consisted of parents working their children while the remaining three pairs it was just an adult working with either a close or distant relative in each pair.
\subsection*{\textbf{Pair 1: Mother and Son}}
This pair consisted of a boy aged 17 years of age working with his mother. The boy is referred here with the name ``\textbf{Jabulani}'' and his mother is referred with the name ``\textbf{Nandipha}''. Jabulani lived with his mother and other siblings. He appeared to be passionate in engaging with a cellphone. He described the intimate relationship he had with his cellphone.

\userquote{\textbf{Jabulani}} {``There is a time I lost my cellphone. It was like the end of the world to me because I didn't have anything to play with''}

The excerpt above shows that a cellphone could be a source of intrinsic motivation for young users in this context. 

He mentioned that he felt happy helping his mother. He also articulated the reason for helping his mother as that he felt it was his duty since the mother took care of him when he was growing up. 

\userquote{\textbf{Jabulani}} {``I feel happy when I am helping her because she helping me when I was growing up so it is my turn to help her''}

\textbf{Nandipha} also felt very happy being helped by her son and she mentioned that she thinks her son is very brilliant more than her in technology and she gets to know things because of him. 

Jabulani and Nandipha were the first one to engage with the app for at least three different days. All of a sudden their use stopped. When I asked Jabulani together with his mother of challenges that might have prevented them from using the app, their responses indicated that were more conscious about airtime as it was one of the reasons of why they didn't login more often. So there were times where they ran out of airtime. They thought having more data bundles might solve the problem. Another reason for why they didn't use the app more often is associated with lack of competitions from other pairs and also they had accomplished the highest challenge within few days. In the first few days they were so curious about attaining the highest badge. Jabulani claimed that her mother was walking up and down so that they reach that goal. Jabulani discussed with his mother that they must reach that goal in  a week. They managed to reach the goal and there was no more boundary to break.

Badges were one source of motivation to this pair. Jabulani felt motivated by the badges and he was persuading his mother to work harder so that they reach the highest badge which was Queen/King.

\userquote{\textbf{Jabulani}} {``We talked me and my mum that we must not reach only for today but for the whole week. That was our goal to reach the queen and the king for the whole week. I remember a day that was the best day. My mum woke up very early to walk around, to go to Philippi Makasikava [A location within the neighbourhood] just to reach that goal''}

Also Jabulani noticed something on the scoreboard. He and his mother were there leading. Another team (Pair 3) was in the second position. The third position was held by Pair 4. Then after few days Jabulani noticed that Pair 4 moved from a third position to a second position. So Jabulani told his mother that ``we must not drop down because they (Pair 4) are going to reach us''. In that context competition with others was a source of motivation for Jabulani. Although Jabulani was helping his mother but he thought like the ownership of the winning process as theirs because he used the word “We” all the time to imply that he felt that he was part of that process. Additionally, Jabulani enjoyed information displayed by a botanical garden and a fish bowl. He explained why he was so interested in such abstract visualizations. When he was growing up he used to watch cartoons. So when he sees those pictures of trees and fish he feels he is part of that process of making those images/cartoons. So drawing fish and trees through their team's performance motivates him more and he tells his mother that they must have more fish in the bowl. Also the idea of fish in the bowl motivated Nandipha to walk more. She mentioned  that she didn't like to see her bowl empty without any fish, so she tried to walk more steps as she could. These ideas of abstract visualization such as fish bowls/tanks and garden have been previously used in systems that involved only one user on interaction with user interfaces \citep{lin2006:fish, klasnja2009:using}, the only difference in this context is that, the same ideas were extended and tested with two users who were collaborating to attain one objective. 
\subsection*{\textbf{Pair 2: Mother and Son}}
``\textbf{Dumisani}'' was a 14 years of age who lived with his mother, ``\textbf{Kholiwe}''. Dumisani was acting as an intermediary for Kholiwe. Dumisani and Kholiwe used the system for only the first three days and they dropped out. On responding to the question of why was it the case, Kholiwe mentioned that it was the inability to access the system every time they tried out. The web page was always giving them time-outs and this discourages them from trying. But it was also observed that Dumisani was not very familiar with Facebook authentication as he didn't have an account before. I created one account for him of which it wasn't very helpful. The decision in using Facebook authentication was based on an assumption that all intermediaries may have Facebook accounts which was not the case. However, despite technical challenges this pair also showed enthusiasm in using the app.
\subsection*{\textbf{Pair 3: Mother and Daughter}}
``\textbf{Zama}'' who was a 20 years of age was supposed to act as an intermediary for her mother, ``\textbf{Fikile}''. Since the daughter appeared to be interested to help her mother, then one would think that intermediation is possible. Unfortunately, the two lived in different houses and they never used the system at all. Their contact to discuss issues about the system was limited as Zama was raising a toddler at that time. In addition, Fikile appeared to had some expertise in using technology as she already was using Facebook, therefore she was interested to learn how to operate the system on her own but she failed because of the situation of her daughter. However, the system had been set up only to allow Facebook account for Zama. 
\subsection*{\textbf{Pair 4: Close Relatives}}
``\textbf{Lindiwe}'' was a young girl in her early twenties. Lindiwe was acting as an intermediary for her auntie ``\textbf{Nceba}'' but they never lived together in the same house. The pair had not been interacting with the application at all. When I interviewed Nceba of why they were not using the app, her response was that she doesn't know how to operate it on her own and her intermediary seems not  to be around most of the time. She is curious to access the information but her intermediary seemed not to be cooperative. So she suggested to bring someone else who was also a close relative.
\subsection*{\textbf{Pair 5: Close Relatives}}
``\textbf{Neliswa}'' was a girl aged 23 years of age. Neliswa  was acting as an intermediary for her auntie ``\textbf{Nkosazana}'' but they lived together in the same house. The pair had not been interacting with the application at all. I never had a chance to interact with this pair since they were not available. But from a personal observation during recruitment, Neliswa appeared to be less interested in the intervention even though she signed the consent form to participate. 
\subsection*{\textbf{Pair 6: Distant Relatives}}
``\textbf{Nkululeko}'' was a boy aged 19 years of age. He was acting as an intermediary for her distant relative, ``\textbf{Noluthando}. Nkululeko and Noluthando didn't live so close to each other but they did see each other more often. System logs showed that this particular pair had not been engaging with the application. I interviewed both of them to find why that was the case. Nkululeko pointed out number of things. The first one was that he tried to access the application a couple of times but he was unable to proceed after login. He was using his personal phone. I checked his personal phone and I discovered that his web browser was the problem. He had never tried to do it using the experimental phone that was in possession of Noluthando. We tried together and it was okay on the other phone.  But in addition to phone's problems he claimed to be busy with school. Despite him being busy, and his phone not being able to support the application,  the absence of things like reciprocal benefits  and a close social relationship with the beneficiary, might be the cause for his low intrinsic motivation in engaging. The previous user, Jabulani had a problem of accessing the application using his personal phone but he made an effort to access using the phone  given to his mother. So the closeness/bond of the two sets users might be the base for the network effect to happen.
\subsection{Discussion}
Only two pairs of users engaged with the system for more than two days. Both of these two pairs consisted of a beneficiary and an  intermediary living in the same house. These pairs consisted of mothers working with their sons. One of these two pairs was very motivated and enthusiastic about the system. But after some time they also got bored because they were not getting any competition from other teams and they had attained all the challenges within a short period of time. In a third pair, a girl was working with her mother but they were not living together so it was difficult for her to commit to the application. Intermediaries from the remaining two pairs showed little enthusiasm in the project. There were three hypotheses for this lack of enthusiasm to engage with the system and these were: (1) due to lack of motivation to engage with the system; (2) lack of a prior social relationship between the two users within each pair; and (3) Low frequency of interactions between the two users within a pair due to distance.

There was an indication that a prior social relationship is instrumental for intermediaries to perceive value in the act of helping their beneficiary users. In this case the interaction became more meaningful. It also becomes easier for the two users within a pair to negotiate for interaction. For three pairs that consisted of mother/son or mother/daughter there was a tendency for the two users to show the eagerness of working together. For three pairs where members of a pair didn't  have a parent/child relationship, intermediaries showed little enthusiasm in the intervention. Another advantage of a prior social relationship comes to sharing of phones. It was observed that it was easier for an experimental phone to move from a beneficiary to an intermediary when a parent and child were involved in a pair. There was a form of trust that existed between the two users with a prior social relationship. In addition, intermediaries had more authority when they were helping a person who was close to them. If a pair with a prior social relationship needed to interact with the app, then the frequency of these interactions depended on proximity between the two users. For cases where they cohabited or lived nearby it increased the chances of face to face meetings and negotiation for interaction. For instance, ``\textbf{Zama}'', an intermediary participant aged 20 years old was working with her mother. The challenge with this pair is that they didn't cohabit and Zama had a toddler hence this lowered her ability to participate in the intervention.  

Prior social relationship also worked in parallel with the presence of interest to use the app/gamified features. A combination two factors played a some role in  encouraging the two users within a pair to collaborate when they met. For instance , in the case of of Jabulani and Nandipha (mother and son), they discussed about strategies to win against other pairs. Although Jabulani was helping his mother but he thought like the ownership of the winning process is theirs because he used the word ``We'' all the time to imply that he felt that he was part of that process. In addition, if intermediaries are motivated they can become persuaders of beneficiaries that they have a prior social relationship with as it can be seen on Jabulani who encouraged his mother to walk more steps.

There were some drawbacks in utilization of this prototype. From participants' perspective , intermittent internet connectivity, insufficient airtime, less motivated intermediaries, and lack of competition/challenges with others in the gamified system were the key issues mentioned. Other factors include how often the two user meet (Whether they cohabit or they meet more often), and reminders were not timely. I had very high expectations that Facebook reminders will work for this community. An assumption was that every intermediary is probably using Facebook. Actually this was not the case. There were some intermediaries who had never engaged with Facebook before. And the ones who had engaged with Facebook were not doing it so often as I anticipated. For instance ``Jabulani'' had never used Facebook before. ``Jabulani'' was only engaging with Facebook at most twice in a week. Therefore, Facebook might not be an on time-platform for delivery of reminders or any messages to intermediaries in this context. Findings from this informative evaluation led to another iteration in the design. It also informed the manner in which evaluations in chapters \ref{prototype2chapter} (Prototype II) and \ref{summativeevalchapter} (Summative Evaluation) were conducted.
%\begin{flushright}
%\end{flushright}

\input{Chapters/Prototype2Chapter}
        % Chapter 1

\chapter{Summative Evaluation} % Main chapter title

\label{summativeevalchapter} % For referencing the chapter elsewhere, use \ref{Chapter1} 

\lhead{Chapter \ref{summativeevalchapter}. \emph{Summative Evaluation}} % This is for the header on each page - perhaps a shortened title

%----------------------------------------------------------------------------------------
\section{Recruitment of Participants}
With help of a research assistant who was a resident of Langa,  we managed to recruit a total of fourteen adult participants (beneficiary users) through a convenient sampling technique. Approaching of potential participants for recruitment was done in October 2015. We recruited these participants from two townships in Cape Town: Langa, and Athlone. In Langa there were five adult participants while in Athlone there were nine adult participants. The was only one male out of fourteen adult participants. Gender imbalance was as the result of females being more eager to participate than males. This trend of domination of females was also apparent in the study reported in Chapter \ref{prototype2chapter}. The average age of these adult participants was 44.21 years with a standard deviation (S.D) of 9.99 years and their age range of was between 26 and 60 years of age. The highest education level was secondary school and the lowest was the last grade of primary school. After recruitment of adults they were requested to elect one of their children/grand children to become their respective intermediary user; hence forming a pair of users. As the result fourteen children were recruited. Thirteen adult participants teamed up with their children, the remaining adult teamed up with her grand child. Children participants had a mean age of 15.42 (with a standard deviations of 2.06 years). The age range children participants was between 12  and 20 years of age. There was a gender balance on intermediary participants. 

I gave out detailed information of what the study was all about to both intermediary and beneficiary participants. I informed them about different modes of which I will be collecting data and these approaches included usage logs, questionnaires, and interviews. All the beneficiary participants signed informed consent forms agreeing to be part of the study. Since all intermediaries were under 21 years of age (legal age for giving consent in South Africa is 21 and above), they signed assent forms which were also signed by their respective parents/guardians who were part of the study.

One day was allocated for training intermediary participants on how to use the ``Family Wellness App''. In addition, each intermediary user was given a user manual. After the training, I gave out one Android phone (Samsung GT-S5300) to each pair of participants. These phones were installed with two natives apps. The first app was a pedometer and the second one was the main ``Family Wellness App''. The ``Family Wellness App'' loaded all its content from a web application hosted remotely from a server hosted at University of Cape Town. Each beneficiary participant was required to carry around the phone that was given to a pair in order for the pedometer app to count their footsteps. The two apps (main app and pedometer) were made available to the pairs of participants for a total period of six weeks. Each pair of participants provided the service provider's number of the SIM card that was inserted on their given Android phone. I allocated 1.3 GB of data to each SIM card. In addition each beneficiary participant was given a total of ZAR (South African Rands) 240 for the duration of the study. That amount of money covered for compensation for participants' transportation and time spent in data collection activities such as administering of questionnaires and interviews. The details of the experiments are outlined on the next section. 
\section{Experiments}
The main objective was to carry out experiments to evaluate the effectiveness of gamification in motivating  intermediated use. As the result, the experiments entailed comparisons between the app that was not gamified and the gamified app. The app that was not gamified was simply a diary that could enable individual pairs  to track diet/nutritional components of meals consumed by a beneficiary participant, ans secondly  to enable pairs to track footsteps walked by a beneficiary participant. The second version of the application was an extension of logbook meaning that it included all the features in logbook  with an addition of a rewards/gamified subsystem. The experiments took place from the mid-October 2015 to the end of November 2015.  The details of how experiments were designed and how data were collected are presented on the next sub-section.
\subsection{Experiment Design}
The study used ``within-group'' design for the experiments which used the same group of participants for both logbook and gamification. The rationale for this design was to reduce interference from confounding factors and in addition to reduce the cost of recruitment as the same group was for both control and intervention.he only problem with this approach is the learning effect and in addition, it lengthens the duration of a study. Pairs of participants were to two separate groups that started with and finished with opposite experimental conditions. These groups were referred to as experimental sequences. In the first experimental sequence there were pairs that started with logbook and later switched to gamification. In the second experimental sequence there were pairs that started with gamification and later they were switched to logbook. I used the following abbreviations ``LG'' and ``GL'' to refer to respective first and second experimental sequences. The rationale behind having two experimental sequences was to give equal chances to both experimental conditions to start at the beginning and if there is there was the learning effect on the outcome then it could have affected both experimental conditions equally. Therefore the learning effects from the two experimental conditions were expected to cancel each other.

A total of seven pairs of participants were assigned to the LG group while the remaining seven pairs were assigned to the GL group. Both groups spent the first four weeks in their respective first experimental conditions of which were logbook app for the LG group and gamified app for the GL group. After 27 days (four weeks) each group was switched to a different experimental condition; hence after the switching, the LG group started using the gamified app while the GL group started using the logbook app. The second phase of the experiment lasted for a total of 14 days (two weeks). 

The explanation of why four weeks in phase 1 and two weeks in phase 2 is as follows. Initially the plan was to have the time spent on each experimental condition, be in three(3) weeks intervals, but phase 1 had extended beyond its allocated block of three weeks up to the fourth week as pairs of participants were not available for midpoint assessments at the end of the third week. Therefore, I carried out the midpoint assessment at the end of the fourth week. After the aforementioned assessment, pairs that were in gamified app were switched to logbook app, and those that were in logbook app were switched to gamified app. It was not feasible to extend phase 2 to go up to four weeks as phase 1 due complexity that was going to be introduced as the result of rescheduling duration of experiments from six to eight weeks. Rescheduling was impossible because it was approaching December of where most people travel for holidays, therefore, gathering participants during that time may have been impractical. As the result this shortened the duration of phase 2 to two weeks. 
\subsection{Research Methods}
Data collection was carried out through a triangulation of application's usage logs, questionnaires, and interviews. In the next subsections, the details of each of the three approaches are provided. 
\subsubsection{Family Wellness App Logs}
Application's logs consisted of information such; when there were users' activities on the app, which pair was accessing the app at that time, and what functionality was being accessed by that pair. Logs were categorized to their respective experimental conditions. Usage was characterized by two dimensions; (1) the number of sessions of where the app had user's activity from a particular pair of users, and (2) the number of times certain features were accessed by a particular pair of users (impressions). A new session was defined as a period of detection of user's activity in an absence of any activity from this user/pair in the past one hour or more. Impressions was defined as the number of times in which a certain feature had been viewed by a pair. So if multiple clicks by one user/pair on one feature happened within an interval of less than one minute between two consecutive clicks on the same feature, such multiple consecutive clicks were grouped as a single impression. If clicks on the same feature differed by a minute or more then the current click was treated as a new impression while the previous click belongs to the previous impression. Therefore, if the time difference between clicks on the same feature is beyond one minute, then it was assumed that the user had gone away or move to a different feature and they are coming to this feature for another iteration of clicks on the feature as the previous iteration of clicks within that feature they are visiting had finished. The purpose of computing pair's total impressions on each feature was to understand where users of the app were likely to go among many options of gamification features. i.e. leaderboard (score board), score badge, botanical garden, and fish tank (fish aquarium). 
 
There were two major analyses conducted on usage logs. The first analysis was a comparison of number of sessions between gamified app and logbook in two steps. This particular analysis had two dimensions.  The first dimension entailed comparing the daily total number of sessions between the two experimental conditions for 41 days of experiments. This was a comparison between daily total sessions of all users in logbook app and daily total sessions of all users in gamified app. The second dimension entailed a pairwise comparison of users' sessions in between logbook and gamification conditions (repeated measures for usage during logbook and usage during gamification by the same pair of users). In order to ensure this comparison of usage doesn't get affected by the difference in experimental durations between phase 1 (period before switching of experimental conditions) and phase 2 (period after switching of experimental conditions), I opted to use a relative (normalized) number of sessions  as a unit of measurement. In this case I used the number of sessions per day since the number of sessions on a particular experimental condition was relative to the number of days on which a particular version of the app was made available to the pair of users -- duration of deployment. This duration of deployment differed between the two  experimental conditions from participants within the same experimental sequence. For pairs that were assigned to LG group four and two weeks were  spent in logbook  and gamification respectively while for pairs that were assigned to GL, four and two weeks were spent in gamification and logbook respectively. This implies if a pair that belongs to an experimental group (sequence), \emph{Z} spent an \emph{X} amount of days in an experimental condition \emph{Y} and had \emph{n} number of sessions, then their normalized number of sessions in experimental condition \emph{Y} will be \emph{\textbf{n}} sessions divide by \emph{\textbf{X}} days. So in order to make a comparison on repeated measures of usage (usage of individual pairs between logbook and gamification conditions) the hypothesis of interest was as follows :

\begin{enumerate}
\item{Hypothesis 1}
\begin{itemize}
\item{H\SB{0}}:There is no difference in normalized number of sessions between a logbook app and gamified app
\item{H\SB{A}}:There is a difference in normalized number of sessions between a logbook app and gamified app
\end{itemize}
\end{enumerate}

In the process of doing the aforementioned comparison, a decision was made to exclude four pairs in testing the aforementioned hypothesis. These were pairs that faced hurdles on utilizing the app as the result of technical glitches and this affected their ability to fully experience both experimental conditions. The excluded pairs are listed on Table \ref{table:usageproblems}.

\begin{table}[h!]
  \begin{center}
    \caption{Excluded pairs as the result of technical glitches}
    \label{table:usageproblems}
	\begin{tabular}{|l|l|l|p{6cm}|}
		\hline
		&Pair&Group&Problem\\
		\hline
		1&Pair-A&GL-sequence &App failed to load due to poor internet signal\\
		\hline
		2&Pair-B&GL-sequence&This pair didn't have data bundles due misallocation. \\
		\hline
		3&Pair-C & LG-sequence.& Their pedometer had never transmitted any throughout the duration of the study.\\
		\hline
		4&Pair-D & LG-sequence.& Pedometer worked for a while before it stopped transmission of data.\\
	\hline
	\end{tabular}
  \end{center}
\end{table}

For \textbf{Pair-A}, the app failed to load every time the intermediary user tried to use it. As the results the intermediary participant complained to the researcher that the app was being unstable. After a follow up it was observed that in the house where this particular pair lived in there was a poor Internet signal, hence the app was always failing to load most of the time and this frustrated the intermediary user. The second pair (\textbf{Pair-B}), data was allocated to the wrong phone number at the beginning of the experiments but they never reported on time. These two pairs (Pair-A and Pair-B) had the lowest usage days which were 2 and 3 days respectively of which usage happened only in gamification condition.

For the last two pairs (\textbf{Pairs-C} and \textbf{Pair-D}) on Table \ref{table:usageproblems}, pedometers had malfunctioned; this affected their ability to experience gamification as other pairs. The two pairs started to experience the technical glitches while still in logbook condition before being switched to gamification condition.

The second major analysis on usage logs examined the impact of gamification features on internalization. Perceived usefulness is a predictor of internalization. This analysis was done by contrasting how number of impressions on certain gamification features affected perceived usefulness. The term``impressions'' has already been defined above as pair's frequency in viewing a particular feature. The total number of impressions on each of gamification feature was calculated for the number of days in which pairs were assigned to gamification condition. In this analysis all fourteen pairs were considered including the four that had technical glitches. Since this comparison only involved one experimental condition; hence all features had equal chances of being accessed provided that the user/pair had managed to get into the app. Also some feedbacks were delivered through SMS meaning that all pairs had received such interactions regardless of whether the App was accessible or not; therefore, it is under the assumption that pairs that had received SMS feedback could be in position of judging the perceived usefulness of the app. In addition to that, intermediary users were frequently interacting to each other in face to face manner to talk about things in the app. Another assumption is that regardless of the app presence intermediary users would perceive helping their parents on app usage as something that as something that is meaningful. Hence if the user/pair had never viewed a particular gamification feature, they would still get zero as the number of impressions in that feature. The objective was to understand how gamification features affected internalization. On literature review section (Chapter \ref{literaturereview}), four types of internalization of behaviour regulation were highlighted. Therefore, this analysis also view internalization with respect to the four types of behaviour regulation. The questionnaires that were used to capture aspects of internalization (perceived usefulness) are mentioned on the next sub-section together with other sub-scales of intrinsic motivation.

\subsubsection{Questionnaires}\label{methodsquestionnaire}
The research team administered questionnaires at baseline, mid-line (during switching of experimental conditions), and end-line. These questionnaires targeted both intermediary and beneficiary participants. The list of questionnaires that are described below are attached at the Appendices \ref{AppendixC} and \ref{AppendixD}.
\begin{enumerate}
\item{\textbf{Intermediaries}}
Intermediaries had three questionnaires that were administered at baseline, midline, and endline.
 
\begin{itemize}
\item{\textbf{Baseline Questionnaire}}: Intermediaries participants' baseline questionnaire had three sections. The first section captured demographic information such as age, gender, and number services/apps used on cellphones.The second section included an IMI (Intrinsic Motivation Inventory) questionnaire  to assess participants' intrinsic motivation in using cellphones. The third section included an IMI questionnaire to assess participants' intrinsic motivation in helping their parents with cellphone based tasks. 

\item{\textbf{Midline Questionnaire}}: Intermediaries participants' midline questionnaire had only one section which included an IMI questionnaire  to assess participants' intrinsic motivation in using the family wellness app.

\item{\textbf{Endline Questionnaire}}: Intermediaries participants' endline questionnaire had only one section which included an IMI questionnaire  to assess participants' intrinsic motivation in using the family wellness app.
\end{itemize}

\item{\textbf{Beneficiaries}}

\begin{itemize}
\item{\textbf{Baseline Questionnaire}}: Beneficiary participants' baseline questionnaire had four sections. The first section included an IMI questionnaire  to assess participants' intrinsic motivation in using the family wellness app. The third section included an IMI questionnaire to assess participants' intrinsic motivation in self-monitoring of diet/nutrition. The fourth section included an IMI questionnaire to assess participants' intrinsic motivation in self-monitoring of physical activity.

\item{\textbf{Midline Questionnaire}}:Beneficiary participants' midline questionnaire had three sections. The first section included an IMI questionnaire  to assess participants' intrinsic motivation in using the family wellness app. The third section included an IMI questionnaire to assess participants' intrinsic motivation in self-monitoring of diet/nutrition.The fourth section included an IMI questionnaire to assess participants' intrinsic motivation in self-monitoring of physical activity.

\item{\textbf{Endline Questionnaire}}: Beneficiary participants' endline questionnaire had three sections. The first section included an IMI questionnaire  to assess participants' intrinsic motivation in using the family wellness app. The third section included an IMI questionnaire to assess participants' intrinsic motivation in self-monitoring of diet/nutrition.The fourth section included an IMI questionnaire to assess participants' intrinsic motivation in self-monitoring of physical activity.
\end{itemize}
\end{enumerate}

I developed the IMI questionnaires based on procedures specified by a website maintained by authors of ``Self-Determination Theory''\footnote{http://www.selfdeterminationtheory.org/intrinsic-motivation-inventory/} (Richard Ryan and Edward Deci\citep{deci1985intrinsic}). I pretested these questionnaires during the informative evaluation of prototype II in chapter \ref{prototype2chapter}. The most important sub-scales for the theoretical construct of this research were perceived competence and perceived autonomy which are part of the three basic psychological needs. The relatedness sub-scale is not yet validated but it was included in all questionnaires. Other sub-scales that were included all questionnaires or some of the questionnaires were perceived enjoyment, and perceived useful. Perceived enjoyment is the only direct measure of intrinsic motivation while perceived competence and perceived autonomy are predictors of intrinsic motivation. Self-Determination theory suggests that a behaviour can be started as externally motivated and if external motivators support the three basic psychological needs which are relatedness, competitiveness, and autonomy then a behaviour that was once externally motivated to people can be internalized and the same people will start to perform an activity just because the find it resonating with their core values and beliefs. Perceived useful is a predictor of internalization.

In addition to the aforementioned sub-scales, perceived efforts also appears in specific questionnaires (i.e self-monitoring of diet and activity, use of cellphone). This additional sub-scale was included as part of the package of IMI inventory questionnaire as it may be directly linked to the important sub-scales. However, its results were of less interest to the theoretical constructs of this research.
  
The overall IMI scores were computed by averaging the scores from each sub-scales. In each question from the IMI sub scales, respondents were supposed to rate there experience in a scale of 1 to 7 points which means that 1 implies the statement is "not true at all" and 7 means the statement is "very true".

There were two main objectives of using the IMI questionnaire. The first objective was to assess the ability of the two prototypes in supporting the participants with the three basic psychological needs. The difference in experimental durations was expected not to have any effect on motivations to use either of the two systems since both logbook and gamification were both present in both phases of experiments. Therefore, effects on motivations due to different durations were expected to cancel each other during analysis. I compared between the capability of the two prototypes in affording three basic psychological needs suggested by self-determination theory. In addition, I also included perceptions on enjoyment as it is a direct measure of intrinsic motivation. The corresponding scales from the IMI questionnaire were administered at midline and endline . Therefore, there were four main sub-scales that were considered and these were; perceived autonomy, perceived competence, perceived enjoyment, and perceived relatedness. There were also one additional supporting sub scale which was perceived useful of which its purpose was to extract pattern on internalization as far as gamification features are concerned.

The second objective of using IMI questionnaires was to assess motivations/self-determinations of beneficiaries in self-monitoring of diet and activity, and motivation/self-determination to use cellphone of both intermediaries and beneficiaries. These IMI questionnaires included perceptions of beneficiaries on enjoyment, competence, autonomy, relatedness, enjoyment, effort, and usefulness.

The hypotheses of interest to both intermediaries and beneficiaries on intrinsic motivation's sub-scales related to usage of the app in different experimental conditions were:

\begin{enumerate}
 \setcounter{enumi}{1}
\item{Hypothesis 2}
\begin{itemize}
\item{H\SB{0}}:There is no difference in scores of perceived competence in using the app between a logbook app and gamified app
\item{H\SB{A}}:There is a difference in scores of perceived competence in using the app between a logbook app and gamified app.
\end{itemize}
\item{Hypothesis 3}
\begin{itemize}
\item{H\SB{0}}:There is no difference in scores perceived autonomy in using the app between a logbook app and gamified app
\item{H\SB{A}}:There is a difference in scores of perceived autonomy in using the app between a logbook app and gamified app
\end{itemize}
\item{Hypothesis 4}
\begin{itemize}
\item{H\SB{0}}:There is no difference in scores of perceived relatedness in using the app between a logbook app and gamified app
\item{H\SB{A}}:There is a difference in scores of perceived relatedness in using the app between a logbook app and gamified app
\end{itemize}
\end{enumerate}

Each of the four aforementioned hypotheses was tested twice. The first test was to intermediary users' scores and the second on beneficiary users' scores. There were also hypotheses of interest for beneficiaries in self-monitoring of behaviours reported below:

\begin{enumerate}
 \setcounter{enumi}{4}
\item{Hypothesis 5}
\begin{itemize}
\item{H\SB{0}}:There is no difference in the overall self-determination to self-monitor diet of between a logbook app and gamified app
\item{H\SB{A}}:There is a difference in the overall self-determination to self-monitor diet of between a logbook app and gamified app
\end{itemize}
\item{Hypothesis 6}
\begin{itemize}
\item{H\SB{0}}:There is no difference in the overall self-determination to self-monitor activity of between a logbook app and gamified app
\item{H\SB{A}}:There is a difference in the overall self-determination to self-monitor activity of between a logbook app and gamified app
\end{itemize}
\end{enumerate}

In the comparison for self-monitoring of diet and activity, the first IMI comparison  entailed comparing the IMI score of each participant at baseline, midline, and endline regardless of an experimental condition. In the second comparison I compared scores at baseline, and at both logbook and gamification conditions. The IMI score was computed from the average of scores obtained from perceived competence sub-scale, perceived autonomy sub-scale, perceived relatedness sub-scale, perceived enjoyment sub-scale, perceived effort sub-scale, and perceived usefulness sub-scale. I  used one way ANOVA with repeated measures to test if there was a difference  between scores at: (1) baseline, midline, and endline, and (2)baseline, logbook and gamification. I used Mauchy's test\footnote{Read more on how Mauchy's test is used from http://www.statisticshell.com/docs/repeatedmeasures.pdf} to checked if different measuring points had the same covariance in each ANOVA test I carried out and this helped in deciding of whether to ``Sphericity Assumed'',``Greenhouse-Geisser'', or ``Huynh-Feldt'' of SPSS output.

Before each statistical test including for all of the aforementioned hypotheseses, samples were tested to find if they follow normal distribution ``\emph{Shapiro-Wilk Normality Test}''\footnote{http://sdittami.altervista.org/shapirotest/ShapiroTest.html}) was used to test for normal distribution. For the case of paired samples, the difference between repeated measures of each data point was used to test for normality. In case there was no normality, I would apply a log transformation on the original data, and repeat the normality test again. If normality is achieved I would proceed into using statistical tests that assume normality. For the case of two dependent samples a student t test with repeated measures was used. For the case three dependent samples, a one way ANOVA with repeated measures was used. In all cases of repeated measured, normality was achieved on the original data, as the result only statistical tests that assume a  normal distribution were used.  For the case of of independent samples, each sample was tested for normality. In the absence of a normal distribution in any of the two independent samples, a log transformation was applied in this context. There was a case of two independent samples where a log transformation couldn't result into any normal distribution, therefore, as the result, it was resorted to the use of Mann-Whitney U Test (a non-parametric test for independent samples) as will be reported on the findings. 
 
\subsubsection{Interviews}
I also conducted short unstructured interviews at midline and endline. I selected fewer intermediaries and beneficiaries for the interviews. Interviews responses were important in supplementing data collected through questionnaires and application's logs. All the names used to refer to either which participant excerpts came from or just a particular participant, are pseudonyms to protect confidentiality of participants. Some of the participants quotes that are presented in the findings have also appeared in a publication~\citep{katule2016family} of which I was the first author. 
\section{Findings}
There were four primary outcomes in analysing the findings and these are: (1)usage trend of the app; (2) user experience/intrinsic motivation  of both intermediaries and intermediaries in using the app; (3) intrinsic motivation of beneficiaries in self-monitoring of diet/nutrition; and (4) intrinsic motivation of beneficiaries in self-monitoring of physical activity. Some of these findings are also reported in paper by~\citep{katule2016family}. On reporting age of participants on interview's excerpts, the notation \emph{yrs} refers to years. 
\subsection{Findings on Application's Logs}
\label{usageoutcome}
\begin{figure}[htbp]
  \centering
    \includegraphics[width=0.6\textwidth]{Figures/scatter_daily_sessions.png}
    \rule{35em}{0.5pt}
  \caption{Total daily number of sessions from the two experimental conditions.}
  \label{figure:usagedailysessions}
\end{figure}
The average number of days on which pairs used both versions of the application was 10.5 (S.D = 7.39) days. The most active usage was from a pair that utilized the app for a total of 26 days. The less active usage was from a pair that had used the app for only two days out of 41 days. Figure \ref{figure:usagedailysessions} demonstrates trends on total daily usage's sessions in between logbook and gamification conditions. These trends indicate that in most days a gamified system had more total number of sessions compared to logbook. This is supported by a statistical comparison of daily total number sessions accumulated from all users in each experimental condition which showed that gamification condition had a significant total number of daily sessions compared to logbook as demonstrated by Mann-Whitney U Test on Table \ref{table:usagedays}.
\begin{table}[h!]
  \begin{center}
    \caption{Daily usage comparison between Logbook and Gamified systems for 41 days}
    \label{table:usagedays}
	\begin{tabular}{|L{3cm}|c|c|c|c|c|c|}
		\hline
		Groups&N (sample size)&Mean&Sum Ranks&U&Z&P\\
		\hline
   		Daily logbook sessions&41&33.72&1701.5&\multirow{2}{*}{1159.5}&\multirow{2}{*}{-2.9538}& \multirow{2}{*}{0.00318}\\\cline{1-4} 
   		 		    Daily gamification sessions&41&49.28& 1701.5&&&\\
\hline
	\end{tabular}
  \end{center}
\end{table}

\begin{figure}[htbp]
  \centering
    \includegraphics[width=0.6\textwidth]{Figures/usagedailysessions_lg_gl.png}
    \rule{35em}{0.5pt}
  \caption{Total daily number of sessions from the two experimental conditions.}
  \label{figure:usagedailysessions_lg_gl}
\end{figure}

An interesting phenomenon that expands on the usage pattern of Figure \ref{figure:usagedailysessions} above is that one of Figure \ref{figure:usagedailysessions_lg_gl} which shows usage trends of LG (logbook-gamification) and GL (gamification-logbook) groups. It can be observed that there is a sudden drop in number of usage sessions for users/pairs in GL group after being switched from gamification app to logbook app. This pattern suggests that most behaviour regulation during gamification condition was as the result of ego-involved (introjected regulation). This hypothesized situation is supported by a further exploration on number impressions on gamification features. On checking the average impressions among the four gamification, leaderboard seems to be having the highest average number of impressions as shown on Figure \ref{figure:gamification_impressions_latest_all}. 
\begin{figure}[htbp]
  \centering
    \includegraphics[width=0.6\textwidth]{Figures/gamification_impressions_latest_all.png}
    \rule{35em}{0.5pt}
  \caption{Total daily impressions for two groups (LG and GL).}
  \label{figure:gamification_impressions_latest_all}
\end{figure}
I conducted statistical comparison on the number of impressions between leaderboard and each of other gamification features (score badges, botanical garden, and fish tank). It was found that the number of impressions was statistically significant higher in leaderboard (Mean = 9.93;S.D = 13.90) compared to score badges (Mean =4.71, S.D = 5.50) (t (13) =2.1747, p = 0.0487). There was no statistical significance on either number of impressions between leaderboard (Mean = 9.93, S.D = 13.90) and botanical garden (Mean = 4.79, S.D = 5.26) (t(13)=1.716, p = 0.1096) or number of impressions between leaderboard (Mean = 9.93, S.D = 13.90) and fish tank (Mean = 5.86, S.D = 7.40) (t(13)=1.5707, p = 0.1403).However, the trend shows dominance of leaderboard over other gamification features. 

A further exploration on  gamification features' impressions revealed some insights on the trend of promotion of, ego involved, and task mastery, climates. This exploration follows the discussion of the previous chapter (Chapter \ref{prototype2chapter}) of where it was observed that some statements from qualitative feedback reflected that gamification features tended to promote either task mastery climate or ego-involved climate. In that previous discussion there was an inclination of the leaderboard to promote the ego-involved climate on some intermediary users while features like fish tank or botanical garden were inclined to promote the task mastery climate. Literature suggests that promotion of task mastery climate may foster integrated internalization of behaviour regulation while promotion of ego-involved climate may only promote introjected internalization of behaviour regulation~\citep{saksono2015spaceship}.

Since perceived usefulness is the only predictor of internalization as reported on the methods section above, therefore, one of the logical steps was to compare perceived usefulness scores between intermediary users who had visited leaderboard more often than other gamification features (number of impressions in leaderboard is greater than the number of impressions on any of the other gamification features) and those intermediary users whose number of impressions on leaderboard was almost similar or less to the number of impressions on any of the other gamification features. Since the average number of impressions on botanical garden, score badges, and fish tank were not so distant from each other as indicated on Figure \ref{figure:gamification_impressions_latest_all}, I selected only the botanical garden as a point reference (to represent features that may promote task master climate) for comparison with leaderboard(to represent those features that may promote ego-involved climate). Perceived usefulness was about how intermediaries rated the usefulness of the app to them and their parents. An assumption is that if the regulation is introjected then the intermediaries would careless about usefulness of the app. Therefore, the next task was to compare perceived usefulness between those intermediary users with high leaderboard impressions relative to the botanical garden and those with low leaderboard impressions relative to the botanical garden. In order to know whether an intermediary user falls under high or low group, the ratio was computed and this ratio was the number of impressions on leaderboard divide by the number of impressions on botanical garden. Two group were formed based on the median of ratios from all fourteen intermediary user which was 0.91. Seven intermediary users were assigned into a group with ratio~\textgreater=~\textbf{Median} while the remaining seven intermediary users where put under a group with ratio~\textless~\textbf{Median}. Therefore, the comparison was done on perceived usefulness of the gamified app between intermediary users with impressions' ratio~\textgreater=~\textbf{Median} and intermediary users with impressions' ratio~\textless~\textbf{Median}. The two independent samples followed a normal distribution; hence the results  of a student t test are shown on Table \ref{table:pu_leaderboard_bias}) which indicates that the group with ratio~\textless~\textbf{Median} scored significantly higher in perceived usefulness that the group with ratio~\textgreater=~\textbf{Median}. The implication of this finding is that those that had never accessed the leaderboard or had accessed it fewer times relative to the botanical garden had a higher tendency of valuing the intervention that utilized the gamified app as useful compared to those that had used the leaderboard relatively higher than the botanical garden. 

\begin{table}[h!]
  \begin{center}
    \caption{Comparison of perceived usefulness between group with ratio~\textgreater=~\textbf{Median} and group with ratio~\textless~\textbf{Median} (ratio = impressions on leaderboard/impressions on botanical garden)}
    \label{table:pu_leaderboard_bias}
	\begin{tabular}{|c|c|c|}
		\hline
		Mean &Group with ratio~\textgreater=~\textbf{Median}&Group with ratio~\textless~\textbf{Median}\\
		\hline
		 \multirow{2}{*}{Perceived usefulness}&Mean = 4.143 (S.D = 0.763)&Mean = 5.171 (S.D = 0.962)\\\cline{2-3} 

		 &\multicolumn{2}{|l|}{t(12) = 2.2156, p = 0.0468, 95\% CI = -2.040  to  -0.017} \\
\hline
	\end{tabular}
  \end{center}
\end{table}

Since the bias on usage of leaderboard showed the significant decrease of scores on perceived usefulness, the next task was to see if that affected gamification condition in general. Let's examine the trend of intermediaries' perceived usefulness between midpoint and endpoint between for both a group with ratio~\textgreater=~\textbf{Median} and ratio~\textless~\textbf{Median}. An earlier assumption was that the relationship between parents and children would make children to value the app as meaningful; hence the regulation would be considered as either identified or integrated regulation. But from the findings above, the trend indicated that most users concentrated on the leaderboard and as the result those with higher number of impressions of leaderboard compared to number of impressions on other gamification features such as the botanical garden appeared to have low scores in perceived usefulness. In order to verify the significance of the aforementioned trend,  the hypothesis of interest was that gamification affected the perceived usefulness of the app through its leaderboard feature. In order to prove this hypothesis, the logical step was to assess perceived usefulness between midpoint and endpoint for the two aforementioned groups. The perceived usefulness was statistically significant higher at endpoint (Mean = 5.4, S.D = 1.058) compared to midpoint (Mean = 4.67, S.D = 1.37) (t(6) = 4.9670, p = 0.0025) for the group with ratio~\textless~\textbf{Median} while for the group with ratio~\textgreater=~\textbf{Median}, there was no statistical significance difference between endpoint (Mean = 4.34, S.D = 1.081) and midpoint (Mean = 4.66, S.D =0.781) (t(6) = 0.8742, p =  0.4156). An intermediary user with the highest ratio, was the one with the lowest scores on perceived usefulness. There was a negative correlation between ratio of impressions (leaderboard/botanical garden) and perceived usefulness without statistical significance (r = 0.52, p = 0.06, N=14). What can be concluded so far is that it is very likely the leaderboard played a role in hindering internalization since perceived usefulness is a good predictor of internalization.  Also high usage of leaderboard suggests most usage in gamification was accounted as introjected regulation of where there is ego-involved. The presence of introjected regulation is supported by the excerpt below from one participant but it also emerges in most of the excerpts of users while in gamification condition. Therefore, usage in such cases is mostly influenced by the desire to dominate others in the competition.

\userquote{\textbf{Ayesha}, a beneficiary working with her son, 35 yrs} {``We [with Keagan] were not talking to others because all we wanted was to win. We didn't want them to know but they could see from the app''}

A different dimension of usage was the one that compares if there was a significance difference on number sessions between two experimental conditions (repeated measures of the same pairs at logbook and gamification conditions). As highlighted in the section that describes the methods above, the four pairs with technical glitches were excluded in order to bring fairness in the comparison of the two experimental conditions (Table \ref{table:usageproblems}). It was showed that the mean of logarithmic transformed data of normalized number of sessions between gamification and logbook were significant different (t(9)= -2.6593, p= 0.0261)~\citep{katule2016family}. This implies that the number of times the app was used per day during the gamified condition was significant higher compared to when pairs were in logbook condition. The log mean had to be used in this comparison because the test for normality on differences of data points between logbook and gamification failed. Therefore, a natural logarithmic transformation rectified the situation.

On the next sub sections, user experiences of both intermediaries and beneficiaries are reported.
\subsection{Intermediaries' User Experience}
In most cases, usage of the app within a pair of users was facilitated by intermediary users in proximate enabling and proximate translation. These types of intermediated interactions have been discussed in the work by \cite{sambasivan2010}. Baseline data indicated that interest of intermediary participants in using cellphones was higher than that beneficiary participants. For instance, in overall IMI scores to use cellphone, intermediaries (Mean = 5.76, S.D= 0.41, N=14) scored  significantly higher than beneficiary participants (Mean = 5.06, S.D= 0.71, N=13) with (t(25)= 3.1764, p = 0.0039, 95\% CI = 0.2472 to 1.1589).

If we refer back to the trend on Figure \ref{figure:usagedailysessions_lg_gl}, at the first days of experiments both gamification and logbook have patterns that are similar. However, after the second week logbook starts to go down on usage while the trend on gamification remained steady for a couple of days. One of the possible explanation of why logbook had the same effect as gamification at the beginning is because both experimental conditions had the novelty effect which was worn out after few days. Also the phone effect had contributed to high usage to most users in logbook condition during the phase 1 of experiments. The phenomenon of sharing intervention's phones was important in nurturing the relationship between intermediaries and beneficiaries. In cases where parents shared the intervention's phone with their respective children, children had a tendency to be more interested in the intervention. Children were using those phones for playing games and visiting online social networks. Therefore, free access to the intervention's phone played a role in increasing engagement of intermediary participants who didn't have their own smart-phones or data bundles in their smart-phones. In some cases, intermediary users installed other app on those phones such as games. To understand how the phone had a strong effect on motivation, the relationship between perceived autonomy to use cellphone at baseline and perceived enjoyment to use the app at midline was explored for the group that started with logbook. An interesting phenomenon was revealed. There was a significant negative relationship between perceived autonomy to use cellphone at baseline and perceived enjoyment to use the app at midline for the LG group (r=-0.84, p = 0.017, N=7). An explanation behind this is that those intermediary users with low autonomy to use cellphone had free access to the intervention  phone and this increased their interest to participate in the intervention. For instance one intermediary user (\textbf{Siphosethu}) from the LG group who was the youngest (12 years of age) among all reported the lowest score in perceived autonomy to use cellphone at baseline while reported the highest score in  perceived enjoyment to use the app at midpoint (after using logbook condition). She reported that she had freedom to use the intervention's phone

There was an inclination that a phone coupled with the novelty effect had mediated engagement for intermediaries that had started with the logbook condition. These intermediaries were helping their parents in return they got to have a free pass to access a cellphone. However, this finding is inconclusive due to the sample size. However, it highlights the discussion on the direction of what should be explored in future studies. For the GL group it was difficult to isolate the gamification effect from the phone and novelty affect and studies have shown that gamification has novelty effect, however, the trend shows that gamification accumulated more sessions compared to logbook hence gamification alone increased utilization of the family wellness app. Therefore, gamification appeared to be the most dominant factor that influenced usage as we have already seen that the frequency of usage showed a higher value in gamification when compared to logbook.

In additional to the novelty and phone effects, and gamification also another factor that contributed to intermediaries interaction with the family wellness app was requests from their respective beneficiaries.
There were times where intermediary users opened the app for interaction only upon receiving requests from beneficiaries. In both absence and presence of gamification, intermediaries had to fulfil requests from beneficiaries.  But during logbook condition, intermediaries appeared to be less enthusiastic in handling those requests. Some beneficiaries complained that there were several incidences of where their respective intermediary users were refusing to fulfil these requests and it happened more often during logbook condition. It was observed that in most of these cases, intermediary participants' autonomy was violated as requests came at times where intermediary participants were either studying for exams or doing something else and they felt it was not the right time to fulfil those requests. As the result, some of the intermediary participants to felt of being nagged by their parents. In one instance, a male intermediary participant appeared to be irritated by constant requests by his mother to interact with the app especially at the times when he is with he phone doing something else. Also a similar scenario was shared by one female intermediary participant who mentioned that her mother's constant demands to engage with the app were sometimes annoying as there was no excitement for her to the app as shown on the below excerpt.

\userquote{\textbf{Jennifer}, an intermediary for her mother, 18 yrs} {``The app was okay first but it started to get boring. You don't want to go into it any more. I think there will be some excitement now if the game comes in. When do we get the game''}

There were two motivational drivers for all requests that were put forward by beneficiary users. The first source of motivation was the instrumental value derived from using the app. The second reason is gamification. Requests that were mediated by gamification fostered a sense of collaboration between members of a pair while in the scenarios where beneficiaries were not concerned about gamification or during logbook conditions, requests tended to be of authoritative nature.

\userquote{\textbf{Ayesha}, a beneficiary working with her son , 35 yrs}{``I would always ask him [Keagan] where are we. Are we first? And what badge do we have? Where are the others? How far is Simon [intermediary] then? How far is that one? `No mum we are on top. We are first. We are the champions' [during gamification].''} 

The aforementioned excerpt from \textbf{Ayesha} demonstrates a sense of collaboration between the two members of a participating pair (\emph{Ayesha} and \emph{Keegan}). Therefore the presence of gamification in this context tended to promote a collective responsibility for pairs where both members of a pair were motivated by gamification unlike in logbook condition where there was an absence of that cooperation. The excerpt below came form a beneficiary participant in logbook condition and it indicated a beneficiary participant as having authority of what should be done and not a cooperation between members of a participating pair. 

\userquote{\textbf{Sisipho}, a beneficiary working with her son, 43 yrs} {``I always start the conversation. Because I always want to make sure if he records because I can't use it. It was difficult for me to use it. [during logbook]''}

The ``Gamified App'' was designed in such a way that a pair will earn rewards based on usage and the average number of steps walked by a beneficiary participant who is a member of the pair. The purpose of rewards was to foster users' intrinsic experiences such as competitiveness and a sense of autonomy which are predictors of intrinsic motivation with the goal of improving collaboration between members of a pair. Rewards depended on four parameters and these were the number of steps walked by a beneficiary user, the number of days the app has been used by an intermediary to either to record meals or to view feedback on meals, points, steps, gardens, etc. The presence of gamification nurtured  team work  even though the regulation of self-monitoring through intermediary users was mostly introjected (regulation done for the purpose of becoming better than others) as shown on below excerpts.

\userquote{\textbf{Keagan}, an intermediary for his mother, 16 yrs} {``When I see other people trying to come above me [on the leader board]. I hand over the phone to my mom so she can walk more steps.''}

\userquote{\textbf{Christine}, an intermediary for her mother, 16 yrs} {``I told my mom that me myself I want our team to have the highest points. Yes she said she is going to do that.''}

Through these collaborations goals were set of which in most cases they were set by intermediary users and informed their respective beneficiary about what they wish for the team (pair). A sense of joint ownership of the process was also common for these intermediary users as indicated by the following excerpts. 

\userquote{\textbf{Sophia}, an intermediary for her mother, 17 yrs} {``Sometimes that person may be first so I tell my mom that we must also be at the first place.[She looks at the  leader board and she sees other people at the first place, therefore, she talks to her mother that they should also aim for the first position] ''} 

\userquote{\textbf{Jenner}, a beneficiary working with her son, 45 yrs} {``When he [Leon] looked through it [The app] and sees their points, he would say `Mom, we need to do something here, because look at their points and our points'. So it was quite interesting.''} 

Comparison on virtual rewards among intermediary users motivated them to check the app more often compared to when they were in logbook condition as highlighted on the aforementioned usage findings (sub section \ref{usageoutcome}). Intermediaries were competing which each others through the leaderboard. As the result of this competition there were frequent face to face interactions that entailed comparing each other since most of them either attended the same schools or lived not far from each other. 

\userquote{\textbf{Jenner}, a beneficiary working with her son, 45 yrs} {``He [Leon (her 15 years old son) ] likes this exercise (using the app) because among him and his friends, they would have that competition like `I got more points than you' and that motivated him to get interested with the app''} 

Aforementioned cases above prove that regulation was mostly introjected as it was mediated by the need to be better than others in terms of points or being at the top of the leaderboard. Other features were not discussed in details in the interviews as intermediary users or beneficiary users who were interviewed didn't give any insights. The desire to be better mediated cheating the rules in some cases. For instance, there was a scenario of one pair of which not only the beneficiary user was using the pedometer, an intermediary was also taking turns to use the pedometer, therefore they were collaborating in accumulating steps. Both an intermediary user and a beneficiary user had discussions of whether the person whose turn it was had walk enough steps. They did this to accumulate more steps and hence more points than other pairs of users. 

\userquote{\textbf{Christine}, an intermediary for her mother, 16 yrs} {``I ask her how far did you walk?  She would say she walked very far. She tells me that I must have the phone to walk more steps. She would say `I got more more walking than you' [They were collaborating with her mother in accumulating steps]. She sometimes writes the steps on the page and she tells me yesterday I day I had more points than you [points referring to steps ]''} 

The ambiance of competition with others appeared to also affected pairs that had started with logbook conditions as some intermediary users were pushing their beneficiaries to do more by expecting to get rewards once they are switched to gamification condition; as the result during logbook condition there were intermediary participants who manifested ambitions to win while discussing with their respective beneficiaries~\citep{katule2016family}. The presence of gamification was also inclined to strengthen family bond or relatedness between members of a participating pair.

\userquote{\textbf{Khanyiswa}, a beneficiary working with her daughter, 26 years old} {``I think we talk more (with \textbf{Siphosethu}) than before the family wellness app. Before the family wellness app, after work it was just ``Hi'' and then I go to my room but now. But now she would come to my room  and say let me see your phone,  what did you eat today,  and write everything down on the phone.''} 

On the aspect of relatedness among intermediary users, social features were never utilized except by two users. One of those two users appeared to attempt to make a social connection with other users through the app. It was observed that this particular user was not collocated with other users; hence social features on the app was the only way to feel connected with others.  

\userquote{\textbf{Siphosethu}, an intermediary for her mother, 12 yrs } {``Wow it shows that you are working hard  Clara\#2.[She congratulated [Clara -a female aged 14 years old for having their fish tank ranked number 2 in quality.]''} 

The findings in user experience were mixed with some intermediary users having positive user experience on utilizing gamification while perceived enjoyment on some intermediary users was lower in gamification compared to logbook despite the fact that there were more sessions reported in gamification compared to logbook. Since the leaderboard drew most attention, it could have played a role in demotivating users with lower performance. Leader-board can demotivate those users that are at the bottom but it can foster aspects of relatedness for all users~\citep{sailer2013:psychological}. One way to exclude the impact of this on motivation was to single out intermediary users that never had any advancement in badges but had accessed to the leaderboard for at least two days or those intermediary users that never accessed any gamification feature due the app failing to load but seemed to be interested in accessing gamification features. I used badges as the point of measuring progress for those that had used the leaderboard for at least two days in a row or non consecutively. In this category there were two intermediary users that reported to have higher utilization of gamification features including the leaderboard and never made any progress in badges. These were among users that belonged to teams/pairs that were at the bottom of the leaderboard and their perceived enjoyment was lower in gamification condition compared to logbook condition.  One user was from the `LG' group (Keller), while the other one was from the `GL' group (Leon).  Their beneficiary participants were not walking enough steps despite the fact that these two intermediaries had put  more efforts in using the app during gamification condition than in logbook condition as shown on Figure \ref{figure:badge_failure_2}. 
\begin{figure}[htbp]
  \centering
    \includegraphics[width=0.7\textwidth]{Figures/badgesfailures2.png}
    \rule{35em}{0.5pt}
  \caption{Usage of two users without technical problems but lacked progress in badges.}
  \label{figure:badge_failure_2}
\end{figure} 

In the category of users with technical problems, there are two intermediary users of which one had attempted to access  the leaderboard and the pedometer was not working, and in the other user the app was failing to load completely hence couldn't access any feature. The two users in the latter category are part of the four users reported to be having usage problems as reported on Table \ref{table:usageproblems}. The remaining two users from Table \ref{table:usageproblems} never accessed a leaderboard although they had accessed other gamification features; hence were not so much exposed to the peer comparison as the result of using the leader-board although they had visited other gamification features. The perceived enjoyment of these two users who had accessed all gamification features except for the leaderboard was higher in gamification condition compared to logbook condition.    

There was also a case of where one user was a bit pessimistic about the credibility of the gamified system because her peers appeared to have more interest with the app, but were at lower positions on the leader board compared to her. 

\userquote{\textbf{Anathi}, an intermediary for her mother, 16 yrs} {``It [The experience in using the gamified app] was the same as last time (during logbook condition) except for the game part. I was actually above some of the others. That was weird. Because they were more interested in the app than me.[She was making a reference to intermediary users in pairs C and D that were found to experience technical problems as reported on Table \ref{table:usageproblems}]''} 

Anathi's anticipation was to see her peer being more competent than her once switched to gamification but contrary to her expectations she was ahead of them.  Therefore, this made her to doubt her competence. This further proved by her reported score on perceived competence between logbook and gamification. Anathi reported lower score in gamification compared to logbook despite using gamification for 7 out of 14 days and logbook for only 4 out of 27 days. However, Anathi had reported a slightly higher perceived enjoyment score  during gamification when compared to during logbook.

Therefore, in comparison between logbook and gamification for the support of the three basic psychological needs, a decision was made to exclude the four pairs whose perceived enjoyment was largely affected by either technical glitches or inability to participate fairly in gamification as the result of the app not being able to tailor challenges with skills of intermediary users as their performance relied to the great extent to the performance of their respective beneficiary users. For the intermediary users in the latter case, an assumption was that a negative experience was the result of failure of the gamification design to match challenges with abilities. i.e. efforts of beneficiary differed differed hence their performance had an effect on the overall performance of their respective team. This implies in that context, intermediary users had little control of the performance of their beneficiaries. Some intermediaries reacted in a negative way when they thought their respective intermediaries were not complying to carry the pedometer all the time.

\userquote{\textbf{Jenner}, a beneficiary working with her son, 45 yrs} {``Sometimes may be I forget to take the phone when I go walking and he would ask me `did you take the phone with you' Ooh Gosh I forgot.  Because when I walk to Park Town to exercise and sometimes  I am in such a hurry I forget the phone, he will be crossed with me.''} 

The aforementioned excerpt demonstrates how an intermediary user was attempting to control his beneficiary user as result competition influenced by peer pressure from intermediary users in other pairs. Therefore instead of gamification being something enjoyable in this context it resulted into a negative reaction that could threaten an existing social rapport between a parent and a child.

As for reasons stated above, four intermediary users in total were excluded in the analysis of self-determination theory. As the result only 10 out of 14 intermediary users were considered in the sub-scales that measured the three aspects of SDT (autonomy, competence, relatedness). On perceived competence to engage with the app, the gamified condition scored statistically significant higher than the logbook condition (t(9)=3.495, p = 0.0068)~\citep{katule2016family}. There was no significant difference between logbook and gamification on scores of perceived autonomy (t(9)= -0.027, p =  0.98) and perceived relatedness (t(9)= -0.719, p = 0.49)~\citep{katule2016family}.

\subsection{User Experience of Beneficiaries}
As most beneficiaries only interfaced with the app through intermediary users, beneficiaries' user experience relied on cooperation they got from intermediaries. On support for the three basic psychological needs, there was no difference between logbook and gamification. However, aspects of relatedness (N=14) appeared to improve significantly with time when in comparison between midpoint (Mean = 4.43, S.D=0.92) and endpoint (Mean = 5.38, S.D=1.08)(t(13)= 2.3736, p =  0.0337). Therefore, the intervention in general made beneficiaries felt more related to others (their respective intermediaries and other beneficiaries) who were part of the intervention.
  
On utilizing the app through intermediaries, there are cases where beneficiaries had a negative experience as result of intermediaries refusing to assist upon being given requests. This happened in cases of where intermediary users didn't feel like helping because of being occupies by other tasks such as reading for exams or because they felt the app was boring especially in logbook condition. 

In the next sub-section, the IMIs in self-monitoring of diet and activity are reported. Four pairs with usage problems (Table \ref{table:usageproblems}) were excluded due to their usage in self-monitoring being affected. Therefore, in total only ten out of fourteen beneficiaries had their results included for analysis in order to have only beneficiaries who had meaningful engagement with the app through their respective intermediaries.
\subsubsection{IMI in Self-Monitoring of Diet}
The results on self-monitoring of diet (baseline, midline, and endline) are shown on Table  \ref{table:imidietbenf}. The Mauchly’s test indicated that the assumption of sphericity was not violated with  $\chi{}$\SP{2}(2)=3.76, p = 0.152. The results (N=10) on  ``Self-monitoring of Diet'' shown on Table \ref{table:imidietbenf} were from ``Sphericity Assumed'' output. ANOVA showed that there was a significant difference of average IMI scores on self-monitoring of diet measured at baseline, midline and endline.
\begin{table}[h!]
  \begin{center}
    \caption{Comparison of ten beneficiaries' IMI scores in self-monitoring of diet at baseline, midline and endline}
    \label{table:imidietbenf}
	\begin{tabular}{|L{2.8cm}|L{3.2cm}|L{3.2cm}|L{3.2cm}|}
		\hline
		Mean IMI Score &Baseline&Midline&Endline\\
		\hline
		 %\multirow{3}{*}
		 {Self-monitoring}&Mean = 4.48; S.D = 1.24&Mean = 5.07; S.D = 1.19;&Mean = 5.55; S.D = 0.95\\\cline{2-4} 

		of Diet &\multicolumn{3}{|l|}{F(2,18)=3.787; p = 0.042} \\
\hline	\end{tabular}
  \end{center}
\end{table}
A finding from a pairwise comparisons (a paired student t-test) indicated that the IMI score at endline was significantly higher than at baseline (Table \ref{table:imipairwisediet}). There was no significant difference on baseline versus midline and midline versus endline (Tables \ref{table:imipairwisediet1}, and \ref{table:imipairwisediet2}). Motivation to self-monitor diet appeared to increase with time as shown on Figure \ref{figure:imi_diet}. The interpretation of the above findings are that the wellness app appeared to had a significant effect of time on motivation of beneficiaries to self-monitor their diet.
\begin{table}[h!]
  \begin{center}
    \caption{Pairwise comparisons of IMI scores in self-monitoring of diet: Baseline versus Midline}
    \label{table:imipairwisediet}
	\begin{tabular}{|L{2cm}|L{4cm}|L{4cm}|}
		\hline
		Mean &Baseline&Midline\\
		\hline
		 \multirow{2}{*}{IMI Score}&Mean = 4.48; S.D = 1.24&Mean = 5.07; S.D = 1.19\\\cline{2-3} 

		 &\multicolumn{2}{|l|}{t(9) = 1.298, p = 0.227} \\
\hline
	\end{tabular}
  \end{center}
\end{table}
\begin{table}[h!]
  \begin{center}
    \caption{Pairwise comparisons of IMI scores in self-monitoring of diet: Baseline versus Endline}
    \label{table:imipairwisediet1}
	\begin{tabular}{|L{2cm}|L{4cm}|L{4cm}|}
		\hline
		Mean &Baseline&Endline\\
		\hline
		 \multirow{2}{*}{IMI Score}&Mean = 4.48; S.D = 1.24&Mean = 5.55; S.D = 0.95\\\cline{2-3} 

		 &\multicolumn{2}{|l|}{t(9)= 2.457, p = 0.036} \\
\hline
	\end{tabular}
  \end{center}
\end{table}
\begin{table}[h!]
  \begin{center}
    \caption{Pairwise comparisons of IMI scores in self-monitoring of diet: Midline versus Endline}
    \label{table:imipairwisediet2}
	\begin{tabular}{|L{2cm}|L{4cm}|L{4cm}|}
		\hline
		Mean &Midline&Endline\\
		\hline
		 \multirow{2}{*}{IMI Score}&Mean = 5.07; S.D = 1.19&Mean = 5.55; S.D = 0.95\\\cline{2-3} 

		 &\multicolumn{2}{|l|}{t(9)=-1.975; p = 0.08 ; 95\% CI= -1.0342 to 0.07017} \\
\hline
	\end{tabular}
  \end{center}
\end{table}

\begin{figure}[htbp]
  \centering
    \includegraphics[width=0.4\textwidth]{Figures/imi_diet.png}
    \rule{35em}{0.5pt}
  \caption{Trend on Average IMI Scores of Self-Monitoring of Diet at Baseline, Midline, and Endline.}
  \label{figure:imi_diet}
\end{figure}
The aforementioned ANOVA finding on comparison among baseline, midline, and endline doesn't discern between different experimental conditions of which pairs of users were exposed to. The ANOVA finding (N=10)(Table  \ref{table:imidietbenf2}) on the comparison of IMI scores to self-monitor diet, among baseline, logbook, and gamification conditions showed that there was no significant difference of average IMI scores on self-monitoring of diet measured during baseline, logbook and gamification conditions. This finding is from the ``Sphericity Assumed'' output of the ANOVA test since the Mauchly’s test indicated that the assumption of sphericity was not violated with  $\chi{}$\SP{2}(2)=2.19, p = 0.335. The trend on averages shows both logbook and gamification to be slightly higher than baseline as shown on Figure \ref{figure:imi_diet2}. The conclusion from this finding is that both versions of the prototype have shown an indication of increasing motivation of beneficiaries to self-monitor diet.
\begin{table}[h!]
  \begin{center}
    \caption{Comparison of ten beneficiaries' IMI scores in self-monitoring of diet at baseline, after logbook, and  after gamification conditions}
    \label{table:imidietbenf2}
	\begin{tabular}{|L{2.8cm}|L{2.5cm}|L{2.5cm}|L{2.5cm}|}
		\hline
		Mean IMI Score &Baseline&Logbook&Gamification\\
		\hline
		 %\multirow{3}{*}
		 Self-monitoring&Mean = 4.48; S.D = 1.241&Mean = 5.28; S.D = 1.05&Mean = 5.34; S.D = 1.16\\\cline{2-4} 
		 of Diet&\multicolumn{3}{|l|}{F(2,18)=3.787; p = 0.087} \\
\hline	\end{tabular}
  \end{center}
\end{table}
\begin{figure}[htbp]
  \centering
    \includegraphics[width=0.4\textwidth]{Figures/imi_diet2.png}
    \rule{35em}{0.5pt}
  \caption{Trend on Average IMI Scores of Self-Monitoring of Diet at Baseline, Logbook, and Gamification.}
  \label{figure:imi_diet2}
\end{figure}
\subsubsection{IMI in Self-Monitoring of Activity}
The results (N=9) on self-monitoring of activity are shown on Table  \ref{table:imiactivitybenf}. The results are based on a sample of nine beneficiary users as one participant didn't complete this part of the questionnaire at baseline.  The Mauchly’s test indicated that the assumption of sphericity was violated with  $\chi{}$\SP{2}(2)=8.248, p = 0.016. The value $\epsilon$ on Greenhouse Geisser was ``\textless 0.75'', therefore, the results on  ``Self-monitoring of Diet'' shown on Table \ref{table:imiactivitybenf} were selected from ``Greenhouse-Geisser'' output. ANOVA showed that there was no significant difference of average IMI scores on self-monitoring of activity measured at baseline, midline and endline. The trend of means appears to increase from baseline to endline as shown on Figure \ref{figure:imi_activity}.

There are several factors that could have contributed to results not being significant among baseline, midline,endline points,. The first hypothesized reason is tracking of physical activity appeared to be easy in majority of the participants even without tracking devices as people can estimate the distance they walk daily and they consider this as tracking even though they might have means to record this information, hence their motivation was high at baseline unlike diet self-monitoring which they consider it to be cumbersome due to external barriers such as health food being expensive, therefore at baseline participants felt more motivated to track their activity. The second hypothesized reason is that the sample size was small hence there was a smaller power in detecting significant difference. But we have seen that the trend in motivation increases with time.
\begin{table}[h!]
  \begin{center}
    \caption{Comparison of ten beneficiaries' IMI scores in self-monitoring of activity at baseline, midline and endline}
    \label{table:imiactivitybenf}
	\begin{tabular}{|L{2.8cm}|L{2.5cm}|L{2.5cm}|L{2.5cm}|}
		\hline
		Mean IMI Score &Baseline&Midline&Endline\\
		\hline
		 %\multirow{3}{*}
		 Self-monitoring&Mean = 4.82; S.D = 1.002&Mean = 5.28; S.D = 1.003&Mean = 5.41; S.D = 0.894\\\cline{2-4} 
		 of activity&\multicolumn{3}{|l|}{F(1.182, 9.455)=2.936; p = 0.116} \\
\hline	\end{tabular}
  \end{center}
\end{table}

\begin{figure}[htbp]
  \centering
    \includegraphics[width=0.4\textwidth]{Figures/imi_activity.png}
    \rule{35em}{0.5pt}
  \caption{Trend on Average IMI Scores of Self-Monitoring of Activity at Baseline, Logbook, and Gamification.}
  \label{figure:imi_activity}
\end{figure}
The finding from an analysis (N=9) that examined if there is a difference among baseline,logbook, and gamification in self-monitoring of activity, showed that there was no significant difference of average IMI scores on self-monitoring of activity measured at baseline, logbook and gamification (Table {table:imiactivity2benf}). The Mauchly’s test indicated that the assumption of sphericity was violated with  $\chi{}$\SP{2}(2)=6.788, p =0.034. The value of $\epsilon$ on Greenhouse Geisser was ``\textless 0.75'', therefore, these results on  ``Self-monitoring of Activity'' were selected from ``Greenhouse-Geisser'' output of oen way with repeated measures ANOVA test. The trend in motivation increases in both logbook and gamification compared to baseline as shown on Figure \ref{figure:imi_activity2}
%epsilon=0.617
\begin{table}[h!]
  \begin{center}
    \caption{Comparison of ten beneficiaries' IMI scores in self-monitoring of activity at baseline, logbook and gamification}
    \label{table:imiactivity2benf}
	\begin{tabular}{|L{2.8cm}|L{2.5cm}|L{2.5cm}|L{2.5cm}|}
		\hline
		Mean IMI Score &Baseline&Logbook&Gamification\\
		\hline
		 %\multirow{3}{*}
		 Self-monitoring&Mean = 4.82; S.D = 1.002&Mean = 5.33; S.D = 0.9762&Mean = 5.37; S.D = 0.9276\\\cline{2-4} 
		 of activity&\multicolumn{3}{|l|}{F(1.234, 9.872)=2.783; p = 0.123} \\
\hline	\end{tabular}
  \end{center}
\end{table}
\begin{figure}[htbp]
  \centering
    \includegraphics[width=0.4\textwidth]{Figures/imi_activity2.png}
    \rule{35em}{0.5pt}
  \caption{Trend on Average IMI Scores of Self-Monitoring of Activity at Baseline, Logbook, and Gamification.}
  \label{figure:imi_activity2}
\end{figure}

\section{Discussion}
\subsection{Motivational Affordances' Impact on Intermediaries}
As reported above intermediary users facilitated interaction with the app. Intermediaries were motivated to engage with the app mostly because of two main reasons and these were: either to follow up on competition with others or to serve requests from their respective beneficiary users. Gamification was the main source of motivation to intermediaries and it played a pivotal role in mediating usage during gamification condition or even in logbook condition for intermediary users in pairs that had started with logbook condition before being switched to gamification condition. Expectation of gamification condition at a later stage by some users who had started with logbook condition motivated them to sustain engagement while in logbook before being switched to condition. For instance in a case of \textbf{Siphosethu}, an intermediary user who talked of winning while still in logbook condition. Therefore, this affected the separation of experimental conditions since the randomization to experimental sequences was not blind; hence those that started with logbook already knew that they will start using gamification after a period of three weeks and this positively affected their behaviour towards engagement with the app. On perceived competence, the difference between logbook and gamification was statistically significant meaning intermediary users in gamification felt more competent.Statistical significance in difference was not achieved in cases of autonomy and relatedness due to the learning effect and a smaller sample size.  Perceived competence and perceived autonomy are the most important predictors of intrinsic motivation. 

From the findings it is also evident that gamification played a role in improving intermediaries' perceived enjoyment (a direct measure of intrinsic motivation) in using the app except for fewer cases of where it appeared to harm  perceived enjoyment due to several reasons that have already been highlighted on the findings section above. For instance the use of leaderboard contributed to both positive and negative depending on individual characteristics of an intermediary user and characteristics of a team in general. Intermediary users responded differently to gamification due to a great variation in users' traits (personalities) and skills possessed by their respective team mates (beneficiaries) of where this contributed to having such mixed results. Literature has vastly explored how important it is to tailor game design mechanics to players' personalities. In one survey that investigated how preferences of motivational affordances is linked to individuals' personality traits concluded with some of the following insights: (1) extraverts tend to be motivated by points, levels, and leaderboard; (2) avatars are likely to be preferred by people with high levels imagination; (3) extraverts prefer to be centre of the stage; hence the position they desire in a feature like a leaderboard is the top one; (4) introverts may not be happy to be in a crowd of strangers; (5) and so on..~\citep{jia2016personality}. Personalities that are exhibited by game players may also cross to gamification since the boundary between gamification and games is diminishing~\citep{ferro2013towards}. What may be considered just as a simple gamified system in one context may be socially perceived as a game in a different context~\citep{deterding2011game}. Personalization of healthy interventions to gamer types  is already common\cite{arteaga2010:persuasive,orji2013:tailoring} but has only been used in the context of direct users of technology and not in the context of intermediary and beneficiary users.

Apart from tailoring of game mechanics another aspect of personalization is on involuntary participation in gamification features. This brings to the subject of supporting autonomy. Gamification borrows its game mechanics from games and playing games has been pointed out as voluntary~\citep{seaborn2015:gamification,knaving2013designing}. Participating in a gamification layer should be an opt-in (invisible) and not a mandatory in order not to obscure the main activity being promoted~\citep{knaving2013designing}. In most of current gamification designs the line between voluntary and involuntary participation is not clear as a user may voluntarily participate in gamification but may not involuntarily participate in a feature like a leaderboard which may come as part of the package of gamification~\citep{ferro2013towards}. This brings to the subject of autonomy to users in selecting which of part of gamification features they would like to participate in. In addition to that, there is a different aspect of autonomy that was a shortcoming in this intervention and this is the inability of intermediary users to select the level of gamification appropriate to their skills. The only support for autonomy that was provided is the one that allowed configuration of avatars and editing of team profiles. Intermediaries stated goals as indicated in the findings but never had enough power in accomplishing those goals as the goals largely depended on skills possessed by their respective beneficiaries. Literature suggests approaches on which autonomy could be supported which include but not limited to configuration of profiles, avatars, macros, configurable interface, alternative activities, privacy control, etc. \citep{francisco2012analysis}. These approaches may scale to the context of technology use through intermediaries but there is a need to explore how intermediary users may have autonomy in formulation and execution of goals to tackle challenges based on their skills. When challenges are too difficult as they don't match users' skills, end users can become demotivated \citep{zhang2008motivational}.    

Absence of autonomy in formulation and execution of goals may foster  negative experiences which appeared to harm intrinsic motivation of some intermediary users in this context. For instance in most cases presence of gamification fostered collaboration between intermediary and beneficiary members of a pair. Out of this collaboration intermediaries were attempting to influence or persuade their respective beneficiaries. In such attempts negative experiences emanated when persuasion was not working. An example of a negative experience is the case of where nudging evolved into nagging like in a scenario we have seen above of \textbf{Jenner}, a beneficiary user who described how serious her son was taking the competition with others by constantly reminding her to carry the pedometer whenever she wants to go out and at times the son would get annoyed if his mother forgets to carry the pedometer. The ramification of this is that it deviates from the goal of promoting collaboration between an intermediary user and a beneficiary user and instead it creates a tension between them.  
In such scenarios intermediary users may react out frustration of not having control of the skills possessed by their respective beneficiaries; hence this idea of intermediaries to rely solely on skills of their respective beneficiaries seemed not to resonate with the notion of matching challenges to skills of users and it was the main source of tension between members of a pair as highlighted above. 

One of the approaches that could be used to minimize the effect of the  aforementioned shortcomings is to give users more autonomy to select different levels of gamification they want to participate. There could be levels such as beginners, intermediate, advanced, etc. Pairs that are on the same level could be grouped together and not mixed with pairs with levels that are different. In addition, users could be allowed to select which features they would like to include in their interfaces from a range of features such as chat rooms, leader-boards, botanical gardens etc. More autonomy can also be given in customization of privacy in terms of whether they would like to share their information or not. Customization of avatars is also important because It was observed that most users changed their avatars during gamification and one user explained that she sees the avatar she selected as a representation of herself. Through avatars, these users embodied their identities.

The second possible approach in increasing engagement of intermediary users is to allow intermediary participants to participate with their information, by incorporating their wellness data i.e. steps. The former can also be combined with the latter. There were some observed scenarios that support the utilization of the latter approach. For instance, there was one pair of whereby not only the beneficiary participant was using the pedometer as an intermediary was also using it. They were taking turns to use the pedometer, therefore, they were collaborating in accumulating steps. This pair had discussions of whether the person whose turn it was had walked enough steps. The goal was to accumulate more steps than other pairs. A similar concept has been explored with participants in a low income neighbourhood in USA , of whereby there is an exergame that encourages cooperation between parents and children~\citep{saksono2015spaceship}.

A third proposed way of increasing engagement that could be leveraged is the one reflected by intermediaries who claimed to also be benefiting from nutrition/diet information since the same type of meal is shared at home, therefore, if beneficiary participants ate something that is not healthy while at home then there is a likelihood of an intermediary participant to have eaten the same type of meal too. According to literature, parents who live a healthily lifestyle are likely to also influence their children to live healthily~\citep{grimes2009toward}. It is possible that by creating a system that allows intermediaries to also benefit from usage one can foster regulation that is either type \emph{identified} or type \emph{integrated} which are both on the side of the spectrum nearer to intrinsic motivation. 

Apart from gamification, another important source of motivation of which one could leverage is, sharing of phones between participating members of a pair. Beneficiaries were custodians of intervention's phone. But in many cases when beneficiaries were at home they left the phone with the intermediaries who were interested with social media sites and  games. Intermediaries were interested with those phones because of either of the two reasons or both: (1) Interventions phone's were better than intermediaries' phones or intermediaries didn't have smart phones that can enable to access services they desire; and (2) Availability of data bundles in intervention's phones through inserted SIM cards. In these scenarios, some intermediaries were implicitly reciprocating the favours of having freedom to use the phone by serving requests from their beneficiaries. This kind of non-prescribed use is important and it has been emphasized that it should be viewed as part of a play which is a capability to increase engagement of participants in an ICTD intervention~\citep{ferrplay2015}. Therefore, one can capitalize on this motivation introduced as the result of sharing phones and it can be viewed as part of motivational affordances to encourage ongoing use of a system through young intermediaries within family settings. Utilization of the motivational effect of the phone in mediating such an intervention depends on interest of beneficiaries on the intervention. Without requests from beneficiaries, and with absence gamification on the app for intermediaries, the phone effect itself cannot mediate usage of the app unless it goes in parallel with those two mediating factors for usage.

The last aspect of self-determination worth discussing is perceived relatedness. The trend of intermediaries users of not using social features was recurrent from the previous chapters. There were face to face interactions outside the context usage logs as it can noted from some of the excerpts of findings. These interactions were inseparable between logbook and gamification conditions. As it has been highlighted by~\cite{lin2006:fish}'s study that social features may be appropriate in contexts users are not collocated; hence there is absence of face to face interactions and the only way for users to interact is through social features provided by the app. For instance one intermediary user who was so keen on using social features was not having face to face interactions with the rest because she didn't get a chance to meet them face to face apart from the meeting organized by the researcher. 
\subsection{Motivational Affordances' Impact on Beneficiaries}
As it has been reported on the findings section, beneficiaries engaged with the app through intermediaries upon either intermediaries coming to them or beneficiaries putting a request of something to be done on the app. Requests from beneficiaries were as the result of being interested in either one of both of the following; (1) leader board, and (2) instrumental value provided by the app. Not all beneficiaries were motivated by gamification. Different strategies are required in order to engage older adults. Literature suggests that emotional stability increases with age~\citep{carstensen2011emotional}; this implies that adults may have a tendency of higher emotional stability compared to children. Gamification may be less effective for people with higher emotional stability~\citep{jia2016personality}. In the previous chapter (Chapter \ref{prototype2chapter}) it was observed that adults cared most about social support from other adults and social comparison increased their perceived relatedness. In this evaluation they were less interactions between beneficiaries; hence less social comparison among beneficiaries alone. Strategies that improve relatedness of beneficiaries may be one way to improve engagement of beneficiaries. Also features that promote task mastery climate may be of interest to this group of participants. For instance one intermediary participant reported that her mother was interested with the botanical garden more than her. 

In general the app was perceived well by beneficiaries and gamification was of less importance compared to the perceived value from the app. This is demonstrated by the overall scores of intrinsic motivation inventory (IMI) in self-monitoring of diet and activity. An improvement in IMI score for self-monitoring of diet is significant at endpoint when compared to baseline while for self-monitoring for activity there is an indication of improvement without statistical significance.      
\subsection{Internalization of Helping in Self-Monitoring}
The dominance of leaderboard resulted into introjected regulation in most part of where individuals as individuals don't accept a behaviour as of value or of their own rather they merely perform it for the sake of maintaining their self-worth. This had an impact on internalization. Those intermediary users that had no contact with the leaderboard or minimum contact with respect to other gamification features reported to value the intervention as more useful in gamification compared to in in logbook condition. Therefore it is very clear that leaderboard had a tendency of creating an atmosphere of introjected regulation, and this kind of regulation  to overshadow the main activity (monitoring of health of beneficiary users) being promoted. A challenging task is to design gamification in such a way it doesn't become the main focus instead of an activity being promoted~\citep{knaving2013designing}.

A leaderboard may be appropriate to some users but it may have a negative effect on some users, and this has already been highlighted on the discussion about personalities and gamer types, and perceived autonomy. However, this finding is not at a stage of being conclusive due limitation of the sample size but can be a basis of forming a much larger study with a bigger sample size in order to test the hypothesis about domination of introjected regulation under the presence of the leaderboard.

\subsection{Impact of Cognitive Flow on User Experience}
One of the challenges in the intervention was achieving an optimal flow in the context of intermediated use. The application was installed on one phone and beneficiary users were the ones that had custody of the phone. Therefore, In most cases intermediaries only got access to the phone when intermediaries were within proximity. Achieving timely feedback may pose a challenge in this usage context. For instance in some cases intermediaries had limited access to the app since their respective beneficiaries had gone away with the phone. This had a negative impact on flow of both sets of users as they couldn't self-reflect on time. This brings the discussion of how to optimally maintain flow. Therefore, how users and technology are arranged could have an impact on cognitive flow. Intermediaries need to be able to access to a system even in cases where their respective beneficiary users are around. One important aspect in supporting optimal cognitive flow is to provide feedback on time~\citep{csikszentmihalyiflow}. Therefore, maintenance of flow in the context of intermediated use is crucial for user experience. 
\begin{flushright}
\end{flushright}


% Chapter 1

\chapter{Conclusions and Future Research} % Main chapter title

\label{discussionchapter} % For referencing the chapter elsewhere, use \ref{Chapter1} 

\lhead{Chapter \emph{Conclusions and Future Research}} % This is for the header on each page - perhaps a shortened title

%----------------------------------------------------------------------------------------
The main research questions were centred around factors that could affect utilization of a personal health informatics application through intermediaries, and also the effectiveness of gamification in increasing both engagement of intermediaries and collaboration between members of a pair in intermediated use. This chapter revisits the research questions and presents a discussion of how they were addressed. It also summarizes on takeaways which  are regarded as design considerations for motivational affordances in intermediated use context of a self-monitoring application for promotion of healthy behaviours. These design considerations revealed social factors that could contribute to success of an intervention such as the one in this study, and motivational strategies that could be utilized in order to keep both intermediaries and beneficiaries engaged with a personal health informatics (self-monitoring) system/application.

\section{Discussion on Research Questions}
The main focus of this research was to uncover how social factors and persuasive systems' inspired motivational affordances could impact intermediated use of personal health self-monitoring applications. Intermediated technology use is an interaction model that is prevalent in contexts of low-income communities of developing world. Many health self-monitoring applications are designed for personal use with their motivational affordances targeting only direct users; hence such motivational affordances have not been explored in the context of intermediated use. Therefore, in order to understand design requirements for motivation affordances targeting intermediated use this research aimed at providing answers for the following research questions.

\textbf{RQ1}: What is the role of social-technical settings in intermediated use of a gamified self-monitoring application targeting promotion of healthily eating and physical activity? 

The aforementioned research question had two sub-questions. The first sub-question aimed at identifying prerequisite factors that could affect intermediated use and the second sub-question aimed at exploring the extent to which an understanding of the identified factors is important in the context of intermediated use. In order to provide answers to those research sub-questions, a series of studies were conducted. These studies included: one contextual enquiry (Chapters \ref{contextualenqchapter}) and two consecutive evaluations of two versions of the prototype (Chapters \ref{prototype1chapter} and \ref{prototype2chapter}). 

The most vivid factors that were manifested by aforementioned studies include but not limited to: social relationship; collaborative reflection from a shared device; and motivational affordances  either from the app or socially construed as the result of using the app. Prior social relationship was instrumental in facilitation of negotiation for interaction between members of a pair (an intermediary user and a beneficiary user). Before an interaction took place there was always a negotiation for interaction; hence presence or absence of a prior social relationship was a determinant of whether a negotiation would be successful or not.

Prior social relationship facilitated also a collaborative reflection. Negotiations would be initiated within a pair by either member of a pair. Once a negotiation to initiate interaction was successful then intermediary users navigated through the app. In the process of navigating through the app, anything they found interesting would be shared with their respective beneficiary users. This would spark a conversation between the two users (members of a pair). As the result of sharing information between members of a pair there was a collaborative reflection. 

Informative evaluations revealed that involving children who are family members is the key to success of this kind of an intervention, and by having an app running on a shared device had increased tendency of members of a pair to reflect collaboratively. In such contexts it was no longer just the matter of help seeking and help giving but more of a collaborative effort towards a joint goal. The idea of collaborative interfaces for health information within family settings has been explored in computer supported collaborative work (CSCW) literature. \cite{colineau2011motivating} designed a system to support a family to select a collective health goal and receive feedbacks that entailed comparisons between families. Their system was found to encourage members from within a family or members of different families, to work together and in particular to help each other in finding ways to live a healthily lifestyle. 

Therefore, family settings may provide an idyllic opportunity for members to discuss healthy issues collaboratively. Collaboration between a parent and a child or close family members had a positive impact on child's perception as some intermediaries shared testimonies about their habituation of skills on eating healthy. In addition, intermediaries in some cases logged their data about meals because what they ate was not different from what had been eaten by their respective beneficiaries. A study by~\cite{grimes2009toward} identified four key areas of consideration in which sharing of, and reflection on, health information can be leveraged within family context as follows: (1) overlaps of routines between family members through shared meals, space, etc which can provide opportunities for collaborative data logging and reflection among family members; (2) sharing is done at the expense of balancing competing values of openness, caring, and modelling with the value of protection; (3) understanding of sensitivity on comparisons and competition based upon health information in the context of the family as it may have negative consequences; and (4) collaborative sharing of, reflecting on, health information can also foster family's bond. In the context of this research, it was evident that the app had increased the bond between participating family members as majority of them claimed that were interacting more often. This is also demonstrated by playfulness behaviours that were exhibited in the process of sharing information as it was shown in one of the excerpt in chapter \ref{prototype2chapter} (Prototype II):

\userquote{\textbf{Zandiwe}, a beneficiary} {` When she got time, when she is done with her homework she comes and sees the app. And then laughs at me like `Yo yo yo [An interjection for Xhosa speakers to express the feeling of amazement by something] you can walk yo yo yo', like `you walked a lot today' and what what [She was implying to other words said by Lindiwe]''}

In existing work from computer supported collaborative work it appears the emphasis is on parents trying to model health behaviors of their children.  For a instance in a study by ~\cite{saksono2015spaceship}, a collaborative exergame was developed in order to support both parents and kids to exercise together. Although their goal was to help kids learn from their parents, the collaborative environment was beneficial to both parents and children.

In the context of this research it was peculiar that children were attempting to nudge their parents to live healthily. Therefore, it is not about only a parent attempting to guide his/her child also a child could become a facilitator to guide a parent about healthy choices. This was mediated by an existing familial relationship. In addition to that in most cases of where pairs consisted of a parent working with a child, an intermediary had a tendency of realizing a rationale in fulfilling  requests for interaction from their respective beneficiary users even in cases where intermediaries felt their autonomy was being violated. Empathy led to such intermediaries becoming accountable to the well-being of the people they cared about. The bond was further strengthened by the presence of motivational affordances which had a role to play in making intermediaries believe that information on the app was theirs as well and not for only the beneficiary users who were being assisted; as the result those intermediaries were being responsible team players. Therefore, in such situations reflection was done collaboratively and not at the personal level as it is common in existing personal health informatics applications.  However, perceived interest on motivational affordances differed between intermediary users and beneficiary users. Comparison based on abstract things like points were less meaningful to older beneficiaries as they tended to value more on perceived benefits and social support from others. In the absence of interaction among beneficiaries there was a tendency to have less engagement from the side of beneficiaries. Hence strategies that need to be applied for this user group need to take into consideration of availability of social support from people who already know each other. This kind of social support indicated a tendency to increase relatedness among beneficiary users that were reported in evaluation of a prototype in Chapter \ref{prototype2chapter}. One important conclusion out of this finding is the need to have separate persuasive strategies that discern between beneficiaries and intermediaries. Existing game mechanics may work well with intermediaries while on beneficiaries social support should be encouraged in order to leverage motivational affordance provided by social comparison.

In response to the first main research question it can be concluded that motivational affordances could foster collaboration and subsequently a relationship bond of members of a participating pair in an intervention provided that there is a prior social relationship between members of a pair. This implies a combination of motivational affordances and familiar relationships is crucial in making the collaboration more interesting and enjoyable. Therefore, a prior social relationship and perceived motivation affordances were main determinants for two users from two sets (intermediary set, and beneficiary set) to view any efforts to interaction as carried out on the behalf of the respective team and not for a beneficiary user alone from the team.  

The second research is provided below. This aimed at exploring the impact of using gamification as a means to motivate collaboration that leads to intermediated use of a self-monitoring application.

\textbf{RQ2}: How gamification plays a role in motivating intermediated use of self-monitoring application targeting promotion of healthily eating and physical activity?

This research question was broken down into seven research sub-questions as provided below:

\begin{enumerate}[label=\alph*.]
\item What is the impact of gamification in supporting self-determination of intermediary users to engage with a self monitoring application in intermediated use context?
\item What is the impact of gamification in supporting self-determination of beneficiary users to engage with a self-monitoring application in intermediated use context?
\item What is the impact of gamification on motivation of beneficiaries to self monitor diet?
\item What is the impact of gamification on motivation of beneficiaries to self monitor physical activity?
\item What is the impact of gamification in frequency of utilizing the self-monitoring application in intermediated use context?
\item How the presence of gamification affects the relationship between an intermediary user and beneficiary user?
\item To what extent gamification may encourage or discourage internalization of intermediated use behaviour?

\end{enumerate} 

In order to address the aforementioned research sub-questions that contributed to the second main research question, a summative evaluation (Chapter~\ref{summativeevalchapter}) was conducted that compared between a system without gamification and a system with gamification. 

In support for self-determination in research sub-question 2(a), gamification demonstrated a potential to increase perceived competence of intermediary users in using the self-monitoring app with their respective beneficiary users while aspects of perceived autonomy and perceived relatedness didn't show improvement. Many insights on design considerations were highlighted on aspects of autonomy. One most important was freedom to choose which gamification features to participate in and at what level in order to cater for intermediary users with different personalities and skills. On aspects of relatedness, features that promote socialization and relatedness were only effective for users who are not co-located and this resonated with findings from other literature. 

In research sub-questions; 2(b), 2(c), 2(d), all three aspects of self-determination theory in three sub-questions (to use the app, to self-monitor diet, and to self-monitor physical activity) were the same for each respective comparison between logbook and gamification conditions. However, the presence of a self-monitoring app regardless of whether the app has gamified motivational affordances or not, appeared to increase self-determination of beneficiaries in monitoring their health at endline in comparison to baseline. A significant change was observed in self-monitoring of diet since there was less interest or knowledge in tracking of diet while for physical activity participants seemed to already be doing implicit tracking through approximation of amount of physical activity done while doing daily errands. To majority of the beneficiary participants, gamification was of less importance for reasons that have already been highlighted on discussion for the first research question. Reflecting upon all series of user evaluation studies that were conducted, insights reveal that motivational affordances from gamification that were used in this study may work effectively on intermediaries (who in this case were mostly children) and young beneficiaries. 

In the last three sub-questions; 2(e), 2(f), and 2(g), gamification increased frequency of usage through intermediaries as the result, gamification fostered collaboration in cases were both intermediaries and beneficiaries were interested with gamification features. However, this usage and collaboration from Chapter \ref{summativeevalchapter} (Summative Evaluation) appeared to be mostly accounted to introjected internalization and had negative consequences.
 
Despite the positive outcomes from gamification, the negative consequences as the results of increased competition have become an area of concern~\citep{jia2016personality}. It is even more concerning when such competition results into negative consequences in health settings~\citep{grimes2009toward}. To reduce these negative implications of excessive social comparison and competition, there is an emphasis on supporting challenges on the level of ``\emph{task mastery climate}'' rather than on competition that has ``\emph{ego involved climate}'' as the  former foster intrinsic motivation while the latter can harm it~\citep{saksono2015spaceship}. When ``ego is involved'', participants may do things just to maintain their self-worth, and this is equivalent to introjected regulation as postulated by organismic integration theory of SDT\citep{ryan2000:self}. In introjected regulation individuals don't see a value in regulating a behaviour rather they perform it merely for the purpose of outdoing others or maintaining their social status. For instance, the idea of having a leader board in this intervention encourages competition and it affected motivation of intermediary users who didn't do so well. Features such as fish tank and botanical garden appear to promote task mastery. For instance in one scenario reported in chapter \ref{prototype2chapter} (Prototype II),  a beneficiary user was dissatisfied by the look of their garden. 

\userquote{\textbf{Lulama}, an intermediary} {``She (Nokhanyo) saw the garden. The first day she saw just the house and brownish. She
is like `What is this'. I told her. She said `Aha! [Expressing
dissatisfaction]. It must look green and healthy'. And then
she saw the garden again and said `It is looking good.'''}

This important finding suggests that such features could encourage task mastery climate. If designers have to use a leaderboard, they need to be cautionary of negative impacts on users' competence despite its ability to foster relatedness~\citep{sailer2013:psychological}. Leader board can also result into an extreme competition between intermediaries which can result into a negative impact on a relationship by an intermediaries in cases where an intermediaries feel of being let down by their beneficiaries. Such a scenario is exhibited in the following excerpt on Chapter \ref{summativeevalchapter} (Summative Evaluation)
 
\userquote{\textbf{Jenner}, a female beneficiary from Athlone, 45 yrs old} {``Sometimes may be I forget to take the phone when I go walking and he would ask me `did you take the phone with you' Ooh Gosh I forgot.  Because when I walk to Park Town to exercise and sometimes  I am in such a hurry I forget the phone, he will be crossed with me.''} 

In the aforementioned case, an intermediary got angry because of her mother tendency of forgetting to take the pedometer (phone) when she goes out for walking. Therefore this highlights the importance of paying more attention should on features that support task mastery climate. 

In the context of results on Chapter \ref{summativeevalchapter} (Summative Evaluation), the negative impact of social comparison were more manifested in intermediaries. Beneficiaries who participated in evaluation were doing a lot of social comparison seemed not to affect the motivations was not affecting them in a negative way of where in most cases it proved to increase their enjoyment and motivation to engage with the app. Social comparison challenged beneficiaries to continuously set and revise goals of living healthy. Therefore in the context of beneficiaries it showed indications of promoting task mastery climate.  In addition,  \ref{prototype2chapter} (Prototype II) that as longer as beneficiaries were engaged even in a different way outside gamification context, collaboration between the two users even though their goals may be entirely different. Intermediary users may have goals of achieving rewards from game mechanics while beneficiary users may have goal of accumulating more steps or eating healthy to receive social support from among themselves.

In conclusion, it is clear that both gamification and other motivational affordances that were indirectly situated in the app promoted collaboration between participating pairs that had a prior social rapport. In addition, gamification in particular social comparison and competitions, both the one socially construed by users, and the one implemented as an intentional design goal, increase engagement of both intermediary and beneficiary users.
Gamification was effective in-terms of increasing engagement of intermediaries even though it had some challenges and limitations that need to be addressed as it appeared to affect both intrinsic motivation and internalization in the summative evaluation chapter (Chapter \ref{summativeevalchapter}). The number of participants who were active during summative evaluation had increased hence the negative effects of gamification and especially the leaderboard were conspicuous. 

\section{Limitations}
There are several limitations to generalizability of the findings from this research. The first aspect of limitation is on the sample size. Due not to having sufficient resources and difficulties in recruitment of participants, I ended up using convenient sampling and through approach the number of participants was limited. The sample was not probabilistic and was limited in size; hence affected power of the study and in addition technical glitches contributed to further reduction of power of the study. There is a need to repeat the same study with slightly larger sample size using probabilistic approaches.   The second aspect of limitation is that the researcher relied on interviews in capturing all the experiences. An ethnography study and a diary could be useful in capturing important insights on user experience that could not be recalled or revealed by interviews. The third aspect of limitation of this study is that evaluation of motivational affordances was done in a holistic manner; therefore it difficult to discern the impact of individuals users. In addition to that, evaluation of motivational affordances didn't take into consideration of how personality of users could affect the intervention.  

However despite having limitations that affect generalizability, the series of these small studies was important in laying a good foundation towards an understanding of important social dynamics in order to utilize intermediated use in the context of personal health informatics applications. 

\section*{Future Directions}
In order to enhance user experience, support for factors such as task mastery, support for reflection, enhancement of collaboration within a family (intra-families), or inter-families collaboration should be further explored. In addition, factors such as personality of users and different styles of parenting such as authoritative, neglectful, permissive, and authoritarian should be explored in the context of intermediated use of a personal health informatics application. Spatial arrangement of two users and technology also need further exploration in order to support optimal flow. This issue had an effect on flow of both sets of users. 

Other concepts that need to be further explored by future studies include sustained usage over a long period of time. Long term usage and prolonged benefits of personalized apps are still debatable. One study that a conducted a two years trials that compared among three strategies: interactive smartphone application on a cell phone (CP); personal coaching enhanced by smartphone self-monitoring (PC); or handout (pamphlets) as the control group found that smart-phone to be better than control in short time but in long term the results were not different~\citep{svetkey2015cell}. It is argued that usage of these apps should go hand in hand with other conventional strategies. In addition,  long  term engagement is a key challenge in personalized apps for health. Once the novelty effect is worn out users may tend to revert to their old habits. Gamification is not exceptional when it comes to the issue of the novelty effect. In the context of intermediated technology use it may even be more challenging in sustaining engagement since we are dealing with more that one layer of users. This is still a grey area that needs further exploration.
\begin{flushright}
\end{flushright}
 
%\input{Chapters/Chapter5} 
%\input{Chapters/Chapter6} 
%\input{Chapters/Chapter7} 
%
%----------------------------------------------------------------------------------------
%	THESIS CONTENT - APPENDICEShttps://www.facebook.com/
%----------------------------------------------------------------------------------------

\addtocontents{toc}{\vspace{2em}} % Add a gap in the Contents, for aesthetics

\appendix % Cue to tell LaTeX that the following 'chapters' are Appendices

% Include the appendices of the thesis as separate files from the Appendices folder
% Uncomment the lines as you write the Appendices

%% Appendix A

\chapter{Appendix A. Ethics Approval -- Faculty of Health Sciences} % Main appendix title
\clearpage
\label{AppendixA} % For referencing this appendix elsewhere, use \ref{AppendixA}

\lhead{Appendix A. \emph{Ethics Approval -- Faculty of Health Sciences}} % This is for the header on each page - perhaps a shortened title
\includepdf[pages=-,pagecommand={},width=\textwidth,offset=90 -20]{Pdfs/fhsrec.pdf}

%\newcommand{\insertrep}[1]{%
%\hspace{-2.4cm}
%\fbox{\includegraphics[page=1,scale=0.8]{#1}}
%\includegraphics[page=1,scale=0.8]{#1}
%\includepdf[scale=0.75,pages=1,frame]{#1}
%}

%\subsection{Interesting Letter}
%\insertrep{Pdfs/baseline_ben.pdf}
%\begin{center}
%\includepdf[pagecommand={},scale=0.9]{Pdfs/baseline_ben.pdf} 
%\end{center}

%\input{Appendices/AppendixB}
%\input{Appendices/AppendixC}

\addtocontents{toc}{\vspace{2em}} % Add a gap in the Contents, for aesthetics\textbf{}

\backmatter

%----------------------------------------------------------------------------------------
%	BIBLIOGRAPHY
%----------------------------------------------------------------------------------------

\label{Bibliography}

\lhead{\emph{Bibliography}} % Change the page header to say "Bibliography"

\bibliographystyle{unsrtnat} % Use the "unsrtnat" BibTeX style for formatting the Bibliography
%\bibliographystyle{apa}
\bibliography{Bibliography} % The references (bibliography) information are stored in the file named "Bibliography.bib"

\end{document}  